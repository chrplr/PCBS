% Options for packages loaded elsewhere
\PassOptionsToPackage{unicode}{hyperref}
\PassOptionsToPackage{hyphens}{url}
%
\documentclass[
  8pt,
  ignorenonframetext,
]{beamer}
\usepackage{pgfpages}
\setbeamertemplate{caption}[numbered]
\setbeamertemplate{caption label separator}{: }
\setbeamercolor{caption name}{fg=normal text.fg}
\beamertemplatenavigationsymbolsempty
% Prevent slide breaks in the middle of a paragraph
\widowpenalties 1 10000
\raggedbottom
\setbeamertemplate{part page}{
  \centering
  \begin{beamercolorbox}[sep=16pt,center]{part title}
    \usebeamerfont{part title}\insertpart\par
  \end{beamercolorbox}
}
\setbeamertemplate{section page}{
  \centering
  \begin{beamercolorbox}[sep=12pt,center]{part title}
    \usebeamerfont{section title}\insertsection\par
  \end{beamercolorbox}
}
\setbeamertemplate{subsection page}{
  \centering
  \begin{beamercolorbox}[sep=8pt,center]{part title}
    \usebeamerfont{subsection title}\insertsubsection\par
  \end{beamercolorbox}
}
\AtBeginPart{
  \frame{\partpage}
}
\AtBeginSection{
  \ifbibliography
  \else
    \frame{\sectionpage}
  \fi
}
\AtBeginSubsection{
  \frame{\subsectionpage}
}
\usepackage{amsmath,amssymb}
\usepackage{lmodern}
\usepackage{iftex}
\ifPDFTeX
  \usepackage[T1]{fontenc}
  \usepackage[utf8]{inputenc}
  \usepackage{textcomp} % provide euro and other symbols
\else % if luatex or xetex
  \usepackage{unicode-math}
  \defaultfontfeatures{Scale=MatchLowercase}
  \defaultfontfeatures[\rmfamily]{Ligatures=TeX,Scale=1}
\fi
\usetheme[]{CambridgeUS}
% Use upquote if available, for straight quotes in verbatim environments
\IfFileExists{upquote.sty}{\usepackage{upquote}}{}
\IfFileExists{microtype.sty}{% use microtype if available
  \usepackage[]{microtype}
  \UseMicrotypeSet[protrusion]{basicmath} % disable protrusion for tt fonts
}{}
\makeatletter
\@ifundefined{KOMAClassName}{% if non-KOMA class
  \IfFileExists{parskip.sty}{%
    \usepackage{parskip}
  }{% else
    \setlength{\parindent}{0pt}
    \setlength{\parskip}{6pt plus 2pt minus 1pt}}
}{% if KOMA class
  \KOMAoptions{parskip=half}}
\makeatother
\usepackage{xcolor}
\newif\ifbibliography
\usepackage{color}
\usepackage{fancyvrb}
\newcommand{\VerbBar}{|}
\newcommand{\VERB}{\Verb[commandchars=\\\{\}]}
\DefineVerbatimEnvironment{Highlighting}{Verbatim}{commandchars=\\\{\}}
% Add ',fontsize=\small' for more characters per line
\usepackage{framed}
\definecolor{shadecolor}{RGB}{248,248,248}
\newenvironment{Shaded}{\begin{snugshade}}{\end{snugshade}}
\newcommand{\AlertTok}[1]{\textcolor[rgb]{0.94,0.16,0.16}{#1}}
\newcommand{\AnnotationTok}[1]{\textcolor[rgb]{0.56,0.35,0.01}{\textbf{\textit{#1}}}}
\newcommand{\AttributeTok}[1]{\textcolor[rgb]{0.77,0.63,0.00}{#1}}
\newcommand{\BaseNTok}[1]{\textcolor[rgb]{0.00,0.00,0.81}{#1}}
\newcommand{\BuiltInTok}[1]{#1}
\newcommand{\CharTok}[1]{\textcolor[rgb]{0.31,0.60,0.02}{#1}}
\newcommand{\CommentTok}[1]{\textcolor[rgb]{0.56,0.35,0.01}{\textit{#1}}}
\newcommand{\CommentVarTok}[1]{\textcolor[rgb]{0.56,0.35,0.01}{\textbf{\textit{#1}}}}
\newcommand{\ConstantTok}[1]{\textcolor[rgb]{0.00,0.00,0.00}{#1}}
\newcommand{\ControlFlowTok}[1]{\textcolor[rgb]{0.13,0.29,0.53}{\textbf{#1}}}
\newcommand{\DataTypeTok}[1]{\textcolor[rgb]{0.13,0.29,0.53}{#1}}
\newcommand{\DecValTok}[1]{\textcolor[rgb]{0.00,0.00,0.81}{#1}}
\newcommand{\DocumentationTok}[1]{\textcolor[rgb]{0.56,0.35,0.01}{\textbf{\textit{#1}}}}
\newcommand{\ErrorTok}[1]{\textcolor[rgb]{0.64,0.00,0.00}{\textbf{#1}}}
\newcommand{\ExtensionTok}[1]{#1}
\newcommand{\FloatTok}[1]{\textcolor[rgb]{0.00,0.00,0.81}{#1}}
\newcommand{\FunctionTok}[1]{\textcolor[rgb]{0.00,0.00,0.00}{#1}}
\newcommand{\ImportTok}[1]{#1}
\newcommand{\InformationTok}[1]{\textcolor[rgb]{0.56,0.35,0.01}{\textbf{\textit{#1}}}}
\newcommand{\KeywordTok}[1]{\textcolor[rgb]{0.13,0.29,0.53}{\textbf{#1}}}
\newcommand{\NormalTok}[1]{#1}
\newcommand{\OperatorTok}[1]{\textcolor[rgb]{0.81,0.36,0.00}{\textbf{#1}}}
\newcommand{\OtherTok}[1]{\textcolor[rgb]{0.56,0.35,0.01}{#1}}
\newcommand{\PreprocessorTok}[1]{\textcolor[rgb]{0.56,0.35,0.01}{\textit{#1}}}
\newcommand{\RegionMarkerTok}[1]{#1}
\newcommand{\SpecialCharTok}[1]{\textcolor[rgb]{0.00,0.00,0.00}{#1}}
\newcommand{\SpecialStringTok}[1]{\textcolor[rgb]{0.31,0.60,0.02}{#1}}
\newcommand{\StringTok}[1]{\textcolor[rgb]{0.31,0.60,0.02}{#1}}
\newcommand{\VariableTok}[1]{\textcolor[rgb]{0.00,0.00,0.00}{#1}}
\newcommand{\VerbatimStringTok}[1]{\textcolor[rgb]{0.31,0.60,0.02}{#1}}
\newcommand{\WarningTok}[1]{\textcolor[rgb]{0.56,0.35,0.01}{\textbf{\textit{#1}}}}
\setlength{\emergencystretch}{3em} % prevent overfull lines
\providecommand{\tightlist}{%
  \setlength{\itemsep}{0pt}\setlength{\parskip}{0pt}}
\setcounter{secnumdepth}{-\maxdimen} % remove section numbering
\let\verbatim\undefined
\let\verbatimend\undefined
\usepackage{listings}
\lstnewenvironment{verbatim}{\lstset{breaklines=true,basicstyle=\ttfamily\footnotesize}}{}
\ifLuaTeX
  \usepackage{selnolig}  % disable illegal ligatures
\fi
\IfFileExists{bookmark.sty}{\usepackage{bookmark}}{\usepackage{hyperref}}
\IfFileExists{xurl.sty}{\usepackage{xurl}}{} % add URL line breaks if available
\urlstyle{same} % disable monospaced font for URLs
\hypersetup{
  pdftitle={Intro to programming 8},
  hidelinks,
  pdfcreator={LaTeX via pandoc}}

\title{Intro to programming 8}
\author{Henri Vandendriessche\\
\href{mailto:henri.vandendriessche@ens.fr}{\nolinkurl{henri.vandendriessche@ens.fr}}}
\date{2022-10-25}

\begin{document}
\frame{\titlepage}

\begin{frame}{Where are we now}
\protect\hypertarget{where-are-we-now}{}
\begin{itemize}
\item
  Now that we write programs more and more complicated, we end up
  encountering more more and more complex situation.
\item
  And more and more complicated bugs\ldots{}
\end{itemize}

Your computer will do only what you tell it to do; it won't read your
mind and do what you intended it to do. Even professional programmers
create bugs all the time, so don't feel discouraged if your program has
a problem.

Fortunately, there are a few tools and techniques to identify what
exactly your code is doing and where it's going wrong. First, you will
look at logging and assertions, two features that can help you detect
bugs early. In general, the earlier you catch bugs, the easier they will
be to fix.
\end{frame}

\begin{frame}{Debugging}
\protect\hypertarget{debugging}{}
\end{frame}

\begin{frame}{The easiest debugging rule}
\protect\hypertarget{the-easiest-debugging-rule}{}
\begin{itemize}[<+->]
\tightlist
\item
  When your program do what you asked (what you wrote) but not what you
  wanted\ldots{}
\end{itemize}

\begin{itemize}[<+->]
\tightlist
\item
  The simplest and easiest way to see if there is a problem in you
  program is to check your variables at every key points of your
  program:

  \begin{itemize}[<+->]
  \tightlist
  \item
    When you perform a operation on your variable
  \item
    At the end of loop
  \item
    At the end of a function if there is a return statement
  \item
    When you import data from a file
  \end{itemize}
\end{itemize}

\begin{itemize}[<+->]
\tightlist
\item
  What is the best way to check on your variables and their types ?

  \begin{itemize}[<+->]
  \tightlist
  \item
    print it: \textbf{print(your\_variable)}
  \item
    print the type of your variable:
    \textbf{print(type(your\_variable))}
  \end{itemize}
\end{itemize}

\begin{itemize}[<+->]
\tightlist
\item
  \textbf{print()} can be useful for simple check but please don't debug
  your all your script with print()
\end{itemize}

\begin{itemize}[<+->]
\tightlist
\item
  It's the level 0 of debugging
\end{itemize}
\end{frame}

\begin{frame}[fragile]{Try and except statements 1/5}
\protect\hypertarget{try-and-except-statements-15}{}
\begin{itemize}[<+->]
\tightlist
\item
  If you have an error in your script, the execution is stopped.
\end{itemize}

\begin{itemize}[<+->]
\item
  Example: What's wrong in the following script:

\begin{Shaded}
\begin{Highlighting}[]
\KeywordTok{def}\NormalTok{ isDivided(divisor):}
    \ControlFlowTok{return} \DecValTok{42} \OperatorTok{/}\NormalTok{ divisor}

\BuiltInTok{print}\NormalTok{(isDivided(}\DecValTok{2}\NormalTok{))}
\BuiltInTok{print}\NormalTok{(isDivided(}\DecValTok{12}\NormalTok{))}
\BuiltInTok{print}\NormalTok{(isDivided(}\DecValTok{0}\NormalTok{))}
\BuiltInTok{print}\NormalTok{(isDivided(}\DecValTok{3}\NormalTok{))}
\end{Highlighting}
\end{Shaded}
\end{itemize}
\end{frame}

\begin{frame}[fragile]{Try and except statements 2/5}
\protect\hypertarget{try-and-except-statements-25}{}
\begin{itemize}
\item
  If you have an error in your script, the execution is stropped.
\item
  Example: What's wrong in the following script:

\begin{Shaded}
\begin{Highlighting}[]
\KeywordTok{def}\NormalTok{ isDivided(divisor):}
    \ControlFlowTok{return} \DecValTok{42} \OperatorTok{/}\NormalTok{ divisor}

\BuiltInTok{print}\NormalTok{(isDivided(}\DecValTok{2}\NormalTok{))}
\end{Highlighting}
\end{Shaded}

\begin{verbatim}
## 21.0
\end{verbatim}

\begin{Shaded}
\begin{Highlighting}[]
\BuiltInTok{print}\NormalTok{(isDivided(}\DecValTok{12}\NormalTok{))}
\end{Highlighting}
\end{Shaded}

\begin{verbatim}
## 3.5
\end{verbatim}

\begin{Shaded}
\begin{Highlighting}[]
\BuiltInTok{print}\NormalTok{(isDivided(}\DecValTok{0}\NormalTok{))}
\end{Highlighting}
\end{Shaded}

\begin{verbatim}
## Error in py_call_impl(callable, dots$args, dots$keywords): ZeroDivisionError: division by zero
\end{verbatim}
\end{itemize}
\end{frame}

\begin{frame}[fragile]{Try and except statements 3/5}
\protect\hypertarget{try-and-except-statements-35}{}
\begin{itemize}
\tightlist
\item
  But you can still have your way around this error:

  \begin{itemize}
  \tightlist
  \item
    \textbf{try : }
  \item
    \textbf{except \ldots{} : }
  \end{itemize}
\end{itemize}

\begin{Shaded}
\begin{Highlighting}[]
\KeywordTok{def}\NormalTok{ isDivided(divisor):}
    \ControlFlowTok{try}\NormalTok{:}
      \ControlFlowTok{return} \DecValTok{42} \OperatorTok{/}\NormalTok{ divisor}
    \ControlFlowTok{except} \PreprocessorTok{ZeroDivisionError}\NormalTok{:}
      \BuiltInTok{print}\NormalTok{(}\StringTok{"What have I done again..."}\NormalTok{)}

\BuiltInTok{print}\NormalTok{(isDivided(}\DecValTok{2}\NormalTok{))}
\end{Highlighting}
\end{Shaded}

\begin{verbatim}
## 21.0
\end{verbatim}

\begin{Shaded}
\begin{Highlighting}[]
\BuiltInTok{print}\NormalTok{(isDivided(}\DecValTok{12}\NormalTok{))}
\end{Highlighting}
\end{Shaded}

\begin{verbatim}
## 3.5
\end{verbatim}

\begin{Shaded}
\begin{Highlighting}[]
\BuiltInTok{print}\NormalTok{(isDivided(}\DecValTok{0}\NormalTok{))}
\end{Highlighting}
\end{Shaded}

\begin{verbatim}
## What have I done again...
## None
\end{verbatim}

\begin{Shaded}
\begin{Highlighting}[]
\BuiltInTok{print}\NormalTok{(isDivided(}\DecValTok{3}\NormalTok{))}
\end{Highlighting}
\end{Shaded}

\begin{verbatim}
## 14.0
\end{verbatim}
\end{frame}

\begin{frame}[fragile]{Try and except statements 4/5}
\protect\hypertarget{try-and-except-statements-45}{}
\begin{itemize}[<+->]
\item
  You can as well include the call of your function in the try

\begin{Shaded}
\begin{Highlighting}[]
\KeywordTok{def}\NormalTok{ isDivided(divisor):}
  \ControlFlowTok{return} \DecValTok{42} \OperatorTok{/}\NormalTok{ divisor}

\ControlFlowTok{try}\NormalTok{:}
  \BuiltInTok{print}\NormalTok{(isDivided(}\DecValTok{2}\NormalTok{))}
  \BuiltInTok{print}\NormalTok{(isDivided(}\DecValTok{12}\NormalTok{))}
  \BuiltInTok{print}\NormalTok{(isDivided(}\DecValTok{0}\NormalTok{))}
  \BuiltInTok{print}\NormalTok{(isDivided(}\DecValTok{3}\NormalTok{))}

\ControlFlowTok{except} \PreprocessorTok{ZeroDivisionError}\NormalTok{:}
  \BuiltInTok{print}\NormalTok{(}\StringTok{"What have I done again..."}\NormalTok{)}
\end{Highlighting}
\end{Shaded}

\begin{verbatim}
## 21.0
## 3.5
## What have I done again...
\end{verbatim}
\end{itemize}

\begin{itemize}[<+->]
\tightlist
\item
  Note that \textbf{print(isDivided(3))} is not executed. Once the
  execution jumps in the except statement, it does not goes back to the
  try clause. Instead, it just continues moving down the program as
  normal
\end{itemize}
\end{frame}

\begin{frame}{Try and except statements 5/5}
\protect\hypertarget{try-and-except-statements-55}{}
\begin{itemize}
\tightlist
\item
  Try except is useful:

  \begin{itemize}
  \tightlist
  \item
    For making some checks on your program flow
  \item
    To get a (hopefully) clearer (or more adapted) error message than
    what python can provide
  \end{itemize}
\end{itemize}
\end{frame}

\begin{frame}[fragile]{Raising Exceptions 1/2}
\protect\hypertarget{raising-exceptions-12}{}
\begin{itemize}
\item
  Python raises exceptions every time it attempts to execute an invalid
  code
\item
  Exceptions are raised this way:

  \begin{itemize}
  \tightlist
  \item
    \textbf{raise} keyword
  \item
    \textbf{Exception()} function
  \item
    A useful sentence that will help you understand the problem in the
    Exception function
  \end{itemize}
\end{itemize}

\begin{Shaded}
\begin{Highlighting}[]
\ControlFlowTok{raise} \PreprocessorTok{Exception}\NormalTok{(}\StringTok{\textquotesingle{}Ah Shit, Here We Go Again: another day another bug\textquotesingle{}}\NormalTok{)}
\end{Highlighting}
\end{Shaded}

\begin{verbatim}
## Error in py_call_impl(callable, dots$args, dots$keywords): Exception: Ah Shit, Here We Go Again: another day another bug
\end{verbatim}

\begin{itemize}
\item
  The \textbf{try} and \textbf{except} statement allows us to handle
  those exceptions if we anticipate them
\item
  Without the \textbf{try} and \textbf{except} statement the program
  stops at the first exception raised
\end{itemize}
\end{frame}

\begin{frame}[fragile]{Raising Exceptions 2/2}
\protect\hypertarget{raising-exceptions-22}{}
\begin{itemize}
\item
  When / how to raise exception ?
\item
  Exception can be passed as argument or returned at the end of a
  function
\end{itemize}

\begin{Shaded}
\begin{Highlighting}[]
\KeywordTok{def}\NormalTok{ doBullshit():}
  \ControlFlowTok{raise} \PreprocessorTok{Exception}\NormalTok{(}\StringTok{\textquotesingle{}I did bullshit\textquotesingle{}}\NormalTok{)}

\ControlFlowTok{try}\NormalTok{:}
\NormalTok{  doBullshit()}
\ControlFlowTok{except} \PreprocessorTok{Exception} \ImportTok{as}\NormalTok{ err:}
  \BuiltInTok{print}\NormalTok{(}\StringTok{"Ooops, "}\NormalTok{, }\BuiltInTok{str}\NormalTok{(err) )}
    
\end{Highlighting}
\end{Shaded}

\begin{verbatim}
## Ooops,  I did bullshit
\end{verbatim}
\end{frame}

\begin{frame}[fragile]{Getting the Traceback as a String 1/2}
\protect\hypertarget{getting-the-traceback-as-a-string-12}{}
\begin{itemize}
\item
  Getting the information of your error.
\item
  When your program crashes you always have an error with some
  information like:

  \begin{itemize}
  \tightlist
  \item
    The line of the error / the differents lines if your program uses
    several files
  \item
    The error message
  \item
    The function / the sequence of functions involved (i,e, the call
    stack)
  \end{itemize}
\item
  All of that is called the \textbf{traceback}
\item
  Example:
\end{itemize}

\begin{Shaded}
\begin{Highlighting}[]
\KeywordTok{def}\NormalTok{ callErrorTest():}
\NormalTok{  errorTest()}

\KeywordTok{def}\NormalTok{ errorTest():}
  \ControlFlowTok{raise} \PreprocessorTok{Exception}\NormalTok{(}\StringTok{\textquotesingle{}FATAL ERROR\textquotesingle{}}\NormalTok{)}

\NormalTok{callErrorTest()}
    
\end{Highlighting}
\end{Shaded}

\begin{verbatim}
## Error in py_call_impl(callable, dots$args, dots$keywords): Exception: FATAL ERROR
\end{verbatim}

\begin{itemize}
\tightlist
\item
  Those informations are here to help you locate and understand you
  error.
\end{itemize}
\end{frame}

\begin{frame}[fragile]{Getting the Traceback as a String 2/2}
\protect\hypertarget{getting-the-traceback-as-a-string-22}{}
\begin{itemize}
\item
  Instead of just prompting it on your terminal, you can have access to
  your traceback using: \textbf{traceback.format\_exc()}
\item
  That way you can obtain your traceback information as a string
\item
  You'll need the tracback module to access the function.
\item
  It can be useful if you want to keep track of an error and write the
  info in a file. That way you keep it for later when you'll be mentally
  prepared to debug your code.
\end{itemize}

\begin{Shaded}
\begin{Highlighting}[]
\ImportTok{import}\NormalTok{ traceback}

\ControlFlowTok{try}\NormalTok{:}
  \ControlFlowTok{raise} \PreprocessorTok{Exception}\NormalTok{(}\StringTok{\textquotesingle{}FATAL ERROR\textquotesingle{}}\NormalTok{)}

\ControlFlowTok{except}\NormalTok{:}
\NormalTok{  errorFile }\OperatorTok{=} \BuiltInTok{open}\NormalTok{(}\StringTok{\textquotesingle{}errorInfo.txt\textquotesingle{}}\NormalTok{, }\StringTok{\textquotesingle{}w\textquotesingle{}}\NormalTok{)}
\NormalTok{  errorFile.write(traceback.format\_exc())}
\NormalTok{  errorFile.close()}
  \BuiltInTok{print}\NormalTok{(}\StringTok{\textquotesingle{}Don}\CharTok{\textbackslash{}\textquotesingle{}}\StringTok{t have time now to debug but all info are in errorInfo.txt\textquotesingle{}}\NormalTok{)}
    
\end{Highlighting}
\end{Shaded}

\begin{verbatim}
## 97
## Don't have time now to debug but all info are in errorInfo.txt
\end{verbatim}
\end{frame}

\begin{frame}{Assertions}
\protect\hypertarget{assertions}{}
An assertion is a sanity check to make sure your code isn't doing
something obviously wrong. These sanity checks are performed by assert
statements. If the sanity check fails, then an AssertionError exception
is raised. In code, an assert statement consists of the following:
\end{frame}

\begin{frame}{Logging}
\protect\hypertarget{logging}{}
\begin{itemize}
\tightlist
\item
  Why logging better than print
\end{itemize}
\end{frame}

\begin{frame}{Python debugger: PDB}
\protect\hypertarget{python-debugger-pdb}{}
\end{frame}

\end{document}
