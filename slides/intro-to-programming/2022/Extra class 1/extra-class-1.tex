% Options for packages loaded elsewhere
\PassOptionsToPackage{unicode}{hyperref}
\PassOptionsToPackage{hyphens}{url}
%
\documentclass[
  8pt,
  ignorenonframetext,
]{beamer}
\usepackage{pgfpages}
\setbeamertemplate{caption}[numbered]
\setbeamertemplate{caption label separator}{: }
\setbeamercolor{caption name}{fg=normal text.fg}
\beamertemplatenavigationsymbolsempty
% Prevent slide breaks in the middle of a paragraph
\widowpenalties 1 10000
\raggedbottom
\setbeamertemplate{part page}{
  \centering
  \begin{beamercolorbox}[sep=16pt,center]{part title}
    \usebeamerfont{part title}\insertpart\par
  \end{beamercolorbox}
}
\setbeamertemplate{section page}{
  \centering
  \begin{beamercolorbox}[sep=12pt,center]{part title}
    \usebeamerfont{section title}\insertsection\par
  \end{beamercolorbox}
}
\setbeamertemplate{subsection page}{
  \centering
  \begin{beamercolorbox}[sep=8pt,center]{part title}
    \usebeamerfont{subsection title}\insertsubsection\par
  \end{beamercolorbox}
}
\AtBeginPart{
  \frame{\partpage}
}
\AtBeginSection{
  \ifbibliography
  \else
    \frame{\sectionpage}
  \fi
}
\AtBeginSubsection{
  \frame{\subsectionpage}
}
\usepackage{amsmath,amssymb}
\usepackage{lmodern}
\usepackage{iftex}
\ifPDFTeX
  \usepackage[T1]{fontenc}
  \usepackage[utf8]{inputenc}
  \usepackage{textcomp} % provide euro and other symbols
\else % if luatex or xetex
  \usepackage{unicode-math}
  \defaultfontfeatures{Scale=MatchLowercase}
  \defaultfontfeatures[\rmfamily]{Ligatures=TeX,Scale=1}
\fi
\usetheme[]{CambridgeUS}
% Use upquote if available, for straight quotes in verbatim environments
\IfFileExists{upquote.sty}{\usepackage{upquote}}{}
\IfFileExists{microtype.sty}{% use microtype if available
  \usepackage[]{microtype}
  \UseMicrotypeSet[protrusion]{basicmath} % disable protrusion for tt fonts
}{}
\makeatletter
\@ifundefined{KOMAClassName}{% if non-KOMA class
  \IfFileExists{parskip.sty}{%
    \usepackage{parskip}
  }{% else
    \setlength{\parindent}{0pt}
    \setlength{\parskip}{6pt plus 2pt minus 1pt}}
}{% if KOMA class
  \KOMAoptions{parskip=half}}
\makeatother
\usepackage{xcolor}
\newif\ifbibliography
\usepackage{color}
\usepackage{fancyvrb}
\newcommand{\VerbBar}{|}
\newcommand{\VERB}{\Verb[commandchars=\\\{\}]}
\DefineVerbatimEnvironment{Highlighting}{Verbatim}{commandchars=\\\{\}}
% Add ',fontsize=\small' for more characters per line
\usepackage{framed}
\definecolor{shadecolor}{RGB}{248,248,248}
\newenvironment{Shaded}{\begin{snugshade}}{\end{snugshade}}
\newcommand{\AlertTok}[1]{\textcolor[rgb]{0.94,0.16,0.16}{#1}}
\newcommand{\AnnotationTok}[1]{\textcolor[rgb]{0.56,0.35,0.01}{\textbf{\textit{#1}}}}
\newcommand{\AttributeTok}[1]{\textcolor[rgb]{0.77,0.63,0.00}{#1}}
\newcommand{\BaseNTok}[1]{\textcolor[rgb]{0.00,0.00,0.81}{#1}}
\newcommand{\BuiltInTok}[1]{#1}
\newcommand{\CharTok}[1]{\textcolor[rgb]{0.31,0.60,0.02}{#1}}
\newcommand{\CommentTok}[1]{\textcolor[rgb]{0.56,0.35,0.01}{\textit{#1}}}
\newcommand{\CommentVarTok}[1]{\textcolor[rgb]{0.56,0.35,0.01}{\textbf{\textit{#1}}}}
\newcommand{\ConstantTok}[1]{\textcolor[rgb]{0.00,0.00,0.00}{#1}}
\newcommand{\ControlFlowTok}[1]{\textcolor[rgb]{0.13,0.29,0.53}{\textbf{#1}}}
\newcommand{\DataTypeTok}[1]{\textcolor[rgb]{0.13,0.29,0.53}{#1}}
\newcommand{\DecValTok}[1]{\textcolor[rgb]{0.00,0.00,0.81}{#1}}
\newcommand{\DocumentationTok}[1]{\textcolor[rgb]{0.56,0.35,0.01}{\textbf{\textit{#1}}}}
\newcommand{\ErrorTok}[1]{\textcolor[rgb]{0.64,0.00,0.00}{\textbf{#1}}}
\newcommand{\ExtensionTok}[1]{#1}
\newcommand{\FloatTok}[1]{\textcolor[rgb]{0.00,0.00,0.81}{#1}}
\newcommand{\FunctionTok}[1]{\textcolor[rgb]{0.00,0.00,0.00}{#1}}
\newcommand{\ImportTok}[1]{#1}
\newcommand{\InformationTok}[1]{\textcolor[rgb]{0.56,0.35,0.01}{\textbf{\textit{#1}}}}
\newcommand{\KeywordTok}[1]{\textcolor[rgb]{0.13,0.29,0.53}{\textbf{#1}}}
\newcommand{\NormalTok}[1]{#1}
\newcommand{\OperatorTok}[1]{\textcolor[rgb]{0.81,0.36,0.00}{\textbf{#1}}}
\newcommand{\OtherTok}[1]{\textcolor[rgb]{0.56,0.35,0.01}{#1}}
\newcommand{\PreprocessorTok}[1]{\textcolor[rgb]{0.56,0.35,0.01}{\textit{#1}}}
\newcommand{\RegionMarkerTok}[1]{#1}
\newcommand{\SpecialCharTok}[1]{\textcolor[rgb]{0.00,0.00,0.00}{#1}}
\newcommand{\SpecialStringTok}[1]{\textcolor[rgb]{0.31,0.60,0.02}{#1}}
\newcommand{\StringTok}[1]{\textcolor[rgb]{0.31,0.60,0.02}{#1}}
\newcommand{\VariableTok}[1]{\textcolor[rgb]{0.00,0.00,0.00}{#1}}
\newcommand{\VerbatimStringTok}[1]{\textcolor[rgb]{0.31,0.60,0.02}{#1}}
\newcommand{\WarningTok}[1]{\textcolor[rgb]{0.56,0.35,0.01}{\textbf{\textit{#1}}}}
\setlength{\emergencystretch}{3em} % prevent overfull lines
\providecommand{\tightlist}{%
  \setlength{\itemsep}{0pt}\setlength{\parskip}{0pt}}
\setcounter{secnumdepth}{-\maxdimen} % remove section numbering
\let\verbatim\undefined
\let\verbatimend\undefined
\usepackage{listings}
\lstnewenvironment{verbatim}{\lstset{breaklines=true,basicstyle=\ttfamily\footnotesize}}{}
\ifLuaTeX
  \usepackage{selnolig}  % disable illegal ligatures
\fi
\IfFileExists{bookmark.sty}{\usepackage{bookmark}}{\usepackage{hyperref}}
\IfFileExists{xurl.sty}{\usepackage{xurl}}{} % add URL line breaks if available
\urlstyle{same} % disable monospaced font for URLs
\hypersetup{
  pdftitle={Extra class 1},
  hidelinks,
  pdfcreator={LaTeX via pandoc}}

\title{Extra class 1}
\author{Henri Vandendriessche\\
\href{mailto:henri.vandendriessche@ens.fr}{\nolinkurl{henri.vandendriessche@ens.fr}}}
\date{2023-01-03}

\begin{document}
\frame{\titlepage}

\begin{frame}[fragile]{MCQ data variable}
\protect\hypertarget{mcq-data-variable}{}
\begin{itemize}
\tightlist
\item
  Remove incorrect characters in the name of the variable
\end{itemize}

\begin{Shaded}
\begin{Highlighting}[]
\DecValTok{1}\ErrorTok{\_of\_my}\OperatorTok{{-}}\NormalTok{variable}\OperatorTok{!} \OperatorTok{=} \StringTok{"test"}
\end{Highlighting}
\end{Shaded}
\end{frame}

\begin{frame}[fragile]{MCQ boolean 1}
\protect\hypertarget{mcq-boolean-1}{}
\begin{itemize}
\tightlist
\item
  What is the value of the following statement ?
\end{itemize}

\begin{Shaded}
\begin{Highlighting}[]
\BuiltInTok{print}\NormalTok{(}\DecValTok{5}\OperatorTok{\textgreater{}}\DecValTok{1}\NormalTok{)}
\end{Highlighting}
\end{Shaded}
\end{frame}

\begin{frame}[fragile]{MCQ boolean 1}
\protect\hypertarget{mcq-boolean-1-1}{}
\begin{itemize}
\tightlist
\item
  What is the value of the following statement ?
\end{itemize}

\begin{Shaded}
\begin{Highlighting}[]
\BuiltInTok{print}\NormalTok{(}\DecValTok{5}\OperatorTok{\textgreater{}}\DecValTok{1}\NormalTok{)}
\end{Highlighting}
\end{Shaded}

\begin{verbatim}
## True
\end{verbatim}
\end{frame}

\begin{frame}[fragile]{MCQ boolean 2}
\protect\hypertarget{mcq-boolean-2}{}
\begin{itemize}
\tightlist
\item
  What is the value of the following statement ?
\end{itemize}

\begin{Shaded}
\begin{Highlighting}[]
\BuiltInTok{print}\NormalTok{(}\DecValTok{5} \OperatorTok{==} \DecValTok{1}\NormalTok{)}
\end{Highlighting}
\end{Shaded}
\end{frame}

\begin{frame}[fragile]{MCQ boolean 2}
\protect\hypertarget{mcq-boolean-2-1}{}
\begin{itemize}
\tightlist
\item
  What is the value of the following statement ?
\end{itemize}

\begin{Shaded}
\begin{Highlighting}[]
\BuiltInTok{print}\NormalTok{(}\DecValTok{5} \OperatorTok{==} \DecValTok{1}\NormalTok{)}
\end{Highlighting}
\end{Shaded}

\begin{verbatim}
## False
\end{verbatim}
\end{frame}

\begin{frame}[fragile]{MCQ boolean 3}
\protect\hypertarget{mcq-boolean-3}{}
\begin{itemize}
\tightlist
\item
  What is the value of the following statement ?
\end{itemize}

\begin{Shaded}
\begin{Highlighting}[]
\BuiltInTok{print}\NormalTok{(}\DecValTok{5} \OperatorTok{\textless{}} \DecValTok{1}\NormalTok{)}
\end{Highlighting}
\end{Shaded}
\end{frame}

\begin{frame}[fragile]{MCQ boolean 3}
\protect\hypertarget{mcq-boolean-3-1}{}
\begin{itemize}
\tightlist
\item
  What is the value of the following statement ?
\end{itemize}

\begin{Shaded}
\begin{Highlighting}[]
\BuiltInTok{print}\NormalTok{(}\DecValTok{5} \OperatorTok{\textless{}} \DecValTok{1}\NormalTok{)}
\end{Highlighting}
\end{Shaded}

\begin{verbatim}
## False
\end{verbatim}
\end{frame}

\begin{frame}[fragile]{MCQ boolean 4}
\protect\hypertarget{mcq-boolean-4}{}
\begin{itemize}
\tightlist
\item
  What is the value of the following statement ?
\end{itemize}

\begin{Shaded}
\begin{Highlighting}[]
\BuiltInTok{print}\NormalTok{(}\BuiltInTok{bool}\NormalTok{(}\StringTok{"abc"}\NormalTok{))}
\end{Highlighting}
\end{Shaded}
\end{frame}

\begin{frame}[fragile]{MCQ boolean 4}
\protect\hypertarget{mcq-boolean-4-1}{}
\begin{itemize}
\tightlist
\item
  What is the value of the following statement ?
\end{itemize}

\begin{Shaded}
\begin{Highlighting}[]
\BuiltInTok{print}\NormalTok{(}\BuiltInTok{bool}\NormalTok{(}\StringTok{"abc"}\NormalTok{))}
\end{Highlighting}
\end{Shaded}

\begin{verbatim}
## True
\end{verbatim}
\end{frame}

\begin{frame}[fragile]{MCQ boolean 5}
\protect\hypertarget{mcq-boolean-5}{}
\begin{itemize}
\tightlist
\item
  What is the value of the following statement ?
\end{itemize}

\begin{Shaded}
\begin{Highlighting}[]
\BuiltInTok{print}\NormalTok{(}\BuiltInTok{bool}\NormalTok{(}\DecValTok{0}\NormalTok{))}
\end{Highlighting}
\end{Shaded}
\end{frame}

\begin{frame}[fragile]{MCQ boolean 5}
\protect\hypertarget{mcq-boolean-5-1}{}
\begin{itemize}
\tightlist
\item
  What is the value of the following statement ?
\end{itemize}

\begin{Shaded}
\begin{Highlighting}[]
\BuiltInTok{print}\NormalTok{(}\BuiltInTok{bool}\NormalTok{(}\DecValTok{0}\NormalTok{))}
\end{Highlighting}
\end{Shaded}

\begin{verbatim}
## False
\end{verbatim}
\end{frame}

\begin{frame}[fragile]{MCQ boolean 6}
\protect\hypertarget{mcq-boolean-6}{}
\begin{itemize}
\tightlist
\item
  What is the value of the following statement ?
\end{itemize}

\begin{Shaded}
\begin{Highlighting}[]
\BuiltInTok{print}\NormalTok{(}\BuiltInTok{bool}\NormalTok{(}\DecValTok{1}\NormalTok{))}
\end{Highlighting}
\end{Shaded}
\end{frame}

\begin{frame}[fragile]{MCQ boolean 6}
\protect\hypertarget{mcq-boolean-6-1}{}
\begin{itemize}
\tightlist
\item
  What is the value of the following statement ?
\end{itemize}

\begin{Shaded}
\begin{Highlighting}[]
\BuiltInTok{print}\NormalTok{(}\BuiltInTok{bool}\NormalTok{(}\DecValTok{1}\NormalTok{))}
\end{Highlighting}
\end{Shaded}

\begin{verbatim}
## True
\end{verbatim}
\end{frame}

\begin{frame}[fragile]{MCQ data type}
\protect\hypertarget{mcq-data-type}{}
\begin{itemize}
\tightlist
\item
  What will be printed from the following examples ?
\end{itemize}

\begin{Shaded}
\begin{Highlighting}[]
\NormalTok{x }\OperatorTok{=} \DecValTok{5}
\BuiltInTok{print}\NormalTok{(}\BuiltInTok{type}\NormalTok{(x))}
\end{Highlighting}
\end{Shaded}

\begin{Shaded}
\begin{Highlighting}[]
\NormalTok{x }\OperatorTok{=} \StringTok{"Hello World"}
\BuiltInTok{print}\NormalTok{(}\BuiltInTok{type}\NormalTok{(x))}
\end{Highlighting}
\end{Shaded}

\begin{Shaded}
\begin{Highlighting}[]
\NormalTok{x }\OperatorTok{=} \FloatTok{20.5}
\BuiltInTok{print}\NormalTok{(}\BuiltInTok{type}\NormalTok{(x))}
\end{Highlighting}
\end{Shaded}

\begin{Shaded}
\begin{Highlighting}[]
\NormalTok{x }\OperatorTok{=}\NormalTok{ [}\StringTok{"cat"}\NormalTok{, }\StringTok{"dog"}\NormalTok{, }\StringTok{"horse"}\NormalTok{]}
\BuiltInTok{print}\NormalTok{(}\BuiltInTok{type}\NormalTok{(x))}
\end{Highlighting}
\end{Shaded}

\begin{Shaded}
\begin{Highlighting}[]
\NormalTok{x }\OperatorTok{=}\NormalTok{ (}\StringTok{"cat"}\NormalTok{, }\StringTok{"dog"}\NormalTok{, }\StringTok{"horse"}\NormalTok{)}
\BuiltInTok{print}\NormalTok{(}\BuiltInTok{type}\NormalTok{(x))}
\end{Highlighting}
\end{Shaded}

\begin{Shaded}
\begin{Highlighting}[]
\NormalTok{x }\OperatorTok{=}\NormalTok{ \{}\StringTok{"name"}\NormalTok{ : }\StringTok{"John"}\NormalTok{, }\StringTok{"lastname"}\NormalTok{ : }\StringTok{"Doe"}\NormalTok{, }\StringTok{"age"}\NormalTok{ : }\DecValTok{33}\NormalTok{\}}
\BuiltInTok{print}\NormalTok{(}\BuiltInTok{type}\NormalTok{(x))}
\end{Highlighting}
\end{Shaded}

\begin{Shaded}
\begin{Highlighting}[]
\NormalTok{x }\OperatorTok{=} \VariableTok{True}
\BuiltInTok{print}\NormalTok{(}\BuiltInTok{type}\NormalTok{(x))}
\end{Highlighting}
\end{Shaded}
\end{frame}

\begin{frame}[fragile]{MCQ Strings 1}
\protect\hypertarget{mcq-strings-1}{}
\begin{itemize}
\tightlist
\item
  Get the first character of the string txt
\end{itemize}

\begin{Shaded}
\begin{Highlighting}[]
\NormalTok{txt }\OperatorTok{=} \StringTok{"Hello World"}
\NormalTok{x }\OperatorTok{=}\NormalTok{ ?}
\end{Highlighting}
\end{Shaded}
\end{frame}

\begin{frame}[fragile]{MCQ Strings 1}
\protect\hypertarget{mcq-strings-1-1}{}
\begin{itemize}
\tightlist
\item
  Get the first character of the string txt
\end{itemize}

\begin{Shaded}
\begin{Highlighting}[]
\NormalTok{txt }\OperatorTok{=} \StringTok{"Hello World"}
\NormalTok{x }\OperatorTok{=}\NormalTok{ txt[}\DecValTok{0}\NormalTok{]}
\BuiltInTok{print}\NormalTok{(x)}
\end{Highlighting}
\end{Shaded}

\begin{verbatim}
## H
\end{verbatim}
\end{frame}

\begin{frame}[fragile]{MCQ Strings 2}
\protect\hypertarget{mcq-strings-2}{}
\begin{itemize}
\tightlist
\item
  Get the character from index 2 to 4 (llo)
\end{itemize}

\begin{Shaded}
\begin{Highlighting}[]
\NormalTok{txt }\OperatorTok{=} \StringTok{"Hello World"}
\NormalTok{x }\OperatorTok{=} 
\end{Highlighting}
\end{Shaded}
\end{frame}

\begin{frame}[fragile]{MCQ Strings 2}
\protect\hypertarget{mcq-strings-2-1}{}
\begin{itemize}
\tightlist
\item
  Get the character from index 2 to 4 (llo)
\end{itemize}

\begin{Shaded}
\begin{Highlighting}[]
\NormalTok{txt }\OperatorTok{=} \StringTok{"Hello World"}
\NormalTok{x }\OperatorTok{=}\NormalTok{ txt[}\DecValTok{2}\NormalTok{:}\DecValTok{5}\NormalTok{]}
\BuiltInTok{print}\NormalTok{(x)}
\end{Highlighting}
\end{Shaded}

\begin{verbatim}
## llo
\end{verbatim}
\end{frame}

\begin{frame}[fragile]{MCQ Strings 3}
\protect\hypertarget{mcq-strings-3}{}
\begin{itemize}
\tightlist
\item
  Return the string without any whitespace at the beginning
\end{itemize}

\begin{Shaded}
\begin{Highlighting}[]
\NormalTok{txt }\OperatorTok{=} \StringTok{" Hello World"}
\NormalTok{x }\OperatorTok{=} 
\end{Highlighting}
\end{Shaded}
\end{frame}

\begin{frame}[fragile]{MCQ Strings 3}
\protect\hypertarget{mcq-strings-3-1}{}
\begin{itemize}
\tightlist
\item
  Return the string without any whitespace at the beginning
\end{itemize}

\begin{Shaded}
\begin{Highlighting}[]
\NormalTok{txt }\OperatorTok{=} \StringTok{" Hello World"}
\BuiltInTok{print}\NormalTok{(txt)}
\end{Highlighting}
\end{Shaded}

\begin{verbatim}
##  Hello World
\end{verbatim}

\begin{Shaded}
\begin{Highlighting}[]
\NormalTok{x }\OperatorTok{=}\NormalTok{ txt[}\DecValTok{1}\NormalTok{:]}
\BuiltInTok{print}\NormalTok{(x)}
\end{Highlighting}
\end{Shaded}

\begin{verbatim}
## Hello World
\end{verbatim}

\begin{Shaded}
\begin{Highlighting}[]
\NormalTok{y }\OperatorTok{=}\NormalTok{ txt.strip()}
\BuiltInTok{print}\NormalTok{(y)}
\end{Highlighting}
\end{Shaded}

\begin{verbatim}
## Hello World
\end{verbatim}
\end{frame}

\begin{frame}[fragile]{MCQ Strings 4}
\protect\hypertarget{mcq-strings-4}{}
\begin{itemize}
\tightlist
\item
  Convert the text in upper case
\end{itemize}

\begin{Shaded}
\begin{Highlighting}[]
\NormalTok{txt }\OperatorTok{=} \StringTok{"Hello World"}
\NormalTok{x }\OperatorTok{=} 
\end{Highlighting}
\end{Shaded}
\end{frame}

\begin{frame}[fragile]{MCQ Strings 4}
\protect\hypertarget{mcq-strings-4-1}{}
\begin{itemize}
\tightlist
\item
  Convert the text in upper case
\end{itemize}

\begin{Shaded}
\begin{Highlighting}[]
\NormalTok{txt }\OperatorTok{=} \StringTok{"Hello World"}
\NormalTok{x }\OperatorTok{=}\NormalTok{ txt.upper()}
\BuiltInTok{print}\NormalTok{(x)}
\end{Highlighting}
\end{Shaded}

\begin{verbatim}
## HELLO WORLD
\end{verbatim}
\end{frame}

\begin{frame}[fragile]{MCQ Strings 5}
\protect\hypertarget{mcq-strings-5}{}
\begin{itemize}
\tightlist
\item
  Convert the text in lower case
\end{itemize}

\begin{Shaded}
\begin{Highlighting}[]
\NormalTok{txt }\OperatorTok{=} \StringTok{"Hello World"}
\NormalTok{x }\OperatorTok{=} 
\end{Highlighting}
\end{Shaded}
\end{frame}

\begin{frame}[fragile]{MCQ Strings 5}
\protect\hypertarget{mcq-strings-5-1}{}
\begin{itemize}
\tightlist
\item
  Convert the text in lower case
\end{itemize}

\begin{Shaded}
\begin{Highlighting}[]
\NormalTok{txt }\OperatorTok{=} \StringTok{"Hello World"}
\NormalTok{x }\OperatorTok{=}\NormalTok{ txt.lower()}
\BuiltInTok{print}\NormalTok{(x)}
\end{Highlighting}
\end{Shaded}

\begin{verbatim}
## hello world
\end{verbatim}
\end{frame}

\begin{frame}[fragile]{MCQ Strings 6}
\protect\hypertarget{mcq-strings-6}{}
\begin{itemize}
\tightlist
\item
  Replace H by J
\end{itemize}

\begin{Shaded}
\begin{Highlighting}[]
\NormalTok{txt }\OperatorTok{=} \StringTok{"Hello World"}
\NormalTok{x }\OperatorTok{=} 
\end{Highlighting}
\end{Shaded}
\end{frame}

\begin{frame}[fragile]{MCQ Strings 6}
\protect\hypertarget{mcq-strings-6-1}{}
\begin{itemize}
\tightlist
\item
  Replace H by J
\end{itemize}

\begin{Shaded}
\begin{Highlighting}[]
\NormalTok{txt }\OperatorTok{=} \StringTok{"Hello World"}
\NormalTok{x }\OperatorTok{=}\NormalTok{ txt.replace(}\StringTok{"H"}\NormalTok{, }\StringTok{"J"}\NormalTok{)}
\BuiltInTok{print}\NormalTok{(x)}
\end{Highlighting}
\end{Shaded}

\begin{verbatim}
## Jello World
\end{verbatim}
\end{frame}

\begin{frame}[fragile]{MCQ operator 1}
\protect\hypertarget{mcq-operator-1}{}
\begin{itemize}
\tightlist
\item
  Use the correct membership operator to check if ``cat'' is present in
  the animal object.
\end{itemize}

\begin{Shaded}
\begin{Highlighting}[]
\NormalTok{animal }\OperatorTok{=}\NormalTok{ [}\StringTok{"cat"}\NormalTok{, }\StringTok{"dog"}\NormalTok{]}
\ControlFlowTok{if} \StringTok{"cat"} 
\end{Highlighting}
\end{Shaded}
\end{frame}

\begin{frame}[fragile]{MCQ operator 1}
\protect\hypertarget{mcq-operator-1-1}{}
\begin{itemize}
\tightlist
\item
  Use the correct membership operator to check if ``cat'' is present in
  the animal object.
\end{itemize}

\begin{Shaded}
\begin{Highlighting}[]
\NormalTok{animal }\OperatorTok{=}\NormalTok{ [}\StringTok{"cat"}\NormalTok{, }\StringTok{"dog"}\NormalTok{]}
\ControlFlowTok{if} \StringTok{"cat"} \KeywordTok{in}\NormalTok{ animal:}
  \BuiltInTok{print}\NormalTok{(}\StringTok{"Yes, cat is a animal!"}\NormalTok{)}
\end{Highlighting}
\end{Shaded}

\begin{verbatim}
## Yes, cat is a animal!
\end{verbatim}
\end{frame}

\begin{frame}[fragile]{MCQ operator 2}
\protect\hypertarget{mcq-operator-2}{}
\begin{itemize}
\tightlist
\item
  Use the correct comparison operator to check if 5 is not equal to 1.
\end{itemize}

\begin{Shaded}
\begin{Highlighting}[]
\ControlFlowTok{if} 
\end{Highlighting}
\end{Shaded}
\end{frame}

\begin{frame}[fragile]{MCQ operator 2}
\protect\hypertarget{mcq-operator-2-1}{}
\begin{itemize}
\tightlist
\item
  Use the correct comparison operator to check if 5 is not equal to 1.
\end{itemize}

\begin{Shaded}
\begin{Highlighting}[]
\ControlFlowTok{if} \DecValTok{5} \OperatorTok{!=} \DecValTok{10}\NormalTok{:}
  \BuiltInTok{print}\NormalTok{(}\StringTok{"5 and 10 is not equal"}\NormalTok{)}
\end{Highlighting}
\end{Shaded}

\begin{verbatim}
## 5 and 10 is not equal
\end{verbatim}
\end{frame}

\begin{frame}[fragile]{MCQ operator 3}
\protect\hypertarget{mcq-operator-3}{}
\begin{itemize}
\tightlist
\item
  Use the correct logical operator to check if at least one of two
  statements is True.
\end{itemize}

\begin{Shaded}
\begin{Highlighting}[]
\ControlFlowTok{if} \DecValTok{5} \OperatorTok{==} \DecValTok{10}\NormalTok{ ?? }\DecValTok{4} \OperatorTok{==} \DecValTok{4}\NormalTok{:}
  \BuiltInTok{print}\NormalTok{(}\StringTok{"At least one of the statements is true"}\NormalTok{)}
\end{Highlighting}
\end{Shaded}
\end{frame}

\begin{frame}[fragile]{MCQ operator 3}
\protect\hypertarget{mcq-operator-3-1}{}
\begin{itemize}
\tightlist
\item
  Use the correct logical operator to check if at least one of two
  statements is True.
\end{itemize}

\begin{Shaded}
\begin{Highlighting}[]
\ControlFlowTok{if} \DecValTok{5} \OperatorTok{==} \DecValTok{10} \KeywordTok{or} \DecValTok{4} \OperatorTok{==} \DecValTok{4}\NormalTok{:}
  \BuiltInTok{print}\NormalTok{(}\StringTok{"At least one of the statements is true"}\NormalTok{)}
  
\end{Highlighting}
\end{Shaded}

\begin{verbatim}
## At least one of the statements is true
\end{verbatim}

\begin{Shaded}
\begin{Highlighting}[]
\ControlFlowTok{if}\NormalTok{ (}\DecValTok{5} \OperatorTok{==} \DecValTok{10}\NormalTok{) }\OperatorTok{|}\NormalTok{ (}\DecValTok{4} \OperatorTok{==} \DecValTok{4}\NormalTok{):}
  \BuiltInTok{print}\NormalTok{(}\StringTok{"At least one of the statements is true"}\NormalTok{)}
\end{Highlighting}
\end{Shaded}

\begin{verbatim}
## At least one of the statements is true
\end{verbatim}
\end{frame}

\begin{frame}[fragile]{MCQ List 1}
\protect\hypertarget{mcq-list-1}{}
\begin{itemize}
\tightlist
\item
  Print the third item in the fruits list.
\end{itemize}

\begin{Shaded}
\begin{Highlighting}[]
\NormalTok{animal }\OperatorTok{=}\NormalTok{ [}\StringTok{"cat"}\NormalTok{, }\StringTok{"dog"}\NormalTok{,}\StringTok{"horse"}\NormalTok{]}
\BuiltInTok{print}\NormalTok{()}
\end{Highlighting}
\end{Shaded}
\end{frame}

\begin{frame}[fragile]{MCQ List 1}
\protect\hypertarget{mcq-list-1-1}{}
\begin{itemize}
\tightlist
\item
  Use the correct logical operator to check if at least one of two
  statements is True.
\end{itemize}

\begin{Shaded}
\begin{Highlighting}[]
\NormalTok{animal }\OperatorTok{=}\NormalTok{ [}\StringTok{"cat"}\NormalTok{, }\StringTok{"dog"}\NormalTok{,}\StringTok{"horse"}\NormalTok{]}
\BuiltInTok{print}\NormalTok{(animal[}\DecValTok{2}\NormalTok{])}
\end{Highlighting}
\end{Shaded}

\begin{verbatim}
## horse
\end{verbatim}
\end{frame}

\begin{frame}[fragile]{MCQ List 2}
\protect\hypertarget{mcq-list-2}{}
\begin{itemize}
\tightlist
\item
  Change the value from ``cat'' to ``lion'', in the fruits list.
\end{itemize}

\begin{Shaded}
\begin{Highlighting}[]
\NormalTok{animal }\OperatorTok{=}\NormalTok{ [}\StringTok{"cat"}\NormalTok{, }\StringTok{"dog"}\NormalTok{,}\StringTok{"horse"}\NormalTok{]}
\end{Highlighting}
\end{Shaded}
\end{frame}

\begin{frame}[fragile]{MCQ List 2}
\protect\hypertarget{mcq-list-2-1}{}
\begin{itemize}
\tightlist
\item
  Change the value from ``cat'' to ``lion'', in the fruits list.
\end{itemize}

\begin{Shaded}
\begin{Highlighting}[]
\NormalTok{animal }\OperatorTok{=}\NormalTok{ [}\StringTok{"cat"}\NormalTok{, }\StringTok{"dog"}\NormalTok{,}\StringTok{"horse"}\NormalTok{]}
\NormalTok{animal[}\DecValTok{0}\NormalTok{] }\OperatorTok{=} \StringTok{"lion"}
\BuiltInTok{print}\NormalTok{(animal)}
\end{Highlighting}
\end{Shaded}

\begin{verbatim}
## ['lion', 'dog', 'horse']
\end{verbatim}
\end{frame}

\begin{frame}[fragile]{MCQ List 3}
\protect\hypertarget{mcq-list-3}{}
\begin{itemize}
\tightlist
\item
  Add cow to the animal list
\end{itemize}

\begin{Shaded}
\begin{Highlighting}[]
\NormalTok{animal }\OperatorTok{=}\NormalTok{ [}\StringTok{"cat"}\NormalTok{, }\StringTok{"dog"}\NormalTok{,}\StringTok{"horse"}\NormalTok{]}
\end{Highlighting}
\end{Shaded}
\end{frame}

\begin{frame}[fragile]{MCQ List 3}
\protect\hypertarget{mcq-list-3-1}{}
\begin{itemize}
\tightlist
\item
  Add cow to the animal list
\end{itemize}

\begin{Shaded}
\begin{Highlighting}[]
\NormalTok{animal }\OperatorTok{=}\NormalTok{ [}\StringTok{"cat"}\NormalTok{, }\StringTok{"dog"}\NormalTok{,}\StringTok{"horse"}\NormalTok{]}
\NormalTok{animal.append(}\StringTok{"cow"}\NormalTok{)}
\BuiltInTok{print}\NormalTok{(animal)}
\end{Highlighting}
\end{Shaded}

\begin{verbatim}
## ['cat', 'dog', 'horse', 'cow']
\end{verbatim}
\end{frame}

\begin{frame}[fragile]{MCQ List 4}
\protect\hypertarget{mcq-list-4}{}
\begin{itemize}
\tightlist
\item
  remove dog to the animal list
\end{itemize}

\begin{Shaded}
\begin{Highlighting}[]
\NormalTok{animal }\OperatorTok{=}\NormalTok{ [}\StringTok{"cat"}\NormalTok{, }\StringTok{"dog"}\NormalTok{,}\StringTok{"horse"}\NormalTok{]}
\end{Highlighting}
\end{Shaded}
\end{frame}

\begin{frame}[fragile]{MCQ List 4}
\protect\hypertarget{mcq-list-4-1}{}
\begin{itemize}
\tightlist
\item
  remove dog to the animal list
\end{itemize}

\begin{Shaded}
\begin{Highlighting}[]
\NormalTok{animal }\OperatorTok{=}\NormalTok{ [}\StringTok{"cat"}\NormalTok{, }\StringTok{"dog"}\NormalTok{,}\StringTok{"horse"}\NormalTok{]}
\NormalTok{animal.remove(}\StringTok{"dog"}\NormalTok{)}
\BuiltInTok{print}\NormalTok{(animal)}
\end{Highlighting}
\end{Shaded}

\begin{verbatim}
## ['cat', 'horse']
\end{verbatim}
\end{frame}

\begin{frame}[fragile]{MCQ List 5}
\protect\hypertarget{mcq-list-5}{}
\begin{itemize}
\tightlist
\item
  Use negative indexing to print the last item in the list.
\end{itemize}

\begin{Shaded}
\begin{Highlighting}[]
\NormalTok{animal }\OperatorTok{=}\NormalTok{ [}\StringTok{"cat"}\NormalTok{, }\StringTok{"dog"}\NormalTok{,}\StringTok{"horse"}\NormalTok{]}
\end{Highlighting}
\end{Shaded}
\end{frame}

\begin{frame}[fragile]{MCQ List 5}
\protect\hypertarget{mcq-list-5-1}{}
\begin{itemize}
\tightlist
\item
  Use negative indexing to print the last item in the list.
\end{itemize}

\begin{Shaded}
\begin{Highlighting}[]
\NormalTok{animal }\OperatorTok{=}\NormalTok{ [}\StringTok{"cat"}\NormalTok{, }\StringTok{"dog"}\NormalTok{,}\StringTok{"horse"}\NormalTok{]}
\BuiltInTok{print}\NormalTok{(animal[}\OperatorTok{{-}}\DecValTok{1}\NormalTok{])}
\end{Highlighting}
\end{Shaded}

\begin{verbatim}
## horse
\end{verbatim}
\end{frame}

\begin{frame}[fragile]{MCQ List 6}
\protect\hypertarget{mcq-list-6}{}
\begin{itemize}
\tightlist
\item
  Use the correct syntax to print the number of items in the list.
\end{itemize}

\begin{Shaded}
\begin{Highlighting}[]
\NormalTok{animal }\OperatorTok{=}\NormalTok{ [}\StringTok{"cat"}\NormalTok{, }\StringTok{"dog"}\NormalTok{,}\StringTok{"horse"}\NormalTok{]}
\end{Highlighting}
\end{Shaded}
\end{frame}

\begin{frame}[fragile]{MCQ List 6}
\protect\hypertarget{mcq-list-6-1}{}
\begin{itemize}
\tightlist
\item
  Use the correct syntax to print the number of items in the list.
\end{itemize}

\begin{Shaded}
\begin{Highlighting}[]
\NormalTok{animal }\OperatorTok{=}\NormalTok{ [}\StringTok{"cat"}\NormalTok{, }\StringTok{"dog"}\NormalTok{,}\StringTok{"horse"}\NormalTok{]}
\BuiltInTok{print}\NormalTok{(}\BuiltInTok{len}\NormalTok{(animal))}
\end{Highlighting}
\end{Shaded}

\begin{verbatim}
## 3
\end{verbatim}
\end{frame}

\begin{frame}[fragile]{MCQ List 7}
\protect\hypertarget{mcq-list-7}{}
\begin{itemize}
\tightlist
\item
  Use a range of indexes to print the third, fourth, and fifth item in
  the list.
\end{itemize}

\begin{Shaded}
\begin{Highlighting}[]
\NormalTok{animal }\OperatorTok{=}\NormalTok{ [}\StringTok{"cat"}\NormalTok{, }\StringTok{"dog"}\NormalTok{,}\StringTok{"horse"}\NormalTok{]}
\end{Highlighting}
\end{Shaded}
\end{frame}

\begin{frame}[fragile]{MCQ List 7}
\protect\hypertarget{mcq-list-7-1}{}
\begin{itemize}
\tightlist
\item
  Use a range of indexes to print the third, fourth, and fifth item in
  the list.
\end{itemize}

\begin{Shaded}
\begin{Highlighting}[]
\NormalTok{animal }\OperatorTok{=}\NormalTok{ [}\StringTok{"cat"}\NormalTok{, }\StringTok{"dog"}\NormalTok{,}\StringTok{"horse"}\NormalTok{]}
\BuiltInTok{print}\NormalTok{(animal[}\DecValTok{2}\NormalTok{:}\DecValTok{5}\NormalTok{])}
\end{Highlighting}
\end{Shaded}

\begin{verbatim}
## ['horse']
\end{verbatim}
\end{frame}

\begin{frame}[fragile]{MCQ Dictionaries 1}
\protect\hypertarget{mcq-dictionaries-1}{}
\begin{itemize}
\tightlist
\item
  Use the get method to print the value of the ``model'' key of the car
  dictionary.
\end{itemize}

\begin{Shaded}
\begin{Highlighting}[]
\NormalTok{car }\OperatorTok{=}\NormalTok{   \{}
  \StringTok{"brand"}\NormalTok{: }\StringTok{"Ford"}\NormalTok{,}
  \StringTok{"model"}\NormalTok{: }\StringTok{"Mustang"}\NormalTok{,}
  \StringTok{"year"}\NormalTok{: }\DecValTok{1964}
\NormalTok{\}}
\BuiltInTok{print}\NormalTok{()}
\end{Highlighting}
\end{Shaded}
\end{frame}

\begin{frame}[fragile]{MCQ Dictionaries 1}
\protect\hypertarget{mcq-dictionaries-1-1}{}
\begin{itemize}
\tightlist
\item
  Use the get method to print the value of the ``model'' key of the car
  dictionary.
\end{itemize}

\begin{Shaded}
\begin{Highlighting}[]
\NormalTok{car }\OperatorTok{=}\NormalTok{   \{}
  \StringTok{"brand"}\NormalTok{: }\StringTok{"Ford"}\NormalTok{,}
  \StringTok{"model"}\NormalTok{: }\StringTok{"Mustang"}\NormalTok{,}
  \StringTok{"year"}\NormalTok{: }\DecValTok{1964}
\NormalTok{\}}
\BuiltInTok{print}\NormalTok{(car.get(}\StringTok{"model"}\NormalTok{))}
\end{Highlighting}
\end{Shaded}

\begin{verbatim}
## Mustang
\end{verbatim}
\end{frame}

\begin{frame}[fragile]{MCQ Dictionaries 2}
\protect\hypertarget{mcq-dictionaries-2}{}
Change the ``year'' value from 1964 to 2023.

\begin{Shaded}
\begin{Highlighting}[]
\NormalTok{car }\OperatorTok{=}\NormalTok{   \{}
  \StringTok{"brand"}\NormalTok{: }\StringTok{"Ford"}\NormalTok{,}
  \StringTok{"model"}\NormalTok{: }\StringTok{"Mustang"}\NormalTok{,}
  \StringTok{"year"}\NormalTok{: }\DecValTok{1964}
\NormalTok{\}}
\end{Highlighting}
\end{Shaded}
\end{frame}

\begin{frame}[fragile]{MCQ Dictionaries 2}
\protect\hypertarget{mcq-dictionaries-2-1}{}
Change the ``year'' value from 1964 to 2023.

\begin{Shaded}
\begin{Highlighting}[]
\NormalTok{car }\OperatorTok{=}\NormalTok{   \{}
  \StringTok{"brand"}\NormalTok{: }\StringTok{"Ford"}\NormalTok{,}
  \StringTok{"model"}\NormalTok{: }\StringTok{"Mustang"}\NormalTok{,}
  \StringTok{"year"}\NormalTok{: }\DecValTok{1964}
\NormalTok{\}}
\NormalTok{car[}\StringTok{"year"}\NormalTok{]}\OperatorTok{=}\DecValTok{2023}
\BuiltInTok{print}\NormalTok{(car)}
\end{Highlighting}
\end{Shaded}

\begin{verbatim}
## {'brand': 'Ford', 'model': 'Mustang', 'year': 2023}
\end{verbatim}
\end{frame}

\begin{frame}[fragile]{MCQ Dictionaries 3}
\protect\hypertarget{mcq-dictionaries-3}{}
Add the key/value pair ``color'' : ``red'' to the car dictionary.

\begin{Shaded}
\begin{Highlighting}[]
\NormalTok{car }\OperatorTok{=}\NormalTok{   \{}
  \StringTok{"brand"}\NormalTok{: }\StringTok{"Ford"}\NormalTok{,}
  \StringTok{"model"}\NormalTok{: }\StringTok{"Mustang"}\NormalTok{,}
  \StringTok{"year"}\NormalTok{: }\DecValTok{1964}
\NormalTok{\}}
\end{Highlighting}
\end{Shaded}
\end{frame}

\begin{frame}[fragile]{MCQ Dictionaries 3}
\protect\hypertarget{mcq-dictionaries-3-1}{}
Add the key/value pair ``color'' : ``red'' to the car dictionary.

\begin{Shaded}
\begin{Highlighting}[]
\NormalTok{car }\OperatorTok{=}\NormalTok{   \{}
  \StringTok{"brand"}\NormalTok{: }\StringTok{"Ford"}\NormalTok{,}
  \StringTok{"model"}\NormalTok{: }\StringTok{"Mustang"}\NormalTok{,}
  \StringTok{"year"}\NormalTok{: }\DecValTok{1964}
\NormalTok{\}}
\NormalTok{car[}\StringTok{"color"}\NormalTok{]}\OperatorTok{=}\StringTok{"red"}
\BuiltInTok{print}\NormalTok{(car)}
\end{Highlighting}
\end{Shaded}

\begin{verbatim}
## {'brand': 'Ford', 'model': 'Mustang', 'year': 1964, 'color': 'red'}
\end{verbatim}
\end{frame}

\begin{frame}[fragile]{MCQ Dictionaries 4}
\protect\hypertarget{mcq-dictionaries-4}{}
Use the pop method to remove ``model'' from the car dictionary.

\begin{Shaded}
\begin{Highlighting}[]
\NormalTok{car }\OperatorTok{=}\NormalTok{   \{}
  \StringTok{"brand"}\NormalTok{: }\StringTok{"Ford"}\NormalTok{,}
  \StringTok{"model"}\NormalTok{: }\StringTok{"Mustang"}\NormalTok{,}
  \StringTok{"year"}\NormalTok{: }\DecValTok{1964}
\NormalTok{\}}
\end{Highlighting}
\end{Shaded}
\end{frame}

\begin{frame}[fragile]{MCQ Dictionaries 4}
\protect\hypertarget{mcq-dictionaries-4-1}{}
Use the pop method to remove ``model'' from the car dictionary.

\begin{Shaded}
\begin{Highlighting}[]
\NormalTok{car }\OperatorTok{=}\NormalTok{   \{}
  \StringTok{"brand"}\NormalTok{: }\StringTok{"Ford"}\NormalTok{,}
  \StringTok{"model"}\NormalTok{: }\StringTok{"Mustang"}\NormalTok{,}
  \StringTok{"year"}\NormalTok{: }\DecValTok{1964}
\NormalTok{\}}
\NormalTok{car.pop(}\StringTok{"model"}\NormalTok{)}
\end{Highlighting}
\end{Shaded}

\begin{verbatim}
## 'Mustang'
\end{verbatim}

\begin{Shaded}
\begin{Highlighting}[]
\BuiltInTok{print}\NormalTok{(car)}
\end{Highlighting}
\end{Shaded}

\begin{verbatim}
## {'brand': 'Ford', 'year': 1964}
\end{verbatim}
\end{frame}

\begin{frame}[fragile]{MCQ Dictionaries 5}
\protect\hypertarget{mcq-dictionaries-5}{}
Use the clear method to empty the car dictionary.

\begin{Shaded}
\begin{Highlighting}[]
\NormalTok{car }\OperatorTok{=}\NormalTok{   \{}
  \StringTok{"brand"}\NormalTok{: }\StringTok{"Ford"}\NormalTok{,}
  \StringTok{"model"}\NormalTok{: }\StringTok{"Mustang"}\NormalTok{,}
  \StringTok{"year"}\NormalTok{: }\DecValTok{1964}
\NormalTok{\}}
\end{Highlighting}
\end{Shaded}
\end{frame}

\begin{frame}[fragile]{MCQ Dictionaries 5}
\protect\hypertarget{mcq-dictionaries-5-1}{}
Use the clear method to empty the car dictionary.

\begin{Shaded}
\begin{Highlighting}[]
\NormalTok{car }\OperatorTok{=}\NormalTok{   \{}
  \StringTok{"brand"}\NormalTok{: }\StringTok{"Ford"}\NormalTok{,}
  \StringTok{"model"}\NormalTok{: }\StringTok{"Mustang"}\NormalTok{,}
  \StringTok{"year"}\NormalTok{: }\DecValTok{1964}
\NormalTok{\}}
\NormalTok{car.clear()}
\BuiltInTok{print}\NormalTok{(car)}
\end{Highlighting}
\end{Shaded}

\begin{verbatim}
## {}
\end{verbatim}
\end{frame}

\begin{frame}[fragile]{MCQ If 1}
\protect\hypertarget{mcq-if-1}{}
\begin{itemize}
\tightlist
\item
  Print ``Yes'' if a is equal to b, otherwise print ``No''.
\end{itemize}

\begin{Shaded}
\begin{Highlighting}[]
\NormalTok{a }\OperatorTok{=} \DecValTok{50}
\NormalTok{b }\OperatorTok{=} \DecValTok{10}
\end{Highlighting}
\end{Shaded}
\end{frame}

\begin{frame}[fragile]{MCQ If 1}
\protect\hypertarget{mcq-if-1-1}{}
\begin{itemize}
\tightlist
\item
  Print ``Yes'' if a is equal to b, otherwise print ``No''.
\end{itemize}

\begin{Shaded}
\begin{Highlighting}[]
\NormalTok{a }\OperatorTok{=} \DecValTok{50}
\NormalTok{b }\OperatorTok{=} \DecValTok{10}
\ControlFlowTok{if}\NormalTok{ a }\OperatorTok{==}\NormalTok{ b:}
  \BuiltInTok{print}\NormalTok{(}\StringTok{"Yes"}\NormalTok{)}
\ControlFlowTok{else}\NormalTok{:}
  \BuiltInTok{print}\NormalTok{(}\StringTok{"No"}\NormalTok{)}
\end{Highlighting}
\end{Shaded}

\begin{verbatim}
## No
\end{verbatim}
\end{frame}

\begin{frame}[fragile]{MCQ If 3}
\protect\hypertarget{mcq-if-3}{}
Print ``Hello'' if a is equal to b, and c is equal to d.

\begin{Shaded}
\begin{Highlighting}[]
\NormalTok{a }\OperatorTok{=} \DecValTok{50}
\NormalTok{b }\OperatorTok{=} \DecValTok{10}
\NormalTok{c }\OperatorTok{=} \DecValTok{30}
\NormalTok{d }\OperatorTok{=} \DecValTok{30}
\end{Highlighting}
\end{Shaded}
\end{frame}

\begin{frame}[fragile]{MCQ If 3}
\protect\hypertarget{mcq-if-3-1}{}
Print ``Hello'' if a is equal to b, and c is equal to d.

\begin{Shaded}
\begin{Highlighting}[]
\NormalTok{a }\OperatorTok{=} \DecValTok{50}
\NormalTok{b }\OperatorTok{=} \DecValTok{10}
\NormalTok{c }\OperatorTok{=} \DecValTok{30}
\NormalTok{d }\OperatorTok{=} \DecValTok{30}
\ControlFlowTok{if}\NormalTok{ a }\OperatorTok{==}\NormalTok{ b }\KeywordTok{and}\NormalTok{ c}\OperatorTok{==}\NormalTok{d:}
  \BuiltInTok{print}\NormalTok{(}\StringTok{"Hello"}\NormalTok{)}
\end{Highlighting}
\end{Shaded}
\end{frame}

\begin{frame}[fragile]{MCQ For 1}
\protect\hypertarget{mcq-for-1}{}
Loop through the items in the animal list.

\begin{Shaded}
\begin{Highlighting}[]
\NormalTok{animal }\OperatorTok{=}\NormalTok{ [}\StringTok{"cat"}\NormalTok{, }\StringTok{"dog"}\NormalTok{,}\StringTok{"horse"}\NormalTok{]}
\end{Highlighting}
\end{Shaded}
\end{frame}

\begin{frame}[fragile]{MCQ For 1}
\protect\hypertarget{mcq-for-1-1}{}
Loop through the items in the animal list.

\begin{Shaded}
\begin{Highlighting}[]
\NormalTok{animal }\OperatorTok{=}\NormalTok{ [}\StringTok{"cat"}\NormalTok{, }\StringTok{"dog"}\NormalTok{,}\StringTok{"horse"}\NormalTok{]}
\ControlFlowTok{for}\NormalTok{ idx }\KeywordTok{in}\NormalTok{ animal:}
  \BuiltInTok{print}\NormalTok{(idx)}
\end{Highlighting}
\end{Shaded}

\begin{verbatim}
## cat
## dog
## horse
\end{verbatim}
\end{frame}

\begin{frame}[fragile]{MCQ For 2}
\protect\hypertarget{mcq-for-2}{}
In the loop, when the item value is ``dog'', jump directly to the next
item.

\begin{Shaded}
\begin{Highlighting}[]
\NormalTok{animal }\OperatorTok{=}\NormalTok{ [}\StringTok{"cat"}\NormalTok{, }\StringTok{"dog"}\NormalTok{,}\StringTok{"horse"}\NormalTok{]}
\ControlFlowTok{for}\NormalTok{ idx }\KeywordTok{in}\NormalTok{ animal:}
  \ControlFlowTok{if}\NormalTok{ idx }\OperatorTok{==} \StringTok{"dog"}\NormalTok{:}
\NormalTok{    ?????}
  \BuiltInTok{print}\NormalTok{(idx)}
\end{Highlighting}
\end{Shaded}
\end{frame}

\begin{frame}[fragile]{MCQ For 2}
\protect\hypertarget{mcq-for-2-1}{}
In the loop, when the item value is ``dog'', jump directly to the next
item.

\begin{Shaded}
\begin{Highlighting}[]
\NormalTok{animal }\OperatorTok{=}\NormalTok{ [}\StringTok{"cat"}\NormalTok{, }\StringTok{"dog"}\NormalTok{,}\StringTok{"horse"}\NormalTok{]}
\ControlFlowTok{for}\NormalTok{ idx }\KeywordTok{in}\NormalTok{ animal:}
  \ControlFlowTok{if}\NormalTok{ idx }\OperatorTok{==} \StringTok{"dog"}\NormalTok{:}
    \ControlFlowTok{continue}
  \BuiltInTok{print}\NormalTok{(idx)}
\end{Highlighting}
\end{Shaded}

\begin{verbatim}
## cat
## horse
\end{verbatim}
\end{frame}

\begin{frame}[fragile]{MCQ For 3}
\protect\hypertarget{mcq-for-3}{}
Use the range function to loop through a code set 5 times.

\begin{Shaded}
\begin{Highlighting}[]
\ControlFlowTok{for}\NormalTok{ x }\KeywordTok{in}\NormalTok{ ???? :}
  \BuiltInTok{print}\NormalTok{(x)}
\end{Highlighting}
\end{Shaded}
\end{frame}

\begin{frame}[fragile]{MCQ For 3}
\protect\hypertarget{mcq-for-3-1}{}
Use the range function to loop through a code set 5 times.

\begin{Shaded}
\begin{Highlighting}[]
\ControlFlowTok{for}\NormalTok{ x }\KeywordTok{in} \BuiltInTok{range}\NormalTok{(}\DecValTok{5}\NormalTok{) :}
  \BuiltInTok{print}\NormalTok{(x)}
\end{Highlighting}
\end{Shaded}

\begin{verbatim}
## 0
## 1
## 2
## 3
## 4
\end{verbatim}
\end{frame}

\begin{frame}[fragile]{MCQ While 1}
\protect\hypertarget{mcq-while-1}{}
\begin{itemize}
\tightlist
\item
  Print i as long as i is less than 5.
\end{itemize}

\begin{Shaded}
\begin{Highlighting}[]
\NormalTok{i }\OperatorTok{=} \DecValTok{1}
\end{Highlighting}
\end{Shaded}
\end{frame}

\begin{frame}[fragile]{MCQ While 1}
\protect\hypertarget{mcq-while-1-1}{}
\begin{itemize}
\tightlist
\item
  Print i as long as i is less than 5.
\end{itemize}

\begin{Shaded}
\begin{Highlighting}[]
\NormalTok{i }\OperatorTok{=} \DecValTok{1}
\ControlFlowTok{while}\NormalTok{ i }\OperatorTok{\textless{}} \DecValTok{5}\NormalTok{:}
  \BuiltInTok{print}\NormalTok{(i)}
\NormalTok{  i }\OperatorTok{+=} \DecValTok{1}
\end{Highlighting}
\end{Shaded}

\begin{verbatim}
## 1
## 2
## 3
## 4
\end{verbatim}
\end{frame}

\begin{frame}[fragile]{MCQ While 2}
\protect\hypertarget{mcq-while-2}{}
\begin{itemize}
\tightlist
\item
  Exit the loop if i = 3
\end{itemize}

\begin{Shaded}
\begin{Highlighting}[]
\NormalTok{i }\OperatorTok{=} \DecValTok{1}
\ControlFlowTok{while}\NormalTok{ i }\OperatorTok{\textless{}} \DecValTok{5}\NormalTok{:}
\NormalTok{  i }\OperatorTok{+=} \DecValTok{1}
  \ControlFlowTok{if}\NormalTok{ i }\OperatorTok{==} \DecValTok{3}\NormalTok{:}
\NormalTok{    ?????}
  \BuiltInTok{print}\NormalTok{(i)}
\end{Highlighting}
\end{Shaded}
\end{frame}

\begin{frame}[fragile]{MCQ While 2}
\protect\hypertarget{mcq-while-2-1}{}
\begin{itemize}
\tightlist
\item
  Exit the loop if i = 3
\end{itemize}

\begin{Shaded}
\begin{Highlighting}[]
\NormalTok{i }\OperatorTok{=} \DecValTok{1}
\ControlFlowTok{while}\NormalTok{ i }\OperatorTok{\textless{}} \DecValTok{5}\NormalTok{:}
\NormalTok{  i }\OperatorTok{+=} \DecValTok{1}
  \ControlFlowTok{if}\NormalTok{ i }\OperatorTok{==} \DecValTok{3}\NormalTok{:}
    \ControlFlowTok{break}
  \BuiltInTok{print}\NormalTok{(i)}
\end{Highlighting}
\end{Shaded}

\begin{verbatim}
## 2
\end{verbatim}
\end{frame}

\begin{frame}[fragile]{MCQ Function 1}
\protect\hypertarget{mcq-function-1}{}
\begin{itemize}
\tightlist
\item
  Create a function named my\_function.
\end{itemize}

\begin{Shaded}
\begin{Highlighting}[]
\NormalTok{??????:}
  \BuiltInTok{print}\NormalTok{(}\StringTok{"Hello from a function"}\NormalTok{)}
\end{Highlighting}
\end{Shaded}
\end{frame}

\begin{frame}[fragile]{MCQ Function 1}
\protect\hypertarget{mcq-function-1-1}{}
\begin{itemize}
\tightlist
\item
  Create a function named my\_function.
\end{itemize}

\begin{Shaded}
\begin{Highlighting}[]
\KeywordTok{def}\NormalTok{ my\_function():}
  \BuiltInTok{print}\NormalTok{(}\StringTok{"Hello from a function"}\NormalTok{)}
\end{Highlighting}
\end{Shaded}
\end{frame}

\begin{frame}[fragile]{MCQ Function 2}
\protect\hypertarget{mcq-function-2}{}
\begin{itemize}
\tightlist
\item
  Let the function return the x parameter + 5.
\end{itemize}

\begin{Shaded}
\begin{Highlighting}[]
\NormalTok{var }\OperatorTok{=} \DecValTok{5}

\KeywordTok{def}\NormalTok{ my\_function(x):}
\NormalTok{  ???????}
  
\NormalTok{my\_function(var)}
\end{Highlighting}
\end{Shaded}
\end{frame}

\begin{frame}[fragile]{MCQ Function 2}
\protect\hypertarget{mcq-function-2-1}{}
\begin{itemize}
\tightlist
\item
  Let the function return the x parameter + 5.
\end{itemize}

\begin{Shaded}
\begin{Highlighting}[]
\NormalTok{var }\OperatorTok{=} \DecValTok{5}

\KeywordTok{def}\NormalTok{ my\_function(x):}
  \ControlFlowTok{return}\NormalTok{ x}\OperatorTok{+}\DecValTok{5}

\NormalTok{my\_function(var)}
\end{Highlighting}
\end{Shaded}

\begin{verbatim}
## 10
\end{verbatim}
\end{frame}

\begin{frame}[fragile]{MCQ Function 3}
\protect\hypertarget{mcq-function-3}{}
\begin{itemize}
\tightlist
\item
  If you do not know the number of arguments that will be passed into
  your function, there is a prefix you can add in the function
  definition, which prefix?
\end{itemize}

\begin{Shaded}
\begin{Highlighting}[]
\KeywordTok{def}\NormalTok{ my\_function(???kids):}
  \BuiltInTok{print}\NormalTok{(}\StringTok{"The youngest child is "} \OperatorTok{+}\NormalTok{ kids[}\DecValTok{2}\NormalTok{])}
\end{Highlighting}
\end{Shaded}
\end{frame}

\begin{frame}[fragile]{MCQ Function 3}
\protect\hypertarget{mcq-function-3-1}{}
\begin{itemize}
\tightlist
\item
  If you do not know the number of arguments that will be passed into
  your function, there is a prefix you can add in the function
  definition, which prefix?
\end{itemize}

\begin{Shaded}
\begin{Highlighting}[]
\KeywordTok{def}\NormalTok{ my\_function(}\OperatorTok{*}\NormalTok{kids):}
  \BuiltInTok{print}\NormalTok{(}\StringTok{"The youngest child is "} \OperatorTok{+}\NormalTok{ kids[}\DecValTok{2}\NormalTok{])}
\end{Highlighting}
\end{Shaded}
\end{frame}

\begin{frame}[fragile]{MCQ Function 4}
\protect\hypertarget{mcq-function-4}{}
\begin{itemize}
\tightlist
\item
  If you do not know the number of keyword arguments that will be passed
  into your function, there is a prefix you can add in the function
  definition, which prefix?
\end{itemize}

\begin{Shaded}
\begin{Highlighting}[]
\KeywordTok{def}\NormalTok{ my\_function(???kids):}
  \BuiltInTok{print}\NormalTok{(}\StringTok{"The youngest child is "} \OperatorTok{+}\NormalTok{ kids[}\DecValTok{2}\NormalTok{])}
\end{Highlighting}
\end{Shaded}
\end{frame}

\begin{frame}[fragile]{MCQ Function 4}
\protect\hypertarget{mcq-function-4-1}{}
\begin{itemize}
\tightlist
\item
  If you do not know the number of keyword arguments that will be passed
  into your function, there is a prefix you can add in the function
  definition, which prefix?
\end{itemize}

\begin{Shaded}
\begin{Highlighting}[]
\KeywordTok{def}\NormalTok{ my\_function(}\OperatorTok{**}\NormalTok{kid):}
  \BuiltInTok{print}\NormalTok{(}\StringTok{"His last name is "} \OperatorTok{+}\NormalTok{ kid[}\StringTok{"lname"}\NormalTok{])}
\end{Highlighting}
\end{Shaded}
\end{frame}

\begin{frame}[fragile]{MCQ Module 1}
\protect\hypertarget{mcq-module-1}{}
\begin{itemize}
\tightlist
\item
  What is the correct syntax to import a module named ``mymodule''?
\end{itemize}

\begin{Shaded}
\begin{Highlighting}[]
\NormalTok{ ??? my\_module}
\end{Highlighting}
\end{Shaded}
\end{frame}

\begin{frame}[fragile]{MCQ Module 1}
\protect\hypertarget{mcq-module-1-1}{}
\begin{itemize}
\tightlist
\item
  What is the correct syntax to import a module named ``mymodule''?
\end{itemize}

\begin{Shaded}
\begin{Highlighting}[]
\ImportTok{import}\NormalTok{ my\_module}
\end{Highlighting}
\end{Shaded}
\end{frame}

\begin{frame}[fragile]{MCQ Module 2}
\protect\hypertarget{mcq-module-2}{}
\begin{itemize}
\tightlist
\item
  If you want to refer to a module by using a different name, you can
  create an alias.
\end{itemize}

What is the correct syntax for creating an alias for a module?

\begin{Shaded}
\begin{Highlighting}[]
\ImportTok{import}\NormalTok{ mymodule ?? mx}
\end{Highlighting}
\end{Shaded}
\end{frame}

\begin{frame}[fragile]{MCQ Module 2}
\protect\hypertarget{mcq-module-2-1}{}
\begin{itemize}
\tightlist
\item
  If you want to refer to a module by using a different name, you can
  create an alias.
\end{itemize}

What is the correct syntax for creating an alias for a module?

\begin{Shaded}
\begin{Highlighting}[]
\ImportTok{import}\NormalTok{ mymodule }\ImportTok{as}\NormalTok{ mx}
\end{Highlighting}
\end{Shaded}
\end{frame}

\begin{frame}[fragile]{MCQ Module 3}
\protect\hypertarget{mcq-module-3}{}
\begin{itemize}
\tightlist
\item
  What is the correct syntax of printing all variables and function
  names of the ``random'' module?
\end{itemize}

\begin{Shaded}
\begin{Highlighting}[]
\ImportTok{import}\NormalTok{ random}
\BuiltInTok{print}\NormalTok{(????)}
\end{Highlighting}
\end{Shaded}
\end{frame}

\begin{frame}[fragile]{MCQ Module 3}
\protect\hypertarget{mcq-module-3-1}{}
\begin{itemize}
\tightlist
\item
  What is the correct syntax of printing all variables and function
  names of the ``mymodule'' module?
\end{itemize}

\begin{Shaded}
\begin{Highlighting}[]
\ImportTok{import}\NormalTok{ random}
\BuiltInTok{print}\NormalTok{(}\BuiltInTok{dir}\NormalTok{(random))}
\end{Highlighting}
\end{Shaded}

\begin{verbatim}
## ['BPF', 'LOG4', 'NV_MAGICCONST', 'RECIP_BPF', 'Random', 'SG_MAGICCONST', 'SystemRandom', 'TWOPI', '_Sequence', '_Set', '__all__', '__builtins__', '__cached__', '__doc__', '__file__', '__loader__', '__name__', '__package__', '__spec__', '_accumulate', '_acos', '_bisect', '_ceil', '_cos', '_e', '_exp', '_inst', '_log', '_os', '_pi', '_random', '_repeat', '_sha512', '_sin', '_sqrt', '_test', '_test_generator', '_urandom', '_warn', 'betavariate', 'choice', 'choices', 'expovariate', 'gammavariate', 'gauss', 'getrandbits', 'getstate', 'lognormvariate', 'normalvariate', 'paretovariate', 'randint', 'random', 'randrange', 'sample', 'seed', 'setstate', 'shuffle', 'triangular', 'uniform', 'vonmisesvariate', 'weibullvariate']
\end{verbatim}
\end{frame}

\begin{frame}[fragile]{MCQ Module 4}
\protect\hypertarget{mcq-module-4}{}
\begin{itemize}
\tightlist
\item
  What is the correct syntax of importing only the randint function of
  the ``random'' module?
\end{itemize}

\begin{Shaded}
\begin{Highlighting}[]
\NormalTok{??? random ??? randint}
\end{Highlighting}
\end{Shaded}
\end{frame}

\begin{frame}[fragile]{MCQ Module 4}
\protect\hypertarget{mcq-module-4-1}{}
\begin{itemize}
\tightlist
\item
  What is the correct syntax of importing only the randint function of
  the ``random'' module?
\end{itemize}

\begin{Shaded}
\begin{Highlighting}[]
\ImportTok{from}\NormalTok{ random }\ImportTok{import}\NormalTok{ randint}
\end{Highlighting}
\end{Shaded}
\end{frame}

\end{document}
