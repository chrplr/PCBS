% Options for packages loaded elsewhere
\PassOptionsToPackage{unicode}{hyperref}
\PassOptionsToPackage{hyphens}{url}
%
\documentclass[
  8pt,
  ignorenonframetext,
]{beamer}
\usepackage{pgfpages}
\setbeamertemplate{caption}[numbered]
\setbeamertemplate{caption label separator}{: }
\setbeamercolor{caption name}{fg=normal text.fg}
\beamertemplatenavigationsymbolsempty
% Prevent slide breaks in the middle of a paragraph
\widowpenalties 1 10000
\raggedbottom
\setbeamertemplate{part page}{
  \centering
  \begin{beamercolorbox}[sep=16pt,center]{part title}
    \usebeamerfont{part title}\insertpart\par
  \end{beamercolorbox}
}
\setbeamertemplate{section page}{
  \centering
  \begin{beamercolorbox}[sep=12pt,center]{part title}
    \usebeamerfont{section title}\insertsection\par
  \end{beamercolorbox}
}
\setbeamertemplate{subsection page}{
  \centering
  \begin{beamercolorbox}[sep=8pt,center]{part title}
    \usebeamerfont{subsection title}\insertsubsection\par
  \end{beamercolorbox}
}
\AtBeginPart{
  \frame{\partpage}
}
\AtBeginSection{
  \ifbibliography
  \else
    \frame{\sectionpage}
  \fi
}
\AtBeginSubsection{
  \frame{\subsectionpage}
}
\usepackage{amsmath,amssymb}
\usepackage{lmodern}
\usepackage{iftex}
\ifPDFTeX
  \usepackage[T1]{fontenc}
  \usepackage[utf8]{inputenc}
  \usepackage{textcomp} % provide euro and other symbols
\else % if luatex or xetex
  \usepackage{unicode-math}
  \defaultfontfeatures{Scale=MatchLowercase}
  \defaultfontfeatures[\rmfamily]{Ligatures=TeX,Scale=1}
\fi
\usetheme[]{CambridgeUS}
% Use upquote if available, for straight quotes in verbatim environments
\IfFileExists{upquote.sty}{\usepackage{upquote}}{}
\IfFileExists{microtype.sty}{% use microtype if available
  \usepackage[]{microtype}
  \UseMicrotypeSet[protrusion]{basicmath} % disable protrusion for tt fonts
}{}
\makeatletter
\@ifundefined{KOMAClassName}{% if non-KOMA class
  \IfFileExists{parskip.sty}{%
    \usepackage{parskip}
  }{% else
    \setlength{\parindent}{0pt}
    \setlength{\parskip}{6pt plus 2pt minus 1pt}}
}{% if KOMA class
  \KOMAoptions{parskip=half}}
\makeatother
\usepackage{xcolor}
\newif\ifbibliography
\usepackage{color}
\usepackage{fancyvrb}
\newcommand{\VerbBar}{|}
\newcommand{\VERB}{\Verb[commandchars=\\\{\}]}
\DefineVerbatimEnvironment{Highlighting}{Verbatim}{commandchars=\\\{\}}
% Add ',fontsize=\small' for more characters per line
\usepackage{framed}
\definecolor{shadecolor}{RGB}{248,248,248}
\newenvironment{Shaded}{\begin{snugshade}}{\end{snugshade}}
\newcommand{\AlertTok}[1]{\textcolor[rgb]{0.94,0.16,0.16}{#1}}
\newcommand{\AnnotationTok}[1]{\textcolor[rgb]{0.56,0.35,0.01}{\textbf{\textit{#1}}}}
\newcommand{\AttributeTok}[1]{\textcolor[rgb]{0.77,0.63,0.00}{#1}}
\newcommand{\BaseNTok}[1]{\textcolor[rgb]{0.00,0.00,0.81}{#1}}
\newcommand{\BuiltInTok}[1]{#1}
\newcommand{\CharTok}[1]{\textcolor[rgb]{0.31,0.60,0.02}{#1}}
\newcommand{\CommentTok}[1]{\textcolor[rgb]{0.56,0.35,0.01}{\textit{#1}}}
\newcommand{\CommentVarTok}[1]{\textcolor[rgb]{0.56,0.35,0.01}{\textbf{\textit{#1}}}}
\newcommand{\ConstantTok}[1]{\textcolor[rgb]{0.00,0.00,0.00}{#1}}
\newcommand{\ControlFlowTok}[1]{\textcolor[rgb]{0.13,0.29,0.53}{\textbf{#1}}}
\newcommand{\DataTypeTok}[1]{\textcolor[rgb]{0.13,0.29,0.53}{#1}}
\newcommand{\DecValTok}[1]{\textcolor[rgb]{0.00,0.00,0.81}{#1}}
\newcommand{\DocumentationTok}[1]{\textcolor[rgb]{0.56,0.35,0.01}{\textbf{\textit{#1}}}}
\newcommand{\ErrorTok}[1]{\textcolor[rgb]{0.64,0.00,0.00}{\textbf{#1}}}
\newcommand{\ExtensionTok}[1]{#1}
\newcommand{\FloatTok}[1]{\textcolor[rgb]{0.00,0.00,0.81}{#1}}
\newcommand{\FunctionTok}[1]{\textcolor[rgb]{0.00,0.00,0.00}{#1}}
\newcommand{\ImportTok}[1]{#1}
\newcommand{\InformationTok}[1]{\textcolor[rgb]{0.56,0.35,0.01}{\textbf{\textit{#1}}}}
\newcommand{\KeywordTok}[1]{\textcolor[rgb]{0.13,0.29,0.53}{\textbf{#1}}}
\newcommand{\NormalTok}[1]{#1}
\newcommand{\OperatorTok}[1]{\textcolor[rgb]{0.81,0.36,0.00}{\textbf{#1}}}
\newcommand{\OtherTok}[1]{\textcolor[rgb]{0.56,0.35,0.01}{#1}}
\newcommand{\PreprocessorTok}[1]{\textcolor[rgb]{0.56,0.35,0.01}{\textit{#1}}}
\newcommand{\RegionMarkerTok}[1]{#1}
\newcommand{\SpecialCharTok}[1]{\textcolor[rgb]{0.00,0.00,0.00}{#1}}
\newcommand{\SpecialStringTok}[1]{\textcolor[rgb]{0.31,0.60,0.02}{#1}}
\newcommand{\StringTok}[1]{\textcolor[rgb]{0.31,0.60,0.02}{#1}}
\newcommand{\VariableTok}[1]{\textcolor[rgb]{0.00,0.00,0.00}{#1}}
\newcommand{\VerbatimStringTok}[1]{\textcolor[rgb]{0.31,0.60,0.02}{#1}}
\newcommand{\WarningTok}[1]{\textcolor[rgb]{0.56,0.35,0.01}{\textbf{\textit{#1}}}}
\setlength{\emergencystretch}{3em} % prevent overfull lines
\providecommand{\tightlist}{%
  \setlength{\itemsep}{0pt}\setlength{\parskip}{0pt}}
\setcounter{secnumdepth}{-\maxdimen} % remove section numbering
\let\verbatim\undefined
\let\verbatimend\undefined
\usepackage{listings}
\lstnewenvironment{verbatim}{\lstset{breaklines=true,basicstyle=\ttfamily\footnotesize}}{}
\ifLuaTeX
  \usepackage{selnolig}  % disable illegal ligatures
\fi
\IfFileExists{bookmark.sty}{\usepackage{bookmark}}{\usepackage{hyperref}}
\IfFileExists{xurl.sty}{\usepackage{xurl}}{} % add URL line breaks if available
\urlstyle{same} % disable monospaced font for URLs
\hypersetup{
  pdftitle={Intro to programming 9},
  hidelinks,
  pdfcreator={LaTeX via pandoc}}

\title{Intro to programming 9}
\author{Henri Vandendriessche\\
\href{mailto:henri.vandendriessche@ens.fr}{\nolinkurl{henri.vandendriessche@ens.fr}}}
\date{2022-11-29}

\begin{document}
\frame{\titlepage}

\begin{frame}{Exercise on pdb 1/4}
\protect\hypertarget{exercise-on-pdb-14}{}
\begin{itemize}
\tightlist
\item
  Use my exercise correction from last week on the cameras and:

  \begin{itemize}
  \tightlist
  \item
    1 execute line by line the program using pdb
  \item
    2 print argument(s) of all function and check its type
  \item
    3 set breakpoints right before every return statement of a function
  \end{itemize}
\end{itemize}
\end{frame}

\begin{frame}{Exercise on logging 2/4}
\protect\hypertarget{exercise-on-logging-24}{}
\begin{itemize}
\tightlist
\item
  Use my exercise correction from last week on the cameras and:

  \begin{itemize}
  \tightlist
  \item
    1 log every argument/parameter sent to a function
  \item
    2 log every variable in a return statement
  \end{itemize}
\end{itemize}
\end{frame}

\begin{frame}[fragile]{Exercise on pdb 3/4}
\protect\hypertarget{exercise-on-pdb-34}{}
\begin{itemize}
\tightlist
\item
  Debug the following adding program:
\end{itemize}

\begin{Shaded}
\begin{Highlighting}[]
\BuiltInTok{print}\NormalTok{(}\StringTok{\textquotesingle{}Enter the first number to add:\textquotesingle{}}\NormalTok{)}
\NormalTok{first }\OperatorTok{=} \BuiltInTok{input}\NormalTok{()}
\BuiltInTok{print}\NormalTok{(}\StringTok{\textquotesingle{}Enter the second number to add:\textquotesingle{}}\NormalTok{)}
\NormalTok{second }\OperatorTok{=} \BuiltInTok{input}\NormalTok{()}
\BuiltInTok{print}\NormalTok{(}\StringTok{\textquotesingle{}Enter the third number to add:\textquotesingle{}}\NormalTok{)}
\NormalTok{third }\OperatorTok{=} \BuiltInTok{input}\NormalTok{()}
\BuiltInTok{print}\NormalTok{(}\StringTok{\textquotesingle{}The sum is \textquotesingle{}} \OperatorTok{+}\NormalTok{ first }\OperatorTok{+}\NormalTok{ second }\OperatorTok{+}\NormalTok{ third)}
\end{Highlighting}
\end{Shaded}
\end{frame}

\begin{frame}[fragile]{Exercise on pdb 4/4}
\protect\hypertarget{exercise-on-pdb-44}{}
\begin{itemize}
\tightlist
\item
  Debug the following coin toss program:
\end{itemize}

\begin{Shaded}
\begin{Highlighting}[]
\ImportTok{import}\NormalTok{ random}
\NormalTok{guess }\OperatorTok{=} \StringTok{\textquotesingle{}\textquotesingle{}}
\ControlFlowTok{while}\NormalTok{ guess }\KeywordTok{not} \KeywordTok{in}\NormalTok{ (}\StringTok{\textquotesingle{}heads\textquotesingle{}}\NormalTok{, }\StringTok{\textquotesingle{}tails\textquotesingle{}}\NormalTok{):}
    \BuiltInTok{print}\NormalTok{(}\StringTok{\textquotesingle{}Guess the coin toss! Enter heads or tails:\textquotesingle{}}\NormalTok{)}
\NormalTok{    guess }\OperatorTok{=} \BuiltInTok{input}\NormalTok{()}
\NormalTok{toss }\OperatorTok{=}\NormalTok{ random.randint(}\DecValTok{0}\NormalTok{, }\DecValTok{1}\NormalTok{) }\CommentTok{\# 0 is tails, 1 is heads}
\ControlFlowTok{if}\NormalTok{ toss }\OperatorTok{==}\NormalTok{ guess:}
    \BuiltInTok{print}\NormalTok{(}\StringTok{\textquotesingle{}You got it!\textquotesingle{}}\NormalTok{)}
\ControlFlowTok{else}\NormalTok{:}
    \BuiltInTok{print}\NormalTok{(}\StringTok{\textquotesingle{}Nope! Guess again!\textquotesingle{}}\NormalTok{)}
\NormalTok{    guesss }\OperatorTok{=} \BuiltInTok{input}\NormalTok{()}
    \ControlFlowTok{if}\NormalTok{ toss }\OperatorTok{==}\NormalTok{ guess:}
        \BuiltInTok{print}\NormalTok{(}\StringTok{\textquotesingle{}You got it!\textquotesingle{}}\NormalTok{)}
    \ControlFlowTok{else}\NormalTok{:}
        \BuiltInTok{print}\NormalTok{(}\StringTok{\textquotesingle{}Nope. You are really bad at this game.\textquotesingle{}}\NormalTok{)}
\end{Highlighting}
\end{Shaded}
\end{frame}

\begin{frame}{Exercise 1}
\protect\hypertarget{exercise-1}{}
\begin{itemize}
\item
  Write a program that asks for the first name, last name, age and date
  of birth of a user. Then create one or several functions that perform
  input validation
\item
  Check that the age is a valid number,
\item
  Check that the first name and last name are only letters
\item
  Check that the date of birth is in the valid format dd/mm/yyy and is
  coherent with the their age
\item
  Otherwise, ask again the invalid data
\end{itemize}
\end{frame}

\begin{frame}{Exercise 2}
\protect\hypertarget{exercise-2}{}
\begin{itemize}
\tightlist
\item
  Write a Python program to create a Caesar cipher
  \url{https://en.wikipedia.org/wiki/Caesar_cipher}.
\item
  With one function to encrypt and decrypt the text with a shift.
  Positive shift for encryption and negative shift for decryption.
\item
  The program should consider uppercase and lower case
\item
  Hint: check the function ord(), char() and unicode table.
\end{itemize}
\end{frame}

\begin{frame}{Exercise 3}
\protect\hypertarget{exercise-3}{}
\begin{itemize}
\tightlist
\item
  Write a program that try to perform a Brute-force attack:
\item
  using these collection of characters: first
  ``abcdefghijklmnopqrstuvwxyz'' then
  ``abcdefghijklmnopqrstuvwxyz1234567890'' and finally
  ``abcdefghijklmnopqrstuvwxyzABCDEFGHIJKLMNOPQRSTUVWXYZ1234567890,;:!?./§\%*+=''
\item
  Ask the user a password (1 - 6 character)
\item
  Then write a program that try every possible combination of our set of
  character with all possible length (from 1 to 5)
\item
  Record and print the time to find the solution
\item
  Write at every step the number length you are testing and time taken
  so far.
\item
  To continue, calculate the number of combination according to the
  length of the password and the collection of character selected.
\item
  Hint: check itertools module
\end{itemize}
\end{frame}

\end{document}
