% Options for packages loaded elsewhere
\PassOptionsToPackage{unicode}{hyperref}
\PassOptionsToPackage{hyphens}{url}
%
\documentclass[
  8pt,
  ignorenonframetext,
]{beamer}
\usepackage{pgfpages}
\setbeamertemplate{caption}[numbered]
\setbeamertemplate{caption label separator}{: }
\setbeamercolor{caption name}{fg=normal text.fg}
\beamertemplatenavigationsymbolsempty
% Prevent slide breaks in the middle of a paragraph
\widowpenalties 1 10000
\raggedbottom
\setbeamertemplate{part page}{
  \centering
  \begin{beamercolorbox}[sep=16pt,center]{part title}
    \usebeamerfont{part title}\insertpart\par
  \end{beamercolorbox}
}
\setbeamertemplate{section page}{
  \centering
  \begin{beamercolorbox}[sep=12pt,center]{part title}
    \usebeamerfont{section title}\insertsection\par
  \end{beamercolorbox}
}
\setbeamertemplate{subsection page}{
  \centering
  \begin{beamercolorbox}[sep=8pt,center]{part title}
    \usebeamerfont{subsection title}\insertsubsection\par
  \end{beamercolorbox}
}
\AtBeginPart{
  \frame{\partpage}
}
\AtBeginSection{
  \ifbibliography
  \else
    \frame{\sectionpage}
  \fi
}
\AtBeginSubsection{
  \frame{\subsectionpage}
}
\usepackage{amsmath,amssymb}
\usepackage{iftex}
\ifPDFTeX
  \usepackage[T1]{fontenc}
  \usepackage[utf8]{inputenc}
  \usepackage{textcomp} % provide euro and other symbols
\else % if luatex or xetex
  \usepackage{unicode-math} % this also loads fontspec
  \defaultfontfeatures{Scale=MatchLowercase}
  \defaultfontfeatures[\rmfamily]{Ligatures=TeX,Scale=1}
\fi
\usepackage{lmodern}
\usetheme[]{CambridgeUS}
\ifPDFTeX\else
  % xetex/luatex font selection
\fi
% Use upquote if available, for straight quotes in verbatim environments
\IfFileExists{upquote.sty}{\usepackage{upquote}}{}
\IfFileExists{microtype.sty}{% use microtype if available
  \usepackage[]{microtype}
  \UseMicrotypeSet[protrusion]{basicmath} % disable protrusion for tt fonts
}{}
\makeatletter
\@ifundefined{KOMAClassName}{% if non-KOMA class
  \IfFileExists{parskip.sty}{%
    \usepackage{parskip}
  }{% else
    \setlength{\parindent}{0pt}
    \setlength{\parskip}{6pt plus 2pt minus 1pt}}
}{% if KOMA class
  \KOMAoptions{parskip=half}}
\makeatother
\usepackage{xcolor}
\newif\ifbibliography
\usepackage{color}
\usepackage{fancyvrb}
\newcommand{\VerbBar}{|}
\newcommand{\VERB}{\Verb[commandchars=\\\{\}]}
\DefineVerbatimEnvironment{Highlighting}{Verbatim}{commandchars=\\\{\}}
% Add ',fontsize=\small' for more characters per line
\usepackage{framed}
\definecolor{shadecolor}{RGB}{248,248,248}
\newenvironment{Shaded}{\begin{snugshade}}{\end{snugshade}}
\newcommand{\AlertTok}[1]{\textcolor[rgb]{0.94,0.16,0.16}{#1}}
\newcommand{\AnnotationTok}[1]{\textcolor[rgb]{0.56,0.35,0.01}{\textbf{\textit{#1}}}}
\newcommand{\AttributeTok}[1]{\textcolor[rgb]{0.13,0.29,0.53}{#1}}
\newcommand{\BaseNTok}[1]{\textcolor[rgb]{0.00,0.00,0.81}{#1}}
\newcommand{\BuiltInTok}[1]{#1}
\newcommand{\CharTok}[1]{\textcolor[rgb]{0.31,0.60,0.02}{#1}}
\newcommand{\CommentTok}[1]{\textcolor[rgb]{0.56,0.35,0.01}{\textit{#1}}}
\newcommand{\CommentVarTok}[1]{\textcolor[rgb]{0.56,0.35,0.01}{\textbf{\textit{#1}}}}
\newcommand{\ConstantTok}[1]{\textcolor[rgb]{0.56,0.35,0.01}{#1}}
\newcommand{\ControlFlowTok}[1]{\textcolor[rgb]{0.13,0.29,0.53}{\textbf{#1}}}
\newcommand{\DataTypeTok}[1]{\textcolor[rgb]{0.13,0.29,0.53}{#1}}
\newcommand{\DecValTok}[1]{\textcolor[rgb]{0.00,0.00,0.81}{#1}}
\newcommand{\DocumentationTok}[1]{\textcolor[rgb]{0.56,0.35,0.01}{\textbf{\textit{#1}}}}
\newcommand{\ErrorTok}[1]{\textcolor[rgb]{0.64,0.00,0.00}{\textbf{#1}}}
\newcommand{\ExtensionTok}[1]{#1}
\newcommand{\FloatTok}[1]{\textcolor[rgb]{0.00,0.00,0.81}{#1}}
\newcommand{\FunctionTok}[1]{\textcolor[rgb]{0.13,0.29,0.53}{\textbf{#1}}}
\newcommand{\ImportTok}[1]{#1}
\newcommand{\InformationTok}[1]{\textcolor[rgb]{0.56,0.35,0.01}{\textbf{\textit{#1}}}}
\newcommand{\KeywordTok}[1]{\textcolor[rgb]{0.13,0.29,0.53}{\textbf{#1}}}
\newcommand{\NormalTok}[1]{#1}
\newcommand{\OperatorTok}[1]{\textcolor[rgb]{0.81,0.36,0.00}{\textbf{#1}}}
\newcommand{\OtherTok}[1]{\textcolor[rgb]{0.56,0.35,0.01}{#1}}
\newcommand{\PreprocessorTok}[1]{\textcolor[rgb]{0.56,0.35,0.01}{\textit{#1}}}
\newcommand{\RegionMarkerTok}[1]{#1}
\newcommand{\SpecialCharTok}[1]{\textcolor[rgb]{0.81,0.36,0.00}{\textbf{#1}}}
\newcommand{\SpecialStringTok}[1]{\textcolor[rgb]{0.31,0.60,0.02}{#1}}
\newcommand{\StringTok}[1]{\textcolor[rgb]{0.31,0.60,0.02}{#1}}
\newcommand{\VariableTok}[1]{\textcolor[rgb]{0.00,0.00,0.00}{#1}}
\newcommand{\VerbatimStringTok}[1]{\textcolor[rgb]{0.31,0.60,0.02}{#1}}
\newcommand{\WarningTok}[1]{\textcolor[rgb]{0.56,0.35,0.01}{\textbf{\textit{#1}}}}
\usepackage{graphicx}
\makeatletter
\def\maxwidth{\ifdim\Gin@nat@width>\linewidth\linewidth\else\Gin@nat@width\fi}
\def\maxheight{\ifdim\Gin@nat@height>\textheight\textheight\else\Gin@nat@height\fi}
\makeatother
% Scale images if necessary, so that they will not overflow the page
% margins by default, and it is still possible to overwrite the defaults
% using explicit options in \includegraphics[width, height, ...]{}
\setkeys{Gin}{width=\maxwidth,height=\maxheight,keepaspectratio}
% Set default figure placement to htbp
\makeatletter
\def\fps@figure{htbp}
\makeatother
\setlength{\emergencystretch}{3em} % prevent overfull lines
\providecommand{\tightlist}{%
  \setlength{\itemsep}{0pt}\setlength{\parskip}{0pt}}
\setcounter{secnumdepth}{-\maxdimen} % remove section numbering
\let\verbatim\undefined
\let\verbatimend\undefined
\usepackage{listings}
\lstnewenvironment{verbatim}{\lstset{breaklines=true,basicstyle=\ttfamily\footnotesize}}{}
\ifLuaTeX
  \usepackage{selnolig}  % disable illegal ligatures
\fi
\IfFileExists{bookmark.sty}{\usepackage{bookmark}}{\usepackage{hyperref}}
\IfFileExists{xurl.sty}{\usepackage{xurl}}{} % add URL line breaks if available
\urlstyle{same}
\hypersetup{
  pdftitle={Intro to programming 8},
  hidelinks,
  pdfcreator={LaTeX via pandoc}}

\title{Intro to programming 8}
\author{Henri Vandendriessche\\
\href{mailto:henri.vandendriessche@ens.fr}{\nolinkurl{henri.vandendriessche@ens.fr}}}
\date{2023-11-21}

\begin{document}
\frame{\titlepage}

\begin{frame}[fragile]{Terminal cheat sheet reminder}
\protect\hypertarget{terminal-cheat-sheet-reminder}{}
\begin{itemize}
\item
  Bash commands to navigate directories

  \begin{itemize}
  \tightlist
  \item
    Print Working Directory. Print the path of the current directory
  \end{itemize}

\begin{Shaded}
\begin{Highlighting}[]
\BuiltInTok{pwd}
\end{Highlighting}
\end{Shaded}

  \begin{itemize}
  \tightlist
  \item
    List all files of the current directory
  \end{itemize}

\begin{Shaded}
\begin{Highlighting}[]
\FunctionTok{ls}\NormalTok{ folder}
\end{Highlighting}
\end{Shaded}

  \begin{itemize}
  \tightlist
  \item
    Moving into folder1 and subfolder2 at once.
  \end{itemize}

\begin{Shaded}
\begin{Highlighting}[]
\BuiltInTok{cd}\NormalTok{ folder1/subfolder2}
\end{Highlighting}
\end{Shaded}

  \begin{itemize}
  \tightlist
  \item
    Moving out of a directory
  \end{itemize}

\begin{Shaded}
\begin{Highlighting}[]
\BuiltInTok{cd}\NormalTok{ ..}
\end{Highlighting}
\end{Shaded}

  \begin{itemize}
  \tightlist
  \item
    Going back and forth in the directory tree
  \end{itemize}

\begin{Shaded}
\begin{Highlighting}[]
\BuiltInTok{cd}\NormalTok{ ../../folder1/subfolder1}
\end{Highlighting}
\end{Shaded}

  \begin{itemize}
  \tightlist
  \item
    Going back to the root directory
  \end{itemize}

\begin{Shaded}
\begin{Highlighting}[]
\BuiltInTok{cd}\NormalTok{ \textasciitilde{}}
\end{Highlighting}
\end{Shaded}
\item
  ``\textbf{Tab}'' to use the auto-completion
\item
  \textbf{Ctrl + C} to stop a program execution
\item
  ``\textbf{Upper arrow}'' to see last commands
\item
  Many more bash commands to use\ldots{}
\end{itemize}
\end{frame}

\begin{frame}{Where are we now}
\protect\hypertarget{where-are-we-now}{}
\begin{itemize}[<+->]
\tightlist
\item
  As programs become increasingly complex, we find ourselves dealing
  with more intricate situations.
\end{itemize}

\begin{itemize}[<+->]
\tightlist
\item
  Consequently, more complex bugs may arise.
\end{itemize}

\begin{itemize}[<+->]
\tightlist
\item
  Remember, your computer only executes what you instruct it to do; it
  can't read your mind and perform your intended actions.
\end{itemize}

\begin{itemize}[<+->]
\tightlist
\item
  Everyone creates bugs, and everyone has to correct them.
\end{itemize}

\begin{itemize}[<+->]
\tightlist
\item
  Fortunately, Python comes with tools to assist you in overcoming these
  challenges.
\end{itemize}
\end{frame}

\begin{frame}{Today}
\protect\hypertarget{today}{}
\begin{itemize}
\item
  Debugging level 0
\item
  Assertion
\item
  Logging
\item
  pdb module
\end{itemize}
\end{frame}

\begin{frame}{Disclaimer}
\protect\hypertarget{disclaimer}{}
\begin{itemize}
\item
  This class is highly based on \textbf{Automate the Boring Stuff with
  Python} chapter 11\ldots{}
\item
  \url{https://automatetheboringstuff.com/2e/chapter11/}
\end{itemize}
\end{frame}

\begin{frame}{The easiest debugging rule}
\protect\hypertarget{the-easiest-debugging-rule}{}
\begin{itemize}[<+->]
\tightlist
\item
  When your program does what you instructed (what you wrote) but not
  what you intended\ldots{}
\end{itemize}

\begin{itemize}[<+->]
\tightlist
\item
  The simplest and most straightforward way to identify issues in your
  program is to check your variables at key points:

  \begin{itemize}[<+->]
  \tightlist
  \item
    After performing an operation on your variable.
  \item
    At the end of a loop.
  \item
    At the end of a function, especially if there is a return statement.
  \item
    When importing data from a file.
  \end{itemize}
\end{itemize}

\begin{itemize}[<+->]
\tightlist
\item
  How can you best check your variables and their types?

  \begin{itemize}[<+->]
  \tightlist
  \item
    print the variable: \textbf{print(your\_variable)}
  \item
    print the type of the variable: \textbf{print(type(your\_variable))}
  \end{itemize}
\end{itemize}

\begin{itemize}[<+->]
\tightlist
\item
  While \textbf{print()} can be useful for simple checks, it's advisable
  not to debug your entire script with print().
\end{itemize}

\begin{itemize}[<+->]
\tightlist
\item
  Consider it as the initial level (level 0) of debugging.
\end{itemize}
\end{frame}

\begin{frame}[fragile]{Errors and exceptions}
\protect\hypertarget{errors-and-exceptions}{}
\begin{itemize}
\item
  Until now we haven't spent much time looking at the error messages.
\item
  There are (at least) two distinguishable kinds of errors: syntax
  errors and exceptions.
\item
  Example of syntax errors:
\end{itemize}

\begin{Shaded}
\begin{Highlighting}[]
\NormalTok{test }\OperatorTok{=} \VariableTok{True}
\ControlFlowTok{if}\NormalTok{ test }\OperatorTok{==} \VariableTok{False}
  \BuiltInTok{print}\NormalTok{(}\StringTok{"Failed"}\NormalTok{)}
\end{Highlighting}
\end{Shaded}

\begin{verbatim}
## invalid syntax (<string>, line 2)
\end{verbatim}
\end{frame}

\begin{frame}{Handling exceptions}
\protect\hypertarget{handling-exceptions}{}
\begin{itemize}
\item
  Even if a statement or expression is syntactically correct, it may
  cause an error when an attempt is made to execute it.
\item
  Errors detected during execution are called exceptions and are not
  unconditionally fatal.
\item
  To handle exceptions there are several options:

  \begin{itemize}
  \tightlist
  \item
    The \textbf{raise} statement.
  \item
    The \textbf{try} statement.
  \end{itemize}
\end{itemize}
\end{frame}

\begin{frame}[fragile]{Raising Exceptions 1/2}
\protect\hypertarget{raising-exceptions-12}{}
\begin{itemize}
\item
  Python raises exceptions every time it attempts to execute invalid
  code.
\item
  Exceptions are raised using the following:

  \begin{itemize}
  \tightlist
  \item
    The \textbf{raise} keyword
  \item
    The \textbf{Exception()} function
  \item
    A descriptive sentence that helps you understand the problem in the
    Exception function.
  \end{itemize}
\end{itemize}

\begin{Shaded}
\begin{Highlighting}[]
\ControlFlowTok{raise} \PreprocessorTok{Exception}\NormalTok{(}\StringTok{\textquotesingle{}Ah Shit, Here We Go Again: another day another bug\textquotesingle{}}\NormalTok{)}
\end{Highlighting}
\end{Shaded}

\begin{verbatim}
## Ah Shit, Here We Go Again: another day another bug
\end{verbatim}
\end{frame}

\begin{frame}[fragile]{Raising Exceptions 2/2}
\protect\hypertarget{raising-exceptions-22}{}
\begin{itemize}
\item
  When / how to raise exception ?
\item
  Exception can be passed as argument or returned at the end of a
  function
\end{itemize}

\begin{Shaded}
\begin{Highlighting}[]
\KeywordTok{def}\NormalTok{ doBullshit():}
  \ControlFlowTok{raise} \PreprocessorTok{Exception}\NormalTok{(}\StringTok{\textquotesingle{}I did bullshit\textquotesingle{}}\NormalTok{)}

\NormalTok{doBullshit()}
\end{Highlighting}
\end{Shaded}

\begin{verbatim}
## I did bullshit
\end{verbatim}
\end{frame}

\begin{frame}[fragile]{Assertions 1/3}
\protect\hypertarget{assertions-13}{}
\begin{itemize}
\item
  Assertion is a sanity check to ensure that the data has the expected
  format.
\item
  If the sanity check fails, then an \textbf{AssertionError} exception
  is raised.
\item
  Assertions are raised in the following way:

  \begin{itemize}
  \tightlist
  \item
    Using the \textbf{assert} keyword
  \item
    Including a condition.
  \item
    Separating the condition with a comma.
  \item
    Providing a string to display when the check fails.
  \end{itemize}
\item
  Example 1:
\end{itemize}

\begin{Shaded}
\begin{Highlighting}[]
\NormalTok{olympicGamesYears }\OperatorTok{=}\NormalTok{[}\DecValTok{2021}\NormalTok{, }\DecValTok{2012}\NormalTok{, }\DecValTok{2008}\NormalTok{, }\DecValTok{2024}\NormalTok{, }\DecValTok{2016}\NormalTok{, }\DecValTok{2000}\NormalTok{, }\DecValTok{2004}\NormalTok{]}

\ControlFlowTok{assert}\NormalTok{ olympicGamesYears[}\DecValTok{0}\NormalTok{] }\OperatorTok{\textless{}}\NormalTok{ olympicGamesYears[}\OperatorTok{{-}}\DecValTok{1}\NormalTok{] , }\StringTok{"Years doesn\textquotesingle{}t seem sorted"}
\end{Highlighting}
\end{Shaded}

\begin{verbatim}
## Years doesn't seem sorted
\end{verbatim}
\end{frame}

\begin{frame}[fragile]{Assertions 2/3}
\protect\hypertarget{assertions-23}{}
\begin{itemize}
\item
  Assertion is a sanity check to ensure that the data has the expected
  format.
\item
  If the sanity check fails, then an \textbf{AssertionError} exception
  is raised.
\item
  Assertions are raised in the following way:

  \begin{itemize}
  \tightlist
  \item
    Using the \textbf{assert} keyword
  \item
    Including a condition.
  \item
    Separating the condition with a comma.
  \item
    Providing a string to display when the check fails.
  \end{itemize}
\item
  Example 2:
\end{itemize}

\begin{Shaded}
\begin{Highlighting}[]
\NormalTok{olympicGamesYears }\OperatorTok{=}\NormalTok{[}\DecValTok{2004}\NormalTok{, }\DecValTok{2012}\NormalTok{, }\DecValTok{2008}\NormalTok{, }\DecValTok{2024}\NormalTok{, }\DecValTok{2016}\NormalTok{, }\DecValTok{2000}\NormalTok{, }\DecValTok{2021}\NormalTok{]}

\NormalTok{olympicGamesYears.sort()}

\ControlFlowTok{assert}\NormalTok{ olympicGamesYears[}\DecValTok{0}\NormalTok{] }\OperatorTok{\textless{}}\NormalTok{ olympicGamesYears[}\OperatorTok{{-}}\DecValTok{1}\NormalTok{] , }\StringTok{"Years doesn\textquotesingle{}t seem sorted"}
\end{Highlighting}
\end{Shaded}
\end{frame}

\begin{frame}{Assertions 3/3}
\protect\hypertarget{assertions-33}{}
\begin{itemize}
\tightlist
\item
  Assertion are very useful:

  \begin{itemize}
  \tightlist
  \item
    At key points to check that your data has the right format.
  \item
    If well designed, they provide you with the location and the exact
    reason why it failed.
  \item
    To save time in debugging.
  \end{itemize}
\item
  However, you need to first have a very concrete idea of what to expect
  and where in your program to set up some assertions.
\end{itemize}
\end{frame}

\begin{frame}[fragile]{Try and except statements 1/5}
\protect\hypertarget{try-and-except-statements-15}{}
\begin{itemize}[<+->]
\tightlist
\item
  If you encounter an error in your script, the execution is halted.
\end{itemize}

\begin{itemize}[<+->]
\item
  Example: What's wrong in the following script:

\begin{Shaded}
\begin{Highlighting}[]
\KeywordTok{def}\NormalTok{ isDivided(divisor):}
    \ControlFlowTok{return} \DecValTok{42} \OperatorTok{/}\NormalTok{ divisor}

\BuiltInTok{print}\NormalTok{(isDivided(}\DecValTok{2}\NormalTok{))}
\BuiltInTok{print}\NormalTok{(isDivided(}\DecValTok{12}\NormalTok{))}
\BuiltInTok{print}\NormalTok{(isDivided(}\DecValTok{0}\NormalTok{))}
\BuiltInTok{print}\NormalTok{(isDivided(}\DecValTok{3}\NormalTok{))}
\end{Highlighting}
\end{Shaded}
\end{itemize}
\end{frame}

\begin{frame}[fragile]{Try and except statements 2/5}
\protect\hypertarget{try-and-except-statements-25}{}
\begin{itemize}
\item
  If you encounter an error in your script, the execution is halted.
\item
  Example: What's wrong in the following script:

\begin{Shaded}
\begin{Highlighting}[]
\KeywordTok{def}\NormalTok{ isDivided(divisor):}
    \ControlFlowTok{return} \DecValTok{42} \OperatorTok{/}\NormalTok{ divisor}

\BuiltInTok{print}\NormalTok{(isDivided(}\DecValTok{2}\NormalTok{))}
\end{Highlighting}
\end{Shaded}

\begin{verbatim}
## 21.0
\end{verbatim}

\begin{Shaded}
\begin{Highlighting}[]
\BuiltInTok{print}\NormalTok{(isDivided(}\DecValTok{12}\NormalTok{))}
\end{Highlighting}
\end{Shaded}

\begin{verbatim}
## 3.5
\end{verbatim}

\begin{Shaded}
\begin{Highlighting}[]
\BuiltInTok{print}\NormalTok{(isDivided(}\DecValTok{0}\NormalTok{))}
\end{Highlighting}
\end{Shaded}

\begin{verbatim}
## division by zero
\end{verbatim}
\end{itemize}
\end{frame}

\begin{frame}[fragile]{Try and except statements 3/5}
\protect\hypertarget{try-and-except-statements-35}{}
\begin{itemize}
\tightlist
\item
  But you can still have your way around this error:

  \begin{itemize}
  \tightlist
  \item
    \textbf{try : }
  \item
    \textbf{except \ldots{} : }
  \end{itemize}
\end{itemize}

\begin{Shaded}
\begin{Highlighting}[]
\KeywordTok{def}\NormalTok{ isDivided(divisor):}
    \ControlFlowTok{try}\NormalTok{:}
      \ControlFlowTok{return} \DecValTok{42} \OperatorTok{/}\NormalTok{ divisor}
    \ControlFlowTok{except} \PreprocessorTok{ZeroDivisionError}\NormalTok{:}
      \BuiltInTok{print}\NormalTok{(}\StringTok{"What have I done again..."}\NormalTok{)}

\BuiltInTok{print}\NormalTok{(isDivided(}\DecValTok{2}\NormalTok{))}
\end{Highlighting}
\end{Shaded}

\begin{verbatim}
## 21.0
\end{verbatim}

\begin{Shaded}
\begin{Highlighting}[]
\BuiltInTok{print}\NormalTok{(isDivided(}\DecValTok{12}\NormalTok{))}
\end{Highlighting}
\end{Shaded}

\begin{verbatim}
## 3.5
\end{verbatim}

\begin{Shaded}
\begin{Highlighting}[]
\BuiltInTok{print}\NormalTok{(isDivided(}\DecValTok{0}\NormalTok{))}
\end{Highlighting}
\end{Shaded}

\begin{verbatim}
## What have I done again...
## None
\end{verbatim}

\begin{Shaded}
\begin{Highlighting}[]
\BuiltInTok{print}\NormalTok{(isDivided(}\DecValTok{3}\NormalTok{))}
\end{Highlighting}
\end{Shaded}

\begin{verbatim}
## 14.0
\end{verbatim}
\end{frame}

\begin{frame}[fragile]{Try and except statements 4/5}
\protect\hypertarget{try-and-except-statements-45}{}
\begin{itemize}[<+->]
\item
  You can also include the call to your function in the try block.

\begin{Shaded}
\begin{Highlighting}[]
\KeywordTok{def}\NormalTok{ isDivided(divisor):}
  \ControlFlowTok{return} \DecValTok{42} \OperatorTok{/}\NormalTok{ divisor}

\ControlFlowTok{try}\NormalTok{:}
  \BuiltInTok{print}\NormalTok{(isDivided(}\DecValTok{2}\NormalTok{))}
  \BuiltInTok{print}\NormalTok{(isDivided(}\DecValTok{12}\NormalTok{))}
  \BuiltInTok{print}\NormalTok{(isDivided(}\DecValTok{0}\NormalTok{))}
  \BuiltInTok{print}\NormalTok{(isDivided(}\DecValTok{3}\NormalTok{))}

\ControlFlowTok{except} \PreprocessorTok{ZeroDivisionError}\NormalTok{:}
  \BuiltInTok{print}\NormalTok{(}\StringTok{"What have I done again..."}\NormalTok{)}
\end{Highlighting}
\end{Shaded}

\begin{verbatim}
## 21.0
## 3.5
## What have I done again...
\end{verbatim}
\end{itemize}

\begin{itemize}[<+->]
\tightlist
\item
  Note that print(isDivided(3)) is not executed. Once the execution
  jumps to the except statement, it does not go back to the try clause.
  Instead, it just continues moving down the program as normal.
\end{itemize}
\end{frame}

\begin{frame}{Try and except statements 5/5}
\protect\hypertarget{try-and-except-statements-55}{}
\begin{itemize}
\tightlist
\item
  \textbf{try except} is useful:

  \begin{itemize}
  \tightlist
  \item
    To perform checks on your program flow.
  \item
    To obtain a (hopefully) clearer or more adapted error message than
    what Python can provide.
  \end{itemize}
\item
  \textbf{try except} is not useful:

  \begin{itemize}
  \tightlist
  \item
    To avoid errors without resolving them.
  \item
    To achieve a running program without a crash.
  \end{itemize}
\end{itemize}
\end{frame}

\begin{frame}[fragile]{Getting the Traceback as a String 1/2}
\protect\hypertarget{getting-the-traceback-as-a-string-12}{}
\begin{itemize}
\item
  Gathering information about your error.
\item
  When your program crashes, you always receive an error with
  information such as:

  \begin{itemize}
  \tightlist
  \item
    The line of the error / the different lines if your program uses
    several files.
  \item
    The error message
  \item
    The function / the sequence of functions involved (i,e, the call
    stack)
  \end{itemize}
\item
  All of this is referred to as the \textbf{traceback}
\item
  Example:
\end{itemize}

\begin{Shaded}
\begin{Highlighting}[]
\KeywordTok{def}\NormalTok{ callErrorTest():}
\NormalTok{  errorTest()}

\KeywordTok{def}\NormalTok{ errorTest():}
  \ControlFlowTok{raise} \PreprocessorTok{Exception}\NormalTok{(}\StringTok{\textquotesingle{}FATAL ERROR\textquotesingle{}}\NormalTok{)}

\NormalTok{callErrorTest()}
\end{Highlighting}
\end{Shaded}

\begin{verbatim}
## FATAL ERROR
\end{verbatim}

\begin{itemize}
\tightlist
\item
  That information is here to help you locate and understand you error.
\end{itemize}
\end{frame}

\begin{frame}[fragile]{Getting the Traceback as a String 2/2}
\protect\hypertarget{getting-the-traceback-as-a-string-22}{}
\begin{itemize}
\item
  Instead of just displaying it on your terminal, you can access your
  traceback using: \textbf{traceback.format\_exc()}
\item
  This way, you can obtain your traceback information as a string.
\item
  You'll need the \textbf{traceback} module to access the function.
\item
  It can be useful if you want to keep track of an error and write the
  information to a file. This allows you to keep it for later when
  you'll be mentally prepared to debug your code.
\end{itemize}

\begin{Shaded}
\begin{Highlighting}[]
\ImportTok{import}\NormalTok{ traceback}

\ControlFlowTok{try}\NormalTok{:}
  \ControlFlowTok{raise} \PreprocessorTok{Exception}\NormalTok{(}\StringTok{\textquotesingle{}FATAL ERROR\textquotesingle{}}\NormalTok{)}

\ControlFlowTok{except}\NormalTok{:}
\NormalTok{  errorFile }\OperatorTok{=} \BuiltInTok{open}\NormalTok{(}\StringTok{\textquotesingle{}errorInfo.txt\textquotesingle{}}\NormalTok{, }\StringTok{\textquotesingle{}w\textquotesingle{}}\NormalTok{)}
\NormalTok{  errorFile.write(traceback.format\_exc())}
\NormalTok{  errorFile.close()}
  \BuiltInTok{print}\NormalTok{(}\StringTok{\textquotesingle{}Don}\CharTok{\textbackslash{}\textquotesingle{}}\StringTok{t have time now to debug but all info are in errorInfo.txt\textquotesingle{}}\NormalTok{)}
\end{Highlighting}
\end{Shaded}

\begin{verbatim}
## 97
## Don't have time now to debug but all info are in errorInfo.txt
\end{verbatim}
\end{frame}

\begin{frame}{Logging 1/7}
\protect\hypertarget{logging-17}{}
\begin{itemize}
\item
  Logging involves writing down information or variable content from
  your script to keep track of the execution of your program.
\item
  \textbf{print()} serves as a form of logging.
\item
  Why is logging better than print :

  \begin{itemize}
  \tightlist
  \item
    You can access better information, such as timings, for example.
  \item
    It can be systematic and organized.
  \end{itemize}
\item
  Of course, Python has a logging module, and its name is
  \textbf{logging}: \url{https://docs.python.org/3/library/logging.html}
\end{itemize}
\end{frame}

\begin{frame}[fragile]{Logging 2/7}
\protect\hypertarget{logging-27}{}
\begin{itemize}
\tightlist
\item
  Example:
\end{itemize}

\begin{Shaded}
\begin{Highlighting}[]
\KeywordTok{def}\NormalTok{ factorial(n):}
\NormalTok{    total }\OperatorTok{=} \DecValTok{1}
    \ControlFlowTok{for}\NormalTok{ i }\KeywordTok{in} \BuiltInTok{range}\NormalTok{(n }\OperatorTok{+} \DecValTok{1}\NormalTok{):}
\NormalTok{        total }\OperatorTok{*=}\NormalTok{ i}
    \ControlFlowTok{return}\NormalTok{ total}

\BuiltInTok{print}\NormalTok{(factorial(}\DecValTok{5}\NormalTok{))}
\end{Highlighting}
\end{Shaded}
\end{frame}

\begin{frame}[fragile]{Logging 2/7}
\protect\hypertarget{logging-27-1}{}
\begin{itemize}
\tightlist
\item
  Example:
\end{itemize}

\begin{Shaded}
\begin{Highlighting}[]
\KeywordTok{def}\NormalTok{ factorial(n):}
\NormalTok{    total }\OperatorTok{=} \DecValTok{1}
    \ControlFlowTok{for}\NormalTok{ i }\KeywordTok{in} \BuiltInTok{range}\NormalTok{(n }\OperatorTok{+} \DecValTok{1}\NormalTok{):}
\NormalTok{        total }\OperatorTok{*=}\NormalTok{ i}
    \ControlFlowTok{return}\NormalTok{ total}

\BuiltInTok{print}\NormalTok{(factorial(}\DecValTok{5}\NormalTok{))}
\end{Highlighting}
\end{Shaded}

\begin{verbatim}
## 0
\end{verbatim}

\begin{itemize}
\tightlist
\item
  What will be the output ?
\end{itemize}
\end{frame}

\begin{frame}[fragile]{Logging 3/7}
\protect\hypertarget{logging-37}{}
\begin{itemize}
\tightlist
\item
  Example with logging:
\end{itemize}

\begin{Shaded}
\begin{Highlighting}[]
\ImportTok{import}\NormalTok{ logging}

\NormalTok{logging.basicConfig(level}\OperatorTok{=}\NormalTok{logging.DEBUG, }\BuiltInTok{format}\OperatorTok{=}\StringTok{\textquotesingle{}}\SpecialCharTok{\%(asctime)s}\StringTok{ {-}  }\SpecialCharTok{\%(levelname)s}\StringTok{ {-}  }\SpecialCharTok{\%(message)s}\StringTok{\textquotesingle{}}\NormalTok{)}
\NormalTok{logging.debug(}\StringTok{\textquotesingle{}Start of program\textquotesingle{}}\NormalTok{)}
\end{Highlighting}
\end{Shaded}

\begin{verbatim}
## 2023-11-21 14:19:56,526 -  DEBUG -  Start of program
\end{verbatim}

\begin{Shaded}
\begin{Highlighting}[]
\KeywordTok{def}\NormalTok{ factorial(n):}
\NormalTok{    logging.debug(}\StringTok{\textquotesingle{}Start of factorial(}\SpecialCharTok{\%s}\StringTok{)\textquotesingle{}}  \OperatorTok{\%}\NormalTok{ (n))}
\NormalTok{    total }\OperatorTok{=} \DecValTok{1}
    \ControlFlowTok{for}\NormalTok{ i }\KeywordTok{in} \BuiltInTok{range}\NormalTok{(n }\OperatorTok{+} \DecValTok{1}\NormalTok{):}
\NormalTok{        total }\OperatorTok{*=}\NormalTok{ i}
\NormalTok{        logging.debug(}\StringTok{\textquotesingle{}i is \textquotesingle{}} \OperatorTok{+} \BuiltInTok{str}\NormalTok{(i) }\OperatorTok{+} \StringTok{\textquotesingle{}, total is \textquotesingle{}} \OperatorTok{+} \BuiltInTok{str}\NormalTok{(total))}
\NormalTok{    logging.debug(}\StringTok{\textquotesingle{}End of factorial(}\SpecialCharTok{\%s}\StringTok{)\textquotesingle{}}  \OperatorTok{\%}\NormalTok{ (n))}
    \ControlFlowTok{return}\NormalTok{ total}

\NormalTok{logging.debug(}\StringTok{\textquotesingle{}Call of the function\textquotesingle{}}\NormalTok{)}
\end{Highlighting}
\end{Shaded}

\begin{verbatim}
## 2023-11-21 14:19:56,532 -  DEBUG -  Call of the function
\end{verbatim}

\begin{Shaded}
\begin{Highlighting}[]
\BuiltInTok{print}\NormalTok{(factorial(}\DecValTok{5}\NormalTok{))}
\end{Highlighting}
\end{Shaded}

\begin{verbatim}
## 0
## 
## 2023-11-21 14:19:56,534 -  DEBUG -  Start of factorial(5)
## 2023-11-21 14:19:56,534 -  DEBUG -  i is 0, total is 0
## 2023-11-21 14:19:56,534 -  DEBUG -  i is 1, total is 0
## 2023-11-21 14:19:56,534 -  DEBUG -  i is 2, total is 0
## 2023-11-21 14:19:56,534 -  DEBUG -  i is 3, total is 0
## 2023-11-21 14:19:56,534 -  DEBUG -  i is 4, total is 0
## 2023-11-21 14:19:56,534 -  DEBUG -  i is 5, total is 0
## 2023-11-21 14:19:56,534 -  DEBUG -  End of factorial(5)
\end{verbatim}

\begin{Shaded}
\begin{Highlighting}[]
\NormalTok{logging.debug(}\StringTok{\textquotesingle{}End of program\textquotesingle{}}\NormalTok{)}
\end{Highlighting}
\end{Shaded}

\begin{verbatim}
## 2023-11-21 14:19:56,537 -  DEBUG -  End of program
\end{verbatim}
\end{frame}

\begin{frame}[fragile]{Logging 4/7}
\protect\hypertarget{logging-47}{}
\begin{itemize}
\tightlist
\item
  Example with logging:
\end{itemize}

\begin{Shaded}
\begin{Highlighting}[]
\ImportTok{import}\NormalTok{ logging}
\NormalTok{logging.basicConfig(level}\OperatorTok{=}\NormalTok{logging.DEBUG, }\BuiltInTok{format}\OperatorTok{=}\StringTok{\textquotesingle{}}\SpecialCharTok{\%(asctime)s}\StringTok{ {-}  }\SpecialCharTok{\%(levelname)s}\StringTok{ {-}  }\SpecialCharTok{\%(message)s}\StringTok{\textquotesingle{}}\NormalTok{)}
\NormalTok{logging.debug(}\StringTok{\textquotesingle{}Start of program\textquotesingle{}}\NormalTok{)}
\end{Highlighting}
\end{Shaded}

\begin{verbatim}
## 2023-11-21 14:19:56,566 -  DEBUG -  Start of program
\end{verbatim}

\begin{Shaded}
\begin{Highlighting}[]
\KeywordTok{def}\NormalTok{ factorial(n):}
\NormalTok{    logging.debug(}\StringTok{\textquotesingle{}Start of factorial(}\SpecialCharTok{\%s}\StringTok{)\textquotesingle{}}  \OperatorTok{\%}\NormalTok{ (n))}
\NormalTok{    total }\OperatorTok{=} \DecValTok{1}
    \ControlFlowTok{for}\NormalTok{ i }\KeywordTok{in} \BuiltInTok{range}\NormalTok{(}\DecValTok{1}\NormalTok{,n }\OperatorTok{+} \DecValTok{1}\NormalTok{):}
\NormalTok{        total }\OperatorTok{*=}\NormalTok{ i}
\NormalTok{        logging.debug(}\StringTok{\textquotesingle{}i is \textquotesingle{}} \OperatorTok{+} \BuiltInTok{str}\NormalTok{(i) }\OperatorTok{+} \StringTok{\textquotesingle{}, total is \textquotesingle{}} \OperatorTok{+} \BuiltInTok{str}\NormalTok{(total))}
\NormalTok{    logging.debug(}\StringTok{\textquotesingle{}End of factorial(}\SpecialCharTok{\%s}\StringTok{)\textquotesingle{}}  \OperatorTok{\%}\NormalTok{ (n))}
    \ControlFlowTok{return}\NormalTok{ total}

\NormalTok{logging.debug(}\StringTok{\textquotesingle{}Call of the function\textquotesingle{}}\NormalTok{)}
\end{Highlighting}
\end{Shaded}

\begin{verbatim}
## 2023-11-21 14:19:56,570 -  DEBUG -  Call of the function
\end{verbatim}

\begin{Shaded}
\begin{Highlighting}[]
\BuiltInTok{print}\NormalTok{(factorial(}\DecValTok{5}\NormalTok{))}
\end{Highlighting}
\end{Shaded}

\begin{verbatim}
## 120
## 
## 2023-11-21 14:19:56,573 -  DEBUG -  Start of factorial(5)
## 2023-11-21 14:19:56,573 -  DEBUG -  i is 1, total is 1
## 2023-11-21 14:19:56,573 -  DEBUG -  i is 2, total is 2
## 2023-11-21 14:19:56,573 -  DEBUG -  i is 3, total is 6
## 2023-11-21 14:19:56,573 -  DEBUG -  i is 4, total is 24
## 2023-11-21 14:19:56,573 -  DEBUG -  i is 5, total is 120
## 2023-11-21 14:19:56,573 -  DEBUG -  End of factorial(5)
\end{verbatim}

\begin{Shaded}
\begin{Highlighting}[]
\NormalTok{logging.debug(}\StringTok{\textquotesingle{}End of program\textquotesingle{}}\NormalTok{)}
\end{Highlighting}
\end{Shaded}

\begin{verbatim}
## 2023-11-21 14:19:56,575 -  DEBUG -  End of program
\end{verbatim}
\end{frame}

\begin{frame}{Logging 5/7}
\protect\hypertarget{logging-57}{}
\begin{itemize}
\tightlist
\item
  The function \textbf{logging.basicConfig(level=logging.DEBUG,
  format=`\%(asctime)s - \%(levelname)s - \%(message)s')} can be set to
  look at several levels:

  \begin{itemize}
  \tightlist
  \item
    \textbf{DEBUG} logging.debug() The lowest level. Used for small
    details. Usually you care about these messages only when diagnosing
    problems.
  \item
    \textbf{INFO} logging.info() Used to record information on general
    events in your program or confirm that things are working at their
    point in the program.
  \item
    \textbf{WARNING} logging.warning() Used to indicate a potential
    problem that doesn't prevent the program from working but might do
    so in the future.
  \item
    \textbf{ERROR} logging.error() Used to record an error that caused
    the program to fail to do something.
  \item
    \textbf{CRITICAL} logging.critical() The highest level. Used to
    indicate a fatal error that has caused or is about to cause the
    program to stop running entirely.
  \end{itemize}
\end{itemize}
\end{frame}

\begin{frame}[fragile]{Logging 6/7}
\protect\hypertarget{logging-67}{}
\begin{itemize}[<+->]
\tightlist
\item
  Logging can be very useful for systematically inspecting your program
  execution.
\end{itemize}

\begin{itemize}[<+->]
\tightlist
\item
  It takes some time to adapt to it, but it's highly practical and
  customizable as needed.
\end{itemize}

\begin{itemize}[<+->]
\tightlist
\item
  Why is it better than print ?

  \begin{itemize}[<+->]
  \tightlist
  \item
    The print function can serve non-debugging purposes, making it
    challenging to differentiate between debugging print and essential
    print statements.
  \item
    Logging can be easily switched off with
    \textbf{logging.disable(logging.CRITICAL)}
  \end{itemize}
\end{itemize}

\begin{quote}
\begin{Shaded}
\begin{Highlighting}[]
\ImportTok{import}\NormalTok{ logging}
\NormalTok{logging.disable(logging.CRITICAL)}
\NormalTok{logging.basicConfig(level}\OperatorTok{=}\NormalTok{logging.DEBUG, }\BuiltInTok{format}\OperatorTok{=}\StringTok{\textquotesingle{}}\SpecialCharTok{\%(asctime)s}\StringTok{ {-} }\SpecialCharTok{\%(levelname)s}\StringTok{ {-} }\SpecialCharTok{\%(message)s}\StringTok{\textquotesingle{}}\NormalTok{)}
\NormalTok{logging.debug(}\StringTok{\textquotesingle{}Start of program\textquotesingle{}}\NormalTok{)}

\KeywordTok{def}\NormalTok{ factorial(n):}
\NormalTok{    logging.debug(}\StringTok{\textquotesingle{}Start of factorial(}\SpecialCharTok{\%s}\StringTok{)\textquotesingle{}}  \OperatorTok{\%}\NormalTok{ (n))}
\NormalTok{    total }\OperatorTok{=} \DecValTok{1}
    \ControlFlowTok{for}\NormalTok{ i }\KeywordTok{in} \BuiltInTok{range}\NormalTok{(}\DecValTok{1}\NormalTok{,n }\OperatorTok{+} \DecValTok{1}\NormalTok{):}
\NormalTok{        total }\OperatorTok{*=}\NormalTok{ i}
\NormalTok{        logging.debug(}\StringTok{\textquotesingle{}i is \textquotesingle{}} \OperatorTok{+} \BuiltInTok{str}\NormalTok{(i) }\OperatorTok{+} \StringTok{\textquotesingle{}, total is \textquotesingle{}} \OperatorTok{+} \BuiltInTok{str}\NormalTok{(total))}
\NormalTok{    logging.debug(}\StringTok{\textquotesingle{}End of factorial(}\SpecialCharTok{\%s}\StringTok{)\textquotesingle{}}  \OperatorTok{\%}\NormalTok{ (n))}
    \ControlFlowTok{return}\NormalTok{ total}

\NormalTok{logging.debug(}\StringTok{\textquotesingle{}Call of the function\textquotesingle{}}\NormalTok{)}
\BuiltInTok{print}\NormalTok{(factorial(}\DecValTok{5}\NormalTok{))}
\end{Highlighting}
\end{Shaded}

\begin{verbatim}
## 120
\end{verbatim}

\begin{Shaded}
\begin{Highlighting}[]

\NormalTok{logging.debug(}\StringTok{\textquotesingle{}End of program\textquotesingle{}}\NormalTok{)}
\end{Highlighting}
\end{Shaded}
\end{quote}
\end{frame}

\begin{frame}{Logging 7/7}
\protect\hypertarget{logging-77}{}
\begin{itemize}
\item
  Logging is highly useful for debugging without interrupting the
  execution.
\item
  For example, when your program doesn't crash but doesn't produce the
  desired output.
\item
  Debugging can even be done post-mortem (after execution).
\item
  It is especially efficient when you cannot access the terminal (i.e.,
  when your program involves visual stimuli).
\end{itemize}
\end{frame}

\begin{frame}[fragile]{Python debugger: PDB 1/4}
\protect\hypertarget{python-debugger-pdb-14}{}
\begin{itemize}
\item
  The ultimate debugging tool is the \textbf{pdb} module:
  \url{https://docs.python.org/3/library/pdb.html}
\item
  You just need to \textbf{import pdb} and call
  \textbf{pdb.set\_trace()}

\begin{Shaded}
\begin{Highlighting}[]
\ImportTok{import}\NormalTok{ pdb}

\KeywordTok{def}\NormalTok{ addition(a, b):}
\NormalTok{    answer }\OperatorTok{=}\NormalTok{ a }\OperatorTok{*}\NormalTok{ b}
    \ControlFlowTok{return}\NormalTok{ answer}

\NormalTok{pdb.set\_trace()}
\NormalTok{x }\OperatorTok{=} \BuiltInTok{input}\NormalTok{(}\StringTok{"Enter first number : "}\NormalTok{)}
\NormalTok{y }\OperatorTok{=} \BuiltInTok{input}\NormalTok{(}\StringTok{"Enter second number : "}\NormalTok{)}
\BuiltInTok{sum} \OperatorTok{=}\NormalTok{ addition(x, y)}
\BuiltInTok{print}\NormalTok{(}\BuiltInTok{sum}\NormalTok{)}
\end{Highlighting}
\end{Shaded}
\item
  When executing your program with the module \textbf{pdb} you have a
  screen like that
\end{itemize}

\includegraphics[width=0.5\textwidth,height=\textheight]{~/Documents/Cours python/PCBS/slides/intro-to-programming/2023/8th class/screenPDB.png}
\end{frame}

\begin{frame}{Python debugger: PDB 2/4}
\protect\hypertarget{python-debugger-pdb-24}{}
\begin{itemize}
\item
  It is a command line tool that go sequentially at every step of the
  program
\item
  There are a certain number of command to know:

  \begin{itemize}
  \tightlist
  \item
    \textbf{help} To display all commands
  \item
    \textbf{where} Display the stack trace and line number of the
    current line
  \item
    \textbf{next} Execute the current line and move to the next line
    ignoring function calls
  \item
    \textbf{step} Step into functions called at the current line
  \item
    \textbf{whatis} Check the type of variable
  \end{itemize}
\end{itemize}

\includegraphics{~/Documents/Cours python/PCBS/slides/intro-to-programming/2023/8th class/screenPDB_step.png}
\end{frame}

\begin{frame}{Python debugger: PDB 3/4}
\protect\hypertarget{python-debugger-pdb-34}{}
\begin{itemize}
\tightlist
\item
  You can use as well:

  \begin{itemize}
  \tightlist
  \item
    \textbf{args} To get all arguments of a function
  \item
    \textbf{p} To get the value at a time t of a variable
  \end{itemize}
\item
  You can navigate in pdb prompt using:

  \begin{itemize}
  \tightlist
  \item
    \textbf{c} continue execution
  \item
    \textbf{q} quit the debugger/execution
  \item
    \textbf{n} step to next line within the same function
  \item
    \textbf{s} step to next line in this function or a called function
  \item
    \textbf{u} (up)
  \item
    \textbf{d} (down)
  \end{itemize}
\end{itemize}
\end{frame}

\begin{frame}{Python debugger: PDB 4/4}
\protect\hypertarget{python-debugger-pdb-44}{}
\begin{itemize}
\item
  You can also set a breakpoint at a specific point in the script
\item
  To do that you need to write on the terminal: \textbf{break filename:
  lineno, condition}
\end{itemize}

\includegraphics[width=0.5\textwidth,height=\textheight]{~/Documents/Cours python/PCBS/slides/intro-to-programming/2023/8th class/screenPDB_break.png}

\begin{itemize}
\tightlist
\item
  You can then use \textbf{c} to run the program until your breakpoint
\end{itemize}

\includegraphics[width=0.5\textwidth,height=\textheight]{~/Documents/Cours python/PCBS/slides/intro-to-programming/2023/8th class/screenPDB_break2.png}
\end{frame}

\end{document}
