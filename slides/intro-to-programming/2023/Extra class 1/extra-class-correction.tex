% Options for packages loaded elsewhere
\PassOptionsToPackage{unicode}{hyperref}
\PassOptionsToPackage{hyphens}{url}
%
\documentclass[
  8pt,
  ignorenonframetext,
]{beamer}
\usepackage{pgfpages}
\setbeamertemplate{caption}[numbered]
\setbeamertemplate{caption label separator}{: }
\setbeamercolor{caption name}{fg=normal text.fg}
\beamertemplatenavigationsymbolsempty
% Prevent slide breaks in the middle of a paragraph
\widowpenalties 1 10000
\raggedbottom
\setbeamertemplate{part page}{
  \centering
  \begin{beamercolorbox}[sep=16pt,center]{part title}
    \usebeamerfont{part title}\insertpart\par
  \end{beamercolorbox}
}
\setbeamertemplate{section page}{
  \centering
  \begin{beamercolorbox}[sep=12pt,center]{part title}
    \usebeamerfont{section title}\insertsection\par
  \end{beamercolorbox}
}
\setbeamertemplate{subsection page}{
  \centering
  \begin{beamercolorbox}[sep=8pt,center]{part title}
    \usebeamerfont{subsection title}\insertsubsection\par
  \end{beamercolorbox}
}
\AtBeginPart{
  \frame{\partpage}
}
\AtBeginSection{
  \ifbibliography
  \else
    \frame{\sectionpage}
  \fi
}
\AtBeginSubsection{
  \frame{\subsectionpage}
}
\usepackage{amsmath,amssymb}
\usepackage{iftex}
\ifPDFTeX
  \usepackage[T1]{fontenc}
  \usepackage[utf8]{inputenc}
  \usepackage{textcomp} % provide euro and other symbols
\else % if luatex or xetex
  \usepackage{unicode-math} % this also loads fontspec
  \defaultfontfeatures{Scale=MatchLowercase}
  \defaultfontfeatures[\rmfamily]{Ligatures=TeX,Scale=1}
\fi
\usepackage{lmodern}
\usetheme[]{CambridgeUS}
\ifPDFTeX\else
  % xetex/luatex font selection
\fi
% Use upquote if available, for straight quotes in verbatim environments
\IfFileExists{upquote.sty}{\usepackage{upquote}}{}
\IfFileExists{microtype.sty}{% use microtype if available
  \usepackage[]{microtype}
  \UseMicrotypeSet[protrusion]{basicmath} % disable protrusion for tt fonts
}{}
\makeatletter
\@ifundefined{KOMAClassName}{% if non-KOMA class
  \IfFileExists{parskip.sty}{%
    \usepackage{parskip}
  }{% else
    \setlength{\parindent}{0pt}
    \setlength{\parskip}{6pt plus 2pt minus 1pt}}
}{% if KOMA class
  \KOMAoptions{parskip=half}}
\makeatother
\usepackage{xcolor}
\newif\ifbibliography
\usepackage{color}
\usepackage{fancyvrb}
\newcommand{\VerbBar}{|}
\newcommand{\VERB}{\Verb[commandchars=\\\{\}]}
\DefineVerbatimEnvironment{Highlighting}{Verbatim}{commandchars=\\\{\}}
% Add ',fontsize=\small' for more characters per line
\usepackage{framed}
\definecolor{shadecolor}{RGB}{248,248,248}
\newenvironment{Shaded}{\begin{snugshade}}{\end{snugshade}}
\newcommand{\AlertTok}[1]{\textcolor[rgb]{0.94,0.16,0.16}{#1}}
\newcommand{\AnnotationTok}[1]{\textcolor[rgb]{0.56,0.35,0.01}{\textbf{\textit{#1}}}}
\newcommand{\AttributeTok}[1]{\textcolor[rgb]{0.13,0.29,0.53}{#1}}
\newcommand{\BaseNTok}[1]{\textcolor[rgb]{0.00,0.00,0.81}{#1}}
\newcommand{\BuiltInTok}[1]{#1}
\newcommand{\CharTok}[1]{\textcolor[rgb]{0.31,0.60,0.02}{#1}}
\newcommand{\CommentTok}[1]{\textcolor[rgb]{0.56,0.35,0.01}{\textit{#1}}}
\newcommand{\CommentVarTok}[1]{\textcolor[rgb]{0.56,0.35,0.01}{\textbf{\textit{#1}}}}
\newcommand{\ConstantTok}[1]{\textcolor[rgb]{0.56,0.35,0.01}{#1}}
\newcommand{\ControlFlowTok}[1]{\textcolor[rgb]{0.13,0.29,0.53}{\textbf{#1}}}
\newcommand{\DataTypeTok}[1]{\textcolor[rgb]{0.13,0.29,0.53}{#1}}
\newcommand{\DecValTok}[1]{\textcolor[rgb]{0.00,0.00,0.81}{#1}}
\newcommand{\DocumentationTok}[1]{\textcolor[rgb]{0.56,0.35,0.01}{\textbf{\textit{#1}}}}
\newcommand{\ErrorTok}[1]{\textcolor[rgb]{0.64,0.00,0.00}{\textbf{#1}}}
\newcommand{\ExtensionTok}[1]{#1}
\newcommand{\FloatTok}[1]{\textcolor[rgb]{0.00,0.00,0.81}{#1}}
\newcommand{\FunctionTok}[1]{\textcolor[rgb]{0.13,0.29,0.53}{\textbf{#1}}}
\newcommand{\ImportTok}[1]{#1}
\newcommand{\InformationTok}[1]{\textcolor[rgb]{0.56,0.35,0.01}{\textbf{\textit{#1}}}}
\newcommand{\KeywordTok}[1]{\textcolor[rgb]{0.13,0.29,0.53}{\textbf{#1}}}
\newcommand{\NormalTok}[1]{#1}
\newcommand{\OperatorTok}[1]{\textcolor[rgb]{0.81,0.36,0.00}{\textbf{#1}}}
\newcommand{\OtherTok}[1]{\textcolor[rgb]{0.56,0.35,0.01}{#1}}
\newcommand{\PreprocessorTok}[1]{\textcolor[rgb]{0.56,0.35,0.01}{\textit{#1}}}
\newcommand{\RegionMarkerTok}[1]{#1}
\newcommand{\SpecialCharTok}[1]{\textcolor[rgb]{0.81,0.36,0.00}{\textbf{#1}}}
\newcommand{\SpecialStringTok}[1]{\textcolor[rgb]{0.31,0.60,0.02}{#1}}
\newcommand{\StringTok}[1]{\textcolor[rgb]{0.31,0.60,0.02}{#1}}
\newcommand{\VariableTok}[1]{\textcolor[rgb]{0.00,0.00,0.00}{#1}}
\newcommand{\VerbatimStringTok}[1]{\textcolor[rgb]{0.31,0.60,0.02}{#1}}
\newcommand{\WarningTok}[1]{\textcolor[rgb]{0.56,0.35,0.01}{\textbf{\textit{#1}}}}
\setlength{\emergencystretch}{3em} % prevent overfull lines
\providecommand{\tightlist}{%
  \setlength{\itemsep}{0pt}\setlength{\parskip}{0pt}}
\setcounter{secnumdepth}{-\maxdimen} % remove section numbering
\let\verbatim\undefined
\let\verbatimend\undefined
\usepackage{listings}
\lstnewenvironment{verbatim}{\lstset{breaklines=true,basicstyle=\ttfamily\footnotesize}}{}
\ifLuaTeX
  \usepackage{selnolig}  % disable illegal ligatures
\fi
\IfFileExists{bookmark.sty}{\usepackage{bookmark}}{\usepackage{hyperref}}
\IfFileExists{xurl.sty}{\usepackage{xurl}}{} % add URL line breaks if available
\urlstyle{same}
\hypersetup{
  pdftitle={Extra class correction},
  hidelinks,
  pdfcreator={LaTeX via pandoc}}

\title{Extra class correction}
\author{Henri Vandendriessche\\
\href{mailto:henri.vandendriessche@ens.fr}{\nolinkurl{henri.vandendriessche@ens.fr}}}
\date{2023-12-19}

\begin{document}
\frame{\titlepage}

\begin{frame}[fragile]{MCQ test correction}
\protect\hypertarget{mcq-test-correction}{}
Q1. What does the following Python code do?

\begin{Shaded}
\begin{Highlighting}[]
\BuiltInTok{print}\NormalTok{(}\StringTok{"Bonjour"}\NormalTok{)}
\end{Highlighting}
\end{Shaded}
\end{frame}

\begin{frame}[fragile]{MCQ test correction}
\protect\hypertarget{mcq-test-correction-1}{}
Q1. What does the following Python code do?

\begin{Shaded}
\begin{Highlighting}[]
\BuiltInTok{print}\NormalTok{(}\StringTok{"Bonjour"}\NormalTok{)}
\end{Highlighting}
\end{Shaded}

\begin{verbatim}
## Bonjour
\end{verbatim}
\end{frame}

\begin{frame}[fragile]{MCQ test correction}
\protect\hypertarget{mcq-test-correction-2}{}
Q2. What does the following Python code do?

\begin{Shaded}
\begin{Highlighting}[]
\BuiltInTok{print}\NormalTok{(}\DecValTok{3} \OperatorTok{+} \DecValTok{2}\NormalTok{)}
\end{Highlighting}
\end{Shaded}
\end{frame}

\begin{frame}[fragile]{MCQ test correction}
\protect\hypertarget{mcq-test-correction-3}{}
Q2. What does the following Python code do?

\begin{Shaded}
\begin{Highlighting}[]
\BuiltInTok{print}\NormalTok{(}\DecValTok{3} \OperatorTok{+} \DecValTok{2}\NormalTok{)}
\end{Highlighting}
\end{Shaded}

\begin{verbatim}
## 5
\end{verbatim}
\end{frame}

\begin{frame}[fragile]{MCQ test correction}
\protect\hypertarget{mcq-test-correction-4}{}
Q3. What does the following Python code do?

\begin{Shaded}
\begin{Highlighting}[]
\BuiltInTok{print}\NormalTok{(}\StringTok{"3 + 2"}\NormalTok{)}
\end{Highlighting}
\end{Shaded}
\end{frame}

\begin{frame}[fragile]{MCQ test correction}
\protect\hypertarget{mcq-test-correction-5}{}
Q3. What does the following Python code do?

\begin{Shaded}
\begin{Highlighting}[]
\BuiltInTok{print}\NormalTok{(}\StringTok{"3 + 2"}\NormalTok{)}
\end{Highlighting}
\end{Shaded}

\begin{verbatim}
## 3 + 2
\end{verbatim}
\end{frame}

\begin{frame}[fragile]{MCQ test correction}
\protect\hypertarget{mcq-test-correction-6}{}
Q4: What do the following Python code do?

\begin{Shaded}
\begin{Highlighting}[]
\BuiltInTok{print}\NormalTok{(x)}
\end{Highlighting}
\end{Shaded}
\end{frame}

\begin{frame}[fragile]{MCQ test correction}
\protect\hypertarget{mcq-test-correction-7}{}
Q4: What do the following Python code do?

\begin{Shaded}
\begin{Highlighting}[]
\BuiltInTok{print}\NormalTok{(x)}
\end{Highlighting}
\end{Shaded}

\begin{verbatim}
## name 'x' is not defined
\end{verbatim}
\end{frame}

\begin{frame}[fragile]{MCQ test correction}
\protect\hypertarget{mcq-test-correction-8}{}
Q5. what does the following Python code do?

\begin{Shaded}
\begin{Highlighting}[]
\NormalTok{x }\OperatorTok{=} \DecValTok{5}
\BuiltInTok{print}\NormalTok{(x }\OperatorTok{{-}} \DecValTok{3}\NormalTok{)}
\end{Highlighting}
\end{Shaded}
\end{frame}

\begin{frame}[fragile]{MCQ test correction}
\protect\hypertarget{mcq-test-correction-9}{}
Q5. what does the following Python code do?

\begin{Shaded}
\begin{Highlighting}[]
\NormalTok{x }\OperatorTok{=} \DecValTok{5}
\BuiltInTok{print}\NormalTok{(x }\OperatorTok{{-}} \DecValTok{3}\NormalTok{)}
\end{Highlighting}
\end{Shaded}

\begin{verbatim}
## 2
\end{verbatim}
\end{frame}

\begin{frame}[fragile]{MCQ test correction}
\protect\hypertarget{mcq-test-correction-10}{}
Q6. What does the following code print?

\begin{Shaded}
\begin{Highlighting}[]
\NormalTok{x }\OperatorTok{=} \DecValTok{0}
\NormalTok{x }\OperatorTok{=} \DecValTok{1}
\BuiltInTok{print}\NormalTok{(x)}
\end{Highlighting}
\end{Shaded}
\end{frame}

\begin{frame}[fragile]{MCQ test correction}
\protect\hypertarget{mcq-test-correction-11}{}
Q6. What does the following code print?

\begin{Shaded}
\begin{Highlighting}[]
\NormalTok{x }\OperatorTok{=} \DecValTok{0}
\NormalTok{x }\OperatorTok{=} \DecValTok{1}
\BuiltInTok{print}\NormalTok{(x)}
\end{Highlighting}
\end{Shaded}

\begin{verbatim}
## 1
\end{verbatim}
\end{frame}

\begin{frame}[fragile]{MCQ test correction}
\protect\hypertarget{mcq-test-correction-12}{}
Q7. What does the following code print ?

\begin{Shaded}
\begin{Highlighting}[]
\NormalTok{x }\OperatorTok{=} \DecValTok{2}
\NormalTok{x }\OperatorTok{=}\NormalTok{ x }\OperatorTok{{-}} \DecValTok{1}
\BuiltInTok{print}\NormalTok{(x)}
\end{Highlighting}
\end{Shaded}
\end{frame}

\begin{frame}[fragile]{MCQ test correction}
\protect\hypertarget{mcq-test-correction-13}{}
Q7. What does the following code print ?

\begin{Shaded}
\begin{Highlighting}[]
\NormalTok{x }\OperatorTok{=} \DecValTok{2}
\NormalTok{x }\OperatorTok{=}\NormalTok{ x }\OperatorTok{{-}} \DecValTok{1}
\BuiltInTok{print}\NormalTok{(x)}
\end{Highlighting}
\end{Shaded}

\begin{verbatim}
## 1
\end{verbatim}
\end{frame}

\begin{frame}[fragile]{MCQ test correction}
\protect\hypertarget{mcq-test-correction-14}{}
Q8. What does the following Python code print?

\begin{Shaded}
\begin{Highlighting}[]
\BuiltInTok{print}\NormalTok{(}\DecValTok{1} \OperatorTok{==} \DecValTok{2}\NormalTok{)}
\end{Highlighting}
\end{Shaded}
\end{frame}

\begin{frame}[fragile]{MCQ test correction}
\protect\hypertarget{mcq-test-correction-15}{}
Q8. What does the following Python code print?

\begin{Shaded}
\begin{Highlighting}[]
\BuiltInTok{print}\NormalTok{(}\DecValTok{1} \OperatorTok{==} \DecValTok{2}\NormalTok{)}
\end{Highlighting}
\end{Shaded}

\begin{verbatim}
## False
\end{verbatim}
\end{frame}

\begin{frame}[fragile]{MCQ test correction}
\protect\hypertarget{mcq-test-correction-16}{}
Q9. What does the following code print?

\begin{Shaded}
\begin{Highlighting}[]
\NormalTok{x }\OperatorTok{=} \DecValTok{3} \OperatorTok{==} \DecValTok{3}
\BuiltInTok{print}\NormalTok{(x)}
\end{Highlighting}
\end{Shaded}
\end{frame}

\begin{frame}[fragile]{MCQ test correction}
\protect\hypertarget{mcq-test-correction-17}{}
Q9. What does the following code print?

\begin{Shaded}
\begin{Highlighting}[]
\NormalTok{x }\OperatorTok{=} \DecValTok{3} \OperatorTok{==} \DecValTok{3}
\BuiltInTok{print}\NormalTok{(x)}
\end{Highlighting}
\end{Shaded}

\begin{verbatim}
## True
\end{verbatim}
\end{frame}

\begin{frame}[fragile]{MCQ test correction}
\protect\hypertarget{mcq-test-correction-18}{}
Q10. What does the following code print?

\begin{Shaded}
\begin{Highlighting}[]
\ControlFlowTok{if} \DecValTok{0} \OperatorTok{==} \DecValTok{1}\NormalTok{:}
   \BuiltInTok{print}\NormalTok{(}\StringTok{"a"}\NormalTok{)}

\BuiltInTok{print}\NormalTok{(}\StringTok{"b"}\NormalTok{) }
\end{Highlighting}
\end{Shaded}
\end{frame}

\begin{frame}[fragile]{MCQ test correction}
\protect\hypertarget{mcq-test-correction-19}{}
Q10. What does the following code print?

\begin{Shaded}
\begin{Highlighting}[]
\ControlFlowTok{if} \DecValTok{0} \OperatorTok{==} \DecValTok{1}\NormalTok{:}
   \BuiltInTok{print}\NormalTok{(}\StringTok{"a"}\NormalTok{)}

\BuiltInTok{print}\NormalTok{(}\StringTok{"b"}\NormalTok{) }
\end{Highlighting}
\end{Shaded}

\begin{verbatim}
## b
\end{verbatim}
\end{frame}

\begin{frame}[fragile]{MCQ test correction}
\protect\hypertarget{mcq-test-correction-20}{}
Q11. What does the following code print?

\begin{Shaded}
\begin{Highlighting}[]
\ControlFlowTok{if} \DecValTok{0} \OperatorTok{==} \DecValTok{1}\NormalTok{:}
    \BuiltInTok{print}\NormalTok{(}\StringTok{"a"}\NormalTok{)}
\ControlFlowTok{else}\NormalTok{:}
    \BuiltInTok{print}\NormalTok{(}\StringTok{"b"}\NormalTok{)}
\end{Highlighting}
\end{Shaded}
\end{frame}

\begin{frame}[fragile]{MCQ test correction}
\protect\hypertarget{mcq-test-correction-21}{}
Q11. What does the following code print?

\begin{Shaded}
\begin{Highlighting}[]
\ControlFlowTok{if} \DecValTok{0} \OperatorTok{==} \DecValTok{1}\NormalTok{:}
    \BuiltInTok{print}\NormalTok{(}\StringTok{"a"}\NormalTok{)}
\ControlFlowTok{else}\NormalTok{:}
    \BuiltInTok{print}\NormalTok{(}\StringTok{"b"}\NormalTok{)}
\end{Highlighting}
\end{Shaded}

\begin{verbatim}
## b
\end{verbatim}
\end{frame}

\begin{frame}[fragile]{MCQ test correction}
\protect\hypertarget{mcq-test-correction-22}{}
Q12. What does the following code print?

\begin{Shaded}
\begin{Highlighting}[]
\ControlFlowTok{if} \DecValTok{0} \OperatorTok{==} \DecValTok{1}\NormalTok{:}
    \BuiltInTok{print}\NormalTok{(}\StringTok{"a"}\NormalTok{)}
    \BuiltInTok{print}\NormalTok{(}\StringTok{"b"}\NormalTok{)}
\ControlFlowTok{else}\NormalTok{:}
    \BuiltInTok{print}\NormalTok{(}\StringTok{"c"}\NormalTok{)}
    \BuiltInTok{print}\NormalTok{(}\StringTok{"d"}\NormalTok{)}

\BuiltInTok{print}\NormalTok{(}\StringTok{"e"}\NormalTok{)}
\end{Highlighting}
\end{Shaded}
\end{frame}

\begin{frame}[fragile]{MCQ test correction}
\protect\hypertarget{mcq-test-correction-23}{}
Q12. What does the following code print?

\begin{Shaded}
\begin{Highlighting}[]
\ControlFlowTok{if} \DecValTok{0} \OperatorTok{==} \DecValTok{1}\NormalTok{:}
    \BuiltInTok{print}\NormalTok{(}\StringTok{"a"}\NormalTok{)}
    \BuiltInTok{print}\NormalTok{(}\StringTok{"b"}\NormalTok{)}
\ControlFlowTok{else}\NormalTok{:}
    \BuiltInTok{print}\NormalTok{(}\StringTok{"c"}\NormalTok{)}
    \BuiltInTok{print}\NormalTok{(}\StringTok{"d"}\NormalTok{)}
\end{Highlighting}
\end{Shaded}

\begin{verbatim}
## c
## d
\end{verbatim}

\begin{Shaded}
\begin{Highlighting}[]
\BuiltInTok{print}\NormalTok{(}\StringTok{"e"}\NormalTok{)}
\end{Highlighting}
\end{Shaded}

\begin{verbatim}
## e
\end{verbatim}
\end{frame}

\begin{frame}[fragile]{MCQ test correction}
\protect\hypertarget{mcq-test-correction-24}{}
Q13. What does the following code print?

\begin{Shaded}
\begin{Highlighting}[]
\ControlFlowTok{if} \DecValTok{0} \OperatorTok{==} \DecValTok{1}\NormalTok{:}
    \BuiltInTok{print}\NormalTok{(}\StringTok{"a"}\NormalTok{)}
\ControlFlowTok{else}\NormalTok{:}
    \ControlFlowTok{if} \DecValTok{1} \OperatorTok{==} \DecValTok{1}\NormalTok{:}
        \BuiltInTok{print}\NormalTok{(}\StringTok{"b"}\NormalTok{)}
    \ControlFlowTok{else}\NormalTok{:}
        \BuiltInTok{print}\NormalTok{(}\StringTok{"c"}\NormalTok{)}
    \BuiltInTok{print}\NormalTok{(}\StringTok{"d"}\NormalTok{)}
\end{Highlighting}
\end{Shaded}
\end{frame}

\begin{frame}[fragile]{MCQ test correction}
\protect\hypertarget{mcq-test-correction-25}{}
Q13. What does the following code print?

\begin{Shaded}
\begin{Highlighting}[]
\ControlFlowTok{if} \DecValTok{0} \OperatorTok{==} \DecValTok{1}\NormalTok{:}
    \BuiltInTok{print}\NormalTok{(}\StringTok{"a"}\NormalTok{)}
\ControlFlowTok{else}\NormalTok{:}
    \ControlFlowTok{if} \DecValTok{1} \OperatorTok{==} \DecValTok{1}\NormalTok{:}
        \BuiltInTok{print}\NormalTok{(}\StringTok{"b"}\NormalTok{)}
    \ControlFlowTok{else}\NormalTok{:}
        \BuiltInTok{print}\NormalTok{(}\StringTok{"c"}\NormalTok{)}
    \BuiltInTok{print}\NormalTok{(}\StringTok{"d"}\NormalTok{)}
\end{Highlighting}
\end{Shaded}

\begin{verbatim}
## b
## d
\end{verbatim}
\end{frame}

\begin{frame}[fragile]{MCQ test correction}
\protect\hypertarget{mcq-test-correction-26}{}
Q14. How many lines does the following code print?

\begin{Shaded}
\begin{Highlighting}[]
\NormalTok{n }\OperatorTok{=} \DecValTok{0}
\ControlFlowTok{while}\NormalTok{ n }\OperatorTok{\textgreater{}} \DecValTok{1}\NormalTok{:}
    \BuiltInTok{print}\NormalTok{(}\StringTok{"ok"}\NormalTok{)}
\end{Highlighting}
\end{Shaded}
\end{frame}

\begin{frame}[fragile]{MCQ test correction}
\protect\hypertarget{mcq-test-correction-27}{}
Q14. How many lines does the following code print?

\begin{Shaded}
\begin{Highlighting}[]
\NormalTok{n }\OperatorTok{=} \DecValTok{0}
\ControlFlowTok{while}\NormalTok{ n }\OperatorTok{\textgreater{}} \DecValTok{1}\NormalTok{:}
    \BuiltInTok{print}\NormalTok{(}\StringTok{"ok"}\NormalTok{)}
\end{Highlighting}
\end{Shaded}
\end{frame}

\begin{frame}[fragile]{MCQ test correction}
\protect\hypertarget{mcq-test-correction-28}{}
Q15. How many lines does the following code print?

\begin{Shaded}
\begin{Highlighting}[]
\NormalTok{n }\OperatorTok{=} \DecValTok{3}
\ControlFlowTok{while}\NormalTok{ n }\OperatorTok{\textgreater{}} \DecValTok{1}\NormalTok{:}
\NormalTok{    n }\OperatorTok{=}\NormalTok{ n }\OperatorTok{{-}} \DecValTok{1}
    \BuiltInTok{print}\NormalTok{(}\StringTok{"ok"}\NormalTok{)}
\end{Highlighting}
\end{Shaded}
\end{frame}

\begin{frame}[fragile]{MCQ test correction}
\protect\hypertarget{mcq-test-correction-29}{}
Q15. How many lines does the following code print?

\begin{Shaded}
\begin{Highlighting}[]
\NormalTok{n }\OperatorTok{=} \DecValTok{3}
\ControlFlowTok{while}\NormalTok{ n }\OperatorTok{\textgreater{}} \DecValTok{1}\NormalTok{:}
\NormalTok{    n }\OperatorTok{=}\NormalTok{ n }\OperatorTok{{-}} \DecValTok{1}
    \BuiltInTok{print}\NormalTok{(}\StringTok{"ok"}\NormalTok{)}
\end{Highlighting}
\end{Shaded}

\begin{verbatim}
## ok
## ok
\end{verbatim}
\end{frame}

\begin{frame}[fragile]{MCQ test correction}
\protect\hypertarget{mcq-test-correction-30}{}
Q16. What does the following code print?

\begin{Shaded}
\begin{Highlighting}[]
\NormalTok{x }\OperatorTok{=} \DecValTok{1}
\NormalTok{y }\OperatorTok{=} \OperatorTok{{-}}\DecValTok{1}
\ControlFlowTok{while}\NormalTok{ x }\OperatorTok{\textless{}} \DecValTok{5}\NormalTok{:}
\NormalTok{    y }\OperatorTok{=}\NormalTok{ y }\OperatorTok{{-}} \DecValTok{1}
\NormalTok{    x }\OperatorTok{=}\NormalTok{ x }\OperatorTok{*} \DecValTok{2}

\BuiltInTok{print}\NormalTok{(x, y)}
\end{Highlighting}
\end{Shaded}
\end{frame}

\begin{frame}[fragile]{MCQ test correction}
\protect\hypertarget{mcq-test-correction-31}{}
Q16. What does the following code print?

\begin{Shaded}
\begin{Highlighting}[]
\NormalTok{x }\OperatorTok{=} \DecValTok{1}
\NormalTok{y }\OperatorTok{=} \OperatorTok{{-}}\DecValTok{1}
\ControlFlowTok{while}\NormalTok{ x }\OperatorTok{\textless{}} \DecValTok{5}\NormalTok{:}
\NormalTok{    y }\OperatorTok{=}\NormalTok{ y }\OperatorTok{{-}} \DecValTok{1}
\NormalTok{    x }\OperatorTok{=}\NormalTok{ x }\OperatorTok{*} \DecValTok{2}

\BuiltInTok{print}\NormalTok{(x, y)}
\end{Highlighting}
\end{Shaded}

\begin{verbatim}
## 8 -4
\end{verbatim}
\end{frame}

\begin{frame}[fragile]{MCQ test correction}
\protect\hypertarget{mcq-test-correction-32}{}
Q17. What does the following code print?

\begin{Shaded}
\begin{Highlighting}[]
\NormalTok{n }\OperatorTok{=} \DecValTok{0}
\ControlFlowTok{while}\NormalTok{ n }\OperatorTok{\textless{}} \DecValTok{3}\NormalTok{:}
    \ControlFlowTok{if}\NormalTok{ n }\OperatorTok{\textless{}} \DecValTok{2}\NormalTok{:}
        \BuiltInTok{print}\NormalTok{(}\StringTok{"less"}\NormalTok{)}
    \ControlFlowTok{else}\NormalTok{:}
        \BuiltInTok{print}\NormalTok{(}\StringTok{"more"}\NormalTok{)}
\NormalTok{    n }\OperatorTok{=}\NormalTok{ n }\OperatorTok{+} \DecValTok{1}
\end{Highlighting}
\end{Shaded}
\end{frame}

\begin{frame}[fragile]{MCQ test correction}
\protect\hypertarget{mcq-test-correction-33}{}
Q17. What does the following code print?

\begin{Shaded}
\begin{Highlighting}[]
\NormalTok{n }\OperatorTok{=} \DecValTok{0}
\ControlFlowTok{while}\NormalTok{ n }\OperatorTok{\textless{}} \DecValTok{3}\NormalTok{:}
    \ControlFlowTok{if}\NormalTok{ n }\OperatorTok{\textless{}} \DecValTok{2}\NormalTok{:}
        \BuiltInTok{print}\NormalTok{(}\StringTok{"less"}\NormalTok{)}
    \ControlFlowTok{else}\NormalTok{:}
        \BuiltInTok{print}\NormalTok{(}\StringTok{"more"}\NormalTok{)}
\NormalTok{    n }\OperatorTok{=}\NormalTok{ n }\OperatorTok{+} \DecValTok{1}
\end{Highlighting}
\end{Shaded}

\begin{verbatim}
## less
## less
## more
\end{verbatim}
\end{frame}

\begin{frame}[fragile]{MCQ test correction}
\protect\hypertarget{mcq-test-correction-34}{}
Q18. What does the following code print?

\begin{Shaded}
\begin{Highlighting}[]
\ControlFlowTok{for}\NormalTok{ x }\KeywordTok{in}\NormalTok{ [}\DecValTok{3}\NormalTok{, }\DecValTok{1}\NormalTok{]:}
    \ControlFlowTok{for}\NormalTok{ y }\KeywordTok{in}\NormalTok{ [}\DecValTok{2}\NormalTok{, }\DecValTok{4}\NormalTok{]:}
        \BuiltInTok{print}\NormalTok{(x, y)}
\end{Highlighting}
\end{Shaded}
\end{frame}

\begin{frame}[fragile]{MCQ test correction}
\protect\hypertarget{mcq-test-correction-35}{}
Q18. What does the following code print?

\begin{Shaded}
\begin{Highlighting}[]
\ControlFlowTok{for}\NormalTok{ x }\KeywordTok{in}\NormalTok{ [}\DecValTok{3}\NormalTok{, }\DecValTok{1}\NormalTok{]:}
    \ControlFlowTok{for}\NormalTok{ y }\KeywordTok{in}\NormalTok{ [}\DecValTok{2}\NormalTok{, }\DecValTok{4}\NormalTok{]:}
        \BuiltInTok{print}\NormalTok{(x, y)}
\end{Highlighting}
\end{Shaded}

\begin{verbatim}
## 3 2
## 3 4
## 1 2
## 1 4
\end{verbatim}
\end{frame}

\begin{frame}[fragile]{MCQ test correction}
\protect\hypertarget{mcq-test-correction-36}{}
Q19. What does the following code print?

\begin{Shaded}
\begin{Highlighting}[]
\KeywordTok{def}\NormalTok{ print\_one():}
    \BuiltInTok{print}\NormalTok{(}\DecValTok{1}\NormalTok{)}
\end{Highlighting}
\end{Shaded}
\end{frame}

\begin{frame}[fragile]{MCQ test correction}
\protect\hypertarget{mcq-test-correction-37}{}
Q19. What does the following code print?

\begin{Shaded}
\begin{Highlighting}[]
\KeywordTok{def}\NormalTok{ print\_one():}
    \BuiltInTok{print}\NormalTok{(}\DecValTok{1}\NormalTok{)}
\end{Highlighting}
\end{Shaded}
\end{frame}

\begin{frame}[fragile]{MCQ test correction}
\protect\hypertarget{mcq-test-correction-38}{}
Q20. What does the following code print?

\begin{Shaded}
\begin{Highlighting}[]
\KeywordTok{def}\NormalTok{ print\_one():}
    \BuiltInTok{print}\NormalTok{(}\DecValTok{1}\NormalTok{)}

\NormalTok{print\_one()}
\end{Highlighting}
\end{Shaded}
\end{frame}

\begin{frame}[fragile]{MCQ test correction}
\protect\hypertarget{mcq-test-correction-39}{}
Q20. What does the following code print?

\begin{Shaded}
\begin{Highlighting}[]
\KeywordTok{def}\NormalTok{ print\_one():}
    \BuiltInTok{print}\NormalTok{(}\DecValTok{1}\NormalTok{)}

\NormalTok{print\_one()}
\end{Highlighting}
\end{Shaded}

\begin{verbatim}
## 1
\end{verbatim}
\end{frame}

\begin{frame}[fragile]{MCQ test correction}
\protect\hypertarget{mcq-test-correction-40}{}
Q21. Consider the following code:

\begin{Shaded}
\begin{Highlighting}[]
\KeywordTok{def}\NormalTok{ print\_sum(x, y):}
    \BuiltInTok{print}\NormalTok{(x }\OperatorTok{+}\NormalTok{ y)}
\end{Highlighting}
\end{Shaded}

What is the name of the above-defined function?
\end{frame}

\begin{frame}{MCQ test correction}
\protect\hypertarget{mcq-test-correction-41}{}
Q21. Consider the following code:

What is the name of the above-defined function?

print\_sum
\end{frame}

\begin{frame}[fragile]{MCQ test correction}
\protect\hypertarget{mcq-test-correction-42}{}
Q22. Consider the following code:

\begin{Shaded}
\begin{Highlighting}[]
\KeywordTok{def}\NormalTok{ print\_sum(x, y):}
    \BuiltInTok{print}\NormalTok{(x }\OperatorTok{+}\NormalTok{ y)}

\NormalTok{print\_sum(}\DecValTok{5}\NormalTok{,}\DecValTok{6}\NormalTok{)}
\end{Highlighting}
\end{Shaded}

What are the parameters of the above-defined function?
\end{frame}

\begin{frame}{MCQ test correction}
\protect\hypertarget{mcq-test-correction-43}{}
Q22. Consider the following code:

What are the parameters of the above-defined function?

(x, y)
\end{frame}

\begin{frame}[fragile]{MCQ test correction}
\protect\hypertarget{mcq-test-correction-44}{}
Q23. Consider the following code:

\begin{Shaded}
\begin{Highlighting}[]
\KeywordTok{def}\NormalTok{ print\_sum(x, y):}
    \BuiltInTok{print}\NormalTok{(x }\OperatorTok{+}\NormalTok{ y)}
\end{Highlighting}
\end{Shaded}

Which line(s) correspond(s) to the body of the above-defined function?
\end{frame}

\begin{frame}[fragile]{MCQ test correction}
\protect\hypertarget{mcq-test-correction-45}{}
Q23. Consider the following code:

Which line(s) correspond(s) to the body of the above-defined function?

\begin{Shaded}
\begin{Highlighting}[]
\KeywordTok{def}\NormalTok{ print\_sum(x, y):}
    
    \BuiltInTok{print}\NormalTok{(x }\OperatorTok{+}\NormalTok{ y)}
\end{Highlighting}
\end{Shaded}
\end{frame}

\begin{frame}[fragile]{MCQ test correction}
\protect\hypertarget{mcq-test-correction-46}{}
Q24. What does the following code print?

\begin{Shaded}
\begin{Highlighting}[]
\KeywordTok{def} \BuiltInTok{sum}\NormalTok{(x, y):}
    \BuiltInTok{print}\NormalTok{(x)}
    \ControlFlowTok{return}\NormalTok{ x }\OperatorTok{+}\NormalTok{ y}

\NormalTok{a }\OperatorTok{=} \DecValTok{1}
\NormalTok{b }\OperatorTok{=} \BuiltInTok{sum}\NormalTok{(a, }\OperatorTok{{-}}\DecValTok{1}\NormalTok{)}
\BuiltInTok{print}\NormalTok{(b)}
\end{Highlighting}
\end{Shaded}
\end{frame}

\begin{frame}[fragile]{MCQ test correction}
\protect\hypertarget{mcq-test-correction-47}{}
Q24. What does the following code print?

\begin{Shaded}
\begin{Highlighting}[]
\KeywordTok{def} \BuiltInTok{sum}\NormalTok{(x, y):}
    \BuiltInTok{print}\NormalTok{(x)}
    \ControlFlowTok{return}\NormalTok{ x }\OperatorTok{+}\NormalTok{ y}

\NormalTok{a }\OperatorTok{=} \DecValTok{1}
\NormalTok{b }\OperatorTok{=} \BuiltInTok{sum}\NormalTok{(a, }\OperatorTok{{-}}\DecValTok{1}\NormalTok{)}
\end{Highlighting}
\end{Shaded}

\begin{verbatim}
## 1
\end{verbatim}

\begin{Shaded}
\begin{Highlighting}[]
\BuiltInTok{print}\NormalTok{(b)}
\end{Highlighting}
\end{Shaded}

\begin{verbatim}
## 0
\end{verbatim}
\end{frame}

\begin{frame}[fragile]{MCQ test correction}
\protect\hypertarget{mcq-test-correction-48}{}
Q25. What does the following code print?

\begin{Shaded}
\begin{Highlighting}[]
\KeywordTok{def}\NormalTok{ fun\_a(x):}
    \BuiltInTok{print}\NormalTok{(x }\OperatorTok{{-}} \DecValTok{1}\NormalTok{)}
    \ControlFlowTok{return}\NormalTok{ x }\OperatorTok{+} \DecValTok{1}

\KeywordTok{def}\NormalTok{ fun\_b(y):}
    \BuiltInTok{print}\NormalTok{(y)}
\NormalTok{    z }\OperatorTok{=}\NormalTok{ fun\_a(x)}
    \BuiltInTok{print}\NormalTok{(z)}
    \ControlFlowTok{return}\NormalTok{ z }\OperatorTok{*} \DecValTok{2}

\NormalTok{y }\OperatorTok{=} \DecValTok{1}
\NormalTok{z }\OperatorTok{=}\NormalTok{ fun\_b(y)}
\BuiltInTok{print}\NormalTok{(z)}
\end{Highlighting}
\end{Shaded}
\end{frame}

\begin{frame}[fragile]{MCQ test correction}
\protect\hypertarget{mcq-test-correction-49}{}
Q25. What does the following code print?

\begin{Shaded}
\begin{Highlighting}[]
\KeywordTok{def}\NormalTok{ fun\_a(x):}
    \BuiltInTok{print}\NormalTok{(x }\OperatorTok{{-}} \DecValTok{1}\NormalTok{)}
    \ControlFlowTok{return}\NormalTok{ x }\OperatorTok{+} \DecValTok{1}

\KeywordTok{def}\NormalTok{ fun\_b(y):}
    \BuiltInTok{print}\NormalTok{(y)}
\NormalTok{    z }\OperatorTok{=}\NormalTok{ fun\_a(x)}
    \BuiltInTok{print}\NormalTok{(z)}
    \ControlFlowTok{return}\NormalTok{ z }\OperatorTok{*} \DecValTok{2}

\NormalTok{y }\OperatorTok{=} \DecValTok{1}
\NormalTok{z }\OperatorTok{=}\NormalTok{ fun\_b(y)}
\end{Highlighting}
\end{Shaded}

\begin{verbatim}
## 1
## 0
## 2
\end{verbatim}

\begin{Shaded}
\begin{Highlighting}[]
\BuiltInTok{print}\NormalTok{(z)}
\end{Highlighting}
\end{Shaded}

\begin{verbatim}
## 4
\end{verbatim}
\end{frame}

\begin{frame}[fragile]{MCQ test correction}
\protect\hypertarget{mcq-test-correction-50}{}
Q26. How can we extract the name (JB Lewis) from aString?

\begin{Shaded}
\begin{Highlighting}[]
\NormalTok{aString }\OperatorTok{=} \StringTok{"A long complicated string containing a quote of someone important who said \textquotesingle{}Something really important\textquotesingle{}–JB Lewis“}
\end{Highlighting}
\end{Shaded}

\begin{itemize}
\tightlist
\item
  aString{[}aString.index(``--'')+1:{]}
\item
  aString{[}index(``--'')+1:{]}
\item
  aString{[}index(``--'')+1:end{]}
\item
  aString{[}aString.index(``--'')+1:end{]}
\end{itemize}
\end{frame}

\begin{frame}[fragile]{MCQ test correction}
\protect\hypertarget{mcq-test-correction-51}{}
Q26. How can we extract the name (JB Lewis) from aString?

\begin{Shaded}
\begin{Highlighting}[]
\NormalTok{aString }\OperatorTok{=} \StringTok{"A long complicated string containing a quote of someone important who said \textquotesingle{}Something really important\textquotesingle{}–JB Lewis“}

\ErrorTok{print}\NormalTok{(aString[aString.index(}\StringTok{"–"}\NormalTok{)}\OperatorTok{+}\DecValTok{1}\NormalTok{:])}
\end{Highlighting}
\end{Shaded}

\begin{verbatim}
## EOL while scanning string literal (<string>, line 1)
\end{verbatim}
\end{frame}

\begin{frame}[fragile]{MCQ test correction}
\protect\hypertarget{mcq-test-correction-52}{}
Q27. What will the following code print?

\begin{Shaded}
\begin{Highlighting}[]
\NormalTok{listOfNums }\OperatorTok{=}\NormalTok{ [}\StringTok{"2"}\NormalTok{, }\StringTok{"7"}\NormalTok{, }\StringTok{"3"}\NormalTok{, }\StringTok{"4"}\NormalTok{]}
\NormalTok{result }\OperatorTok{=} \StringTok{"+"}\NormalTok{.join(listOfNums)}
\BuiltInTok{print}\NormalTok{(result)}
\end{Highlighting}
\end{Shaded}
\end{frame}

\begin{frame}[fragile]{MCQ test correction}
\protect\hypertarget{mcq-test-correction-53}{}
Q27. What will the following code print?

\begin{Shaded}
\begin{Highlighting}[]
\NormalTok{listOfNums }\OperatorTok{=}\NormalTok{ [}\StringTok{"2"}\NormalTok{, }\StringTok{"7"}\NormalTok{, }\StringTok{"3"}\NormalTok{, }\StringTok{"4"}\NormalTok{]}
\NormalTok{result }\OperatorTok{=} \StringTok{"+"}\NormalTok{.join(listOfNums)}
\BuiltInTok{print}\NormalTok{(result)}
\end{Highlighting}
\end{Shaded}

\begin{verbatim}
## 2+7+3+4
\end{verbatim}
\end{frame}

\begin{frame}[fragile]{MCQ test correction}
\protect\hypertarget{mcq-test-correction-54}{}
Q28.What will the following code print?

\begin{Shaded}
\begin{Highlighting}[]
\NormalTok{start }\OperatorTok{=} \StringTok{"This is a test"}
\NormalTok{end }\OperatorTok{=} \StringTok{" "}\NormalTok{.join(start.split(}\StringTok{" "}\NormalTok{))}
\BuiltInTok{print}\NormalTok{(start }\OperatorTok{==}\NormalTok{ end)}
\end{Highlighting}
\end{Shaded}
\end{frame}

\begin{frame}[fragile]{MCQ test correction}
\protect\hypertarget{mcq-test-correction-55}{}
Q28.What will the following code print?

\begin{Shaded}
\begin{Highlighting}[]
\NormalTok{start }\OperatorTok{=} \StringTok{"This is a test"}
\NormalTok{end }\OperatorTok{=} \StringTok{" "}\NormalTok{.join(start.split(}\StringTok{" "}\NormalTok{))}
\BuiltInTok{print}\NormalTok{(start }\OperatorTok{==}\NormalTok{ end)}
\end{Highlighting}
\end{Shaded}

\begin{verbatim}
## True
\end{verbatim}
\end{frame}

\begin{frame}[fragile]{MCQ test correction}
\protect\hypertarget{mcq-test-correction-56}{}
Q29. What will the following code print?

\begin{Shaded}
\begin{Highlighting}[]
\NormalTok{aStr }\OperatorTok{=} \StringTok{"This is a test"}
\NormalTok{aStr.upper()}
\BuiltInTok{print}\NormalTok{(aStr)}
\end{Highlighting}
\end{Shaded}
\end{frame}

\begin{frame}[fragile]{MCQ test correction}
\protect\hypertarget{mcq-test-correction-57}{}
Q29. What will the following code print?

\begin{Shaded}
\begin{Highlighting}[]
\NormalTok{aStr }\OperatorTok{=} \StringTok{"This is a test"}
\NormalTok{aStr.upper()}
\end{Highlighting}
\end{Shaded}

\begin{verbatim}
## 'THIS IS A TEST'
\end{verbatim}

\begin{Shaded}
\begin{Highlighting}[]
\BuiltInTok{print}\NormalTok{(aStr)}
\end{Highlighting}
\end{Shaded}

\begin{verbatim}
## This is a test
\end{verbatim}
\end{frame}

\begin{frame}[fragile]{MCQ test correction}
\protect\hypertarget{mcq-test-correction-58}{}
Q30. What are the contents of the file `test.txt' after the following
executes?

\begin{Shaded}
\begin{Highlighting}[]
\BuiltInTok{file} \OperatorTok{=} \BuiltInTok{open}\NormalTok{(}\StringTok{\textquotesingle{}test.txt\textquotesingle{}}\NormalTok{, }\StringTok{\textquotesingle{}w\textquotesingle{}}\NormalTok{)}
\BuiltInTok{file}\NormalTok{.write(}\StringTok{\textquotesingle{}This is a test, I think.\textquotesingle{}}\NormalTok{)}
\NormalTok{anotherFile }\OperatorTok{=} \BuiltInTok{open}\NormalTok{(}\StringTok{\textquotesingle{}test.txt\textquotesingle{}}\NormalTok{, }\StringTok{\textquotesingle{}r\textquotesingle{}}\NormalTok{)}
\NormalTok{contents }\OperatorTok{=}\NormalTok{ anotherFile.read()}
\NormalTok{yetAnotherOne }\OperatorTok{=} \BuiltInTok{open}\NormalTok{(}\StringTok{\textquotesingle{}test.txt\textquotesingle{}}\NormalTok{, }\StringTok{\textquotesingle{}a\textquotesingle{}}\NormalTok{)}
\NormalTok{yetAnotherOne.write(}\StringTok{\textquotesingle{} Yes\textquotesingle{}}\NormalTok{)}
\end{Highlighting}
\end{Shaded}
\end{frame}

\begin{frame}[fragile]{MCQ test correction}
\protect\hypertarget{mcq-test-correction-59}{}
Q30. What are the contents of the file `test.txt' after the following
executes?

\begin{Shaded}
\begin{Highlighting}[]
\BuiltInTok{file} \OperatorTok{=} \BuiltInTok{open}\NormalTok{(}\StringTok{\textquotesingle{}test.txt\textquotesingle{}}\NormalTok{, }\StringTok{\textquotesingle{}w\textquotesingle{}}\NormalTok{)}
\BuiltInTok{file}\NormalTok{.write(}\StringTok{\textquotesingle{}This is a test, I think.\textquotesingle{}}\NormalTok{)}
\end{Highlighting}
\end{Shaded}

\begin{verbatim}
## 24
\end{verbatim}

\begin{Shaded}
\begin{Highlighting}[]
\NormalTok{anotherFile }\OperatorTok{=} \BuiltInTok{open}\NormalTok{(}\StringTok{\textquotesingle{}test.txt\textquotesingle{}}\NormalTok{, }\StringTok{\textquotesingle{}r\textquotesingle{}}\NormalTok{)}
\NormalTok{contents }\OperatorTok{=}\NormalTok{ anotherFile.read()}
\NormalTok{yetAnotherOne }\OperatorTok{=} \BuiltInTok{open}\NormalTok{(}\StringTok{\textquotesingle{}test.txt\textquotesingle{}}\NormalTok{, }\StringTok{\textquotesingle{}a\textquotesingle{}}\NormalTok{)}
\NormalTok{yetAnotherOne.write(}\StringTok{\textquotesingle{} Yes\textquotesingle{}}\NormalTok{)}
\end{Highlighting}
\end{Shaded}

\begin{verbatim}
## 4
\end{verbatim}
\end{frame}

\begin{frame}[fragile]{MCQ test correction}
\protect\hypertarget{mcq-test-correction-60}{}
Q31. In which directory is the file test.txt located?

\begin{Shaded}
\begin{Highlighting}[]
\BuiltInTok{file} \OperatorTok{=} \BuiltInTok{open}\NormalTok{(}\StringTok{\textquotesingle{}test.txt\textquotesingle{}}\NormalTok{, }\StringTok{\textquotesingle{}w\textquotesingle{}}\NormalTok{)}
\BuiltInTok{file}\NormalTok{.write(}\StringTok{"This is a test, I think."}\NormalTok{)}
\BuiltInTok{file}\NormalTok{.close()}
\end{Highlighting}
\end{Shaded}
\end{frame}

\begin{frame}[fragile]{MCQ test correction}
\protect\hypertarget{mcq-test-correction-61}{}
Q31. In which directory is the file test.txt located?

\begin{Shaded}
\begin{Highlighting}[]
\BuiltInTok{file} \OperatorTok{=} \BuiltInTok{open}\NormalTok{(}\StringTok{\textquotesingle{}test.txt\textquotesingle{}}\NormalTok{, }\StringTok{\textquotesingle{}w\textquotesingle{}}\NormalTok{)}
\BuiltInTok{file}\NormalTok{.write(}\StringTok{"This is a test, I think."}\NormalTok{)}
\end{Highlighting}
\end{Shaded}

\begin{verbatim}
## 24
\end{verbatim}

\begin{Shaded}
\begin{Highlighting}[]
\BuiltInTok{file}\NormalTok{.close()}
\end{Highlighting}
\end{Shaded}
\end{frame}

\begin{frame}{MCQ test correction}
\protect\hypertarget{mcq-test-correction-62}{}
Q32. Should one generally prefer absolute or relative paths to files and
directories?

\begin{itemize}
\tightlist
\item
  Absolute
\item
  Relative
\item
  It doesn't matter
\item
  Randomly choose one of the two
\end{itemize}
\end{frame}

\begin{frame}{MCQ test correction}
\protect\hypertarget{mcq-test-correction-63}{}
Q32. Should one generally prefer absolute or relative paths to files and
directories?

\begin{itemize}
\tightlist
\item
  Absolute
\item
  \textbf{Relative}
\item
  It doesn't matter
\item
  Randomly choose one of the two
\end{itemize}
\end{frame}

\begin{frame}[fragile]{MCQ test correction}
\protect\hypertarget{mcq-test-correction-64}{}
Q33. Why is it often useful to name a piece of data? e.g.~(several
options possible)

\begin{Shaded}
\begin{Highlighting}[]
\NormalTok{ipLine }\OperatorTok{=}\NormalTok{ data[}\DecValTok{2}\NormalTok{]}
\end{Highlighting}
\end{Shaded}
\end{frame}

\begin{frame}[fragile]{MCQ test correction}
\protect\hypertarget{mcq-test-correction-65}{}
Q33. Why is it often useful to name a piece of data? e.g.~(several
options possible)

\begin{Shaded}
\begin{Highlighting}[]
\NormalTok{ipLine }\OperatorTok{=}\NormalTok{ data[}\DecValTok{2}\NormalTok{]}
\end{Highlighting}
\end{Shaded}

\begin{verbatim}
## name 'data' is not defined
\end{verbatim}
\end{frame}

\begin{frame}[fragile]{MCQ test correction}
\protect\hypertarget{mcq-test-correction-66}{}
Q34. In the following code, what is the value of fileName?

\begin{Shaded}
\begin{Highlighting}[]
\KeywordTok{def}\NormalTok{ wordCount(fileName, word):}
    \BuiltInTok{file}  \OperatorTok{=} \BuiltInTok{open}\NormalTok{(fileName, }\StringTok{\textquotesingle{}r\textquotesingle{}}\NormalTok{)}
\NormalTok{    contents }\OperatorTok{=} \BuiltInTok{file}\NormalTok{.read()}
\NormalTok{    words }\OperatorTok{=}\NormalTok{ contents.split(}\StringTok{\textquotesingle{} \textquotesingle{}}\NormalTok{)}
\NormalTok{        wordCount }\OperatorTok{=}\NormalTok{ words.count(word)}
        \BuiltInTok{print}\NormalTok{(wordCount)}
\end{Highlighting}
\end{Shaded}
\end{frame}

\begin{frame}[fragile]{MCQ test correction}
\protect\hypertarget{mcq-test-correction-67}{}
Q34. In the following code, what is the value of fileName?

\begin{Shaded}
\begin{Highlighting}[]
\KeywordTok{def}\NormalTok{ wordCount(fileName, word):}
    \BuiltInTok{file}  \OperatorTok{=} \BuiltInTok{open}\NormalTok{(fileName, }\StringTok{\textquotesingle{}r\textquotesingle{}}\NormalTok{)}
\NormalTok{    contents }\OperatorTok{=} \BuiltInTok{file}\NormalTok{.read()}
\NormalTok{    words }\OperatorTok{=}\NormalTok{ contents.split(}\StringTok{\textquotesingle{} \textquotesingle{}}\NormalTok{)}
\NormalTok{        wordCount }\OperatorTok{=}\NormalTok{ words.count(word)}
        \BuiltInTok{print}\NormalTok{(wordCount)}
\end{Highlighting}
\end{Shaded}

\begin{verbatim}
## inconsistent use of tabs and spaces in indentation (<string>, line 5)
\end{verbatim}
\end{frame}

\begin{frame}[fragile]{MCQ test correction}
\protect\hypertarget{mcq-test-correction-68}{}
Q35. What will be printed after this program runs?

\begin{Shaded}
\begin{Highlighting}[]
\KeywordTok{def}\NormalTok{ test():}
    \BuiltInTok{print}\NormalTok{(}\StringTok{"This will print 12"}\NormalTok{)}
\end{Highlighting}
\end{Shaded}
\end{frame}

\begin{frame}[fragile]{MCQ test correction}
\protect\hypertarget{mcq-test-correction-69}{}
Q35. What will be printed after this program runs?

\begin{Shaded}
\begin{Highlighting}[]
\KeywordTok{def}\NormalTok{ test():}
    \BuiltInTok{print}\NormalTok{(}\StringTok{"This will print 12"}\NormalTok{)}
\end{Highlighting}
\end{Shaded}
\end{frame}

\begin{frame}[fragile]{MCQ test correction}
\protect\hypertarget{mcq-test-correction-70}{}
Q36. What will be printed when the following program executes?

\begin{Shaded}
\begin{Highlighting}[]
\KeywordTok{def}\NormalTok{ aProgram():}
\BuiltInTok{print}\NormalTok{(}\StringTok{"This will print 12"}\NormalTok{)}
\NormalTok{aProgram()}
\end{Highlighting}
\end{Shaded}
\end{frame}

\begin{frame}[fragile]{MCQ test correction}
\protect\hypertarget{mcq-test-correction-71}{}
Q36. What will be printed when the following program executes?

\begin{Shaded}
\begin{Highlighting}[]
\KeywordTok{def}\NormalTok{ aProgram():}
\BuiltInTok{print}\NormalTok{(}\StringTok{"This will print 12"}\NormalTok{)}
\NormalTok{aProgram()}
\end{Highlighting}
\end{Shaded}

\begin{verbatim}
## expected an indented block (<string>, line 2)
\end{verbatim}
\end{frame}

\begin{frame}[fragile]{MCQ test correction}
\protect\hypertarget{mcq-test-correction-72}{}
Q37. What will be printed when the following program executes?

\begin{Shaded}
\begin{Highlighting}[]
\KeywordTok{def}\NormalTok{ aProgram():}
    \BuiltInTok{print}\NormalTok{(}\StringTok{"This will print 12"}\NormalTok{)}
\NormalTok{aProgram()}
\end{Highlighting}
\end{Shaded}
\end{frame}

\begin{frame}[fragile]{MCQ test correction}
\protect\hypertarget{mcq-test-correction-73}{}
Q37. What will be printed when the following program executes?

\begin{Shaded}
\begin{Highlighting}[]
\KeywordTok{def}\NormalTok{ aProgram():}
    \BuiltInTok{print}\NormalTok{(}\StringTok{"This will print 12"}\NormalTok{)}
\NormalTok{aProgram()}
\end{Highlighting}
\end{Shaded}

\begin{verbatim}
## This will print 12
\end{verbatim}
\end{frame}

\begin{frame}[fragile]{MCQ test correction}
\protect\hypertarget{mcq-test-correction-74}{}
Q38. What will be printed when the following program is executed?

\begin{Shaded}
\begin{Highlighting}[]
\KeywordTok{def}\NormalTok{ myFun(aNum):}
    \ControlFlowTok{return}\NormalTok{ aNum }\OperatorTok{*} \DecValTok{2}

\NormalTok{myNumber }\OperatorTok{=} \DecValTok{4}
\NormalTok{k }\OperatorTok{=}\NormalTok{ myFun(myNumber)}
\BuiltInTok{print}\NormalTok{(k)}
\end{Highlighting}
\end{Shaded}
\end{frame}

\begin{frame}[fragile]{MCQ test correction}
\protect\hypertarget{mcq-test-correction-75}{}
Q38. What will be printed when the following program is executed?

\begin{Shaded}
\begin{Highlighting}[]
\KeywordTok{def}\NormalTok{ myFun(aNum):}
    \ControlFlowTok{return}\NormalTok{ aNum }\OperatorTok{*} \DecValTok{2}

\NormalTok{myNumber }\OperatorTok{=} \DecValTok{4}
\NormalTok{k }\OperatorTok{=}\NormalTok{ myFun(myNumber)}
\BuiltInTok{print}\NormalTok{(k)}
\end{Highlighting}
\end{Shaded}

\begin{verbatim}
## 8
\end{verbatim}
\end{frame}

\begin{frame}[fragile]{MCQ test correction}
\protect\hypertarget{mcq-test-correction-76}{}
Q39. What will be printed when the following program is executed?

\begin{Shaded}
\begin{Highlighting}[]
\KeywordTok{def}\NormalTok{ myFun(aNum):}
\NormalTok{    a }\OperatorTok{=}\NormalTok{ aNum }\OperatorTok{*} \DecValTok{2}

\NormalTok{myNumber }\OperatorTok{=} \DecValTok{4}
\NormalTok{k }\OperatorTok{=}\NormalTok{ myFun(myNumber)}
\BuiltInTok{print}\NormalTok{(k)}
\end{Highlighting}
\end{Shaded}
\end{frame}

\begin{frame}[fragile]{MCQ test correction}
\protect\hypertarget{mcq-test-correction-77}{}
Q39. What will be printed when the following program is executed?

\begin{Shaded}
\begin{Highlighting}[]
\KeywordTok{def}\NormalTok{ myFun(aNum):}
\NormalTok{    a }\OperatorTok{=}\NormalTok{ aNum }\OperatorTok{*} \DecValTok{2}

\NormalTok{myNumber }\OperatorTok{=} \DecValTok{4}
\NormalTok{k }\OperatorTok{=}\NormalTok{ myFun(myNumber)}
\BuiltInTok{print}\NormalTok{(k)}
\end{Highlighting}
\end{Shaded}

\begin{verbatim}
## None
\end{verbatim}
\end{frame}

\begin{frame}[fragile]{MCQ test correction}
\protect\hypertarget{mcq-test-correction-78}{}
Q40. What will be printed when the following program is executed?

\begin{Shaded}
\begin{Highlighting}[]
\KeywordTok{def}\NormalTok{ myFun(aNum):}
\NormalTok{    a }\OperatorTok{=}\NormalTok{ aNum }\OperatorTok{*} \DecValTok{2}
    \ControlFlowTok{return}\NormalTok{ a}
\NormalTok{myNumber }\OperatorTok{=} \DecValTok{4}
\NormalTok{k }\OperatorTok{=}\NormalTok{ myFun(myNumber)}
\BuiltInTok{print}\NormalTok{(k)}
\end{Highlighting}
\end{Shaded}
\end{frame}

\begin{frame}[fragile]{MCQ test correction}
\protect\hypertarget{mcq-test-correction-79}{}
Q40. What will be printed when the following program is executed?

\begin{Shaded}
\begin{Highlighting}[]
\KeywordTok{def}\NormalTok{ myFun(aNum):}
\NormalTok{    a }\OperatorTok{=}\NormalTok{ aNum }\OperatorTok{*} \DecValTok{2}
    \ControlFlowTok{return}\NormalTok{ a}
\NormalTok{myNumber }\OperatorTok{=} \DecValTok{4}
\NormalTok{k }\OperatorTok{=}\NormalTok{ myFun(myNumber)}
\BuiltInTok{print}\NormalTok{(k)}
\end{Highlighting}
\end{Shaded}

\begin{verbatim}
## 8
\end{verbatim}
\end{frame}

\begin{frame}[fragile]{MCQ test correction}
\protect\hypertarget{mcq-test-correction-80}{}
Q41. What will be printed when the following program is executed?

\begin{Shaded}
\begin{Highlighting}[]
\KeywordTok{def}\NormalTok{ myFun(aNum):}
\NormalTok{    a }\OperatorTok{=}\NormalTok{ aNum }\OperatorTok{*} \DecValTok{2}
    \BuiltInTok{print}\NormalTok{(a)}

\NormalTok{myNumber }\OperatorTok{=} \DecValTok{4}
\NormalTok{k }\OperatorTok{=}\NormalTok{ myFun(myNumber)}
\BuiltInTok{print}\NormalTok{(k)}
\end{Highlighting}
\end{Shaded}
\end{frame}

\begin{frame}[fragile]{MCQ test correction}
\protect\hypertarget{mcq-test-correction-81}{}
Q41. What will be printed when the following program is executed?

\begin{Shaded}
\begin{Highlighting}[]
\KeywordTok{def}\NormalTok{ myFun(aNum):}
\NormalTok{    a }\OperatorTok{=}\NormalTok{ aNum }\OperatorTok{*} \DecValTok{2}
    \BuiltInTok{print}\NormalTok{(a)}

\NormalTok{myNumber }\OperatorTok{=} \DecValTok{4}
\NormalTok{k }\OperatorTok{=}\NormalTok{ myFun(myNumber)}
\end{Highlighting}
\end{Shaded}

\begin{verbatim}
## 8
\end{verbatim}

\begin{Shaded}
\begin{Highlighting}[]
\BuiltInTok{print}\NormalTok{(k)}
\end{Highlighting}
\end{Shaded}

\begin{verbatim}
## None
\end{verbatim}
\end{frame}

\begin{frame}[fragile]{MCQ test correction}
\protect\hypertarget{mcq-test-correction-82}{}
Q42. What's the value of the variable k after this program executes?

\begin{Shaded}
\begin{Highlighting}[]
\KeywordTok{def}\NormalTok{ myFun(aNum):}
\NormalTok{    a }\OperatorTok{=}\NormalTok{ aNum }\OperatorTok{*} \DecValTok{2}
    \BuiltInTok{print}\NormalTok{(a)}

\NormalTok{myNumber }\OperatorTok{=} \DecValTok{4}
\NormalTok{k }\OperatorTok{=}\NormalTok{ myFun(myNumber)}
\BuiltInTok{print}\NormalTok{(k)}
\end{Highlighting}
\end{Shaded}
\end{frame}

\begin{frame}[fragile]{MCQ test correction}
\protect\hypertarget{mcq-test-correction-83}{}
Q42. What's the value of the variable k after this program executes?

\begin{Shaded}
\begin{Highlighting}[]
\KeywordTok{def}\NormalTok{ myFun(aNum):}
\NormalTok{    a }\OperatorTok{=}\NormalTok{ aNum }\OperatorTok{*} \DecValTok{2}
    \BuiltInTok{print}\NormalTok{(a)}

\NormalTok{myNumber }\OperatorTok{=} \DecValTok{4}
\NormalTok{k }\OperatorTok{=}\NormalTok{ myFun(myNumber)}
\end{Highlighting}
\end{Shaded}

\begin{verbatim}
## 8
\end{verbatim}

\begin{Shaded}
\begin{Highlighting}[]
\BuiltInTok{print}\NormalTok{(k)}
\end{Highlighting}
\end{Shaded}

\begin{verbatim}
## None
\end{verbatim}
\end{frame}

\begin{frame}[fragile]{MCQ test correction}
\protect\hypertarget{mcq-test-correction-84}{}
Q43. What is/are the correct way(s) to call the following function?
(assume all variables exist. correct means here no error display on the
terminal)

\begin{Shaded}
\begin{Highlighting}[]
\KeywordTok{def}\NormalTok{ readResults(resultFile):}
    \BuiltInTok{file} \OperatorTok{=} \BuiltInTok{open}\NormalTok{(resultFile, }\StringTok{\textquotesingle{}r\textquotesingle{}}\NormalTok{)}\OperatorTok{;}
\NormalTok{    contents }\OperatorTok{=} \BuiltInTok{file}\NormalTok{.read()}
    \ControlFlowTok{return}\NormalTok{ contents}
\end{Highlighting}
\end{Shaded}
\end{frame}

\begin{frame}[fragile]{MCQ test correction}
\protect\hypertarget{mcq-test-correction-85}{}
Q43. What is/are the correct way(s) to call the following function?
(assume all variables exist. correct means here no error display on the
terminal)

\begin{Shaded}
\begin{Highlighting}[]
\KeywordTok{def}\NormalTok{ readResults(resultFile):}
    \BuiltInTok{file} \OperatorTok{=} \BuiltInTok{open}\NormalTok{(resultFile, }\StringTok{\textquotesingle{}r\textquotesingle{}}\NormalTok{)}\OperatorTok{;}
\NormalTok{    contents }\OperatorTok{=} \BuiltInTok{file}\NormalTok{.read()}
    \ControlFlowTok{return}\NormalTok{ contents}
\end{Highlighting}
\end{Shaded}
\end{frame}

\begin{frame}[fragile]{MCQ test correction}
\protect\hypertarget{mcq-test-correction-86}{}
Q44. What will the following code print?

\begin{Shaded}
\begin{Highlighting}[]
\KeywordTok{def}\NormalTok{ functionA(aNumber):}
    \ControlFlowTok{return}\NormalTok{ aNumber}\OperatorTok{+}\DecValTok{1}
\KeywordTok{def}\NormalTok{ functionB(aNumber):}
    \ControlFlowTok{return}\NormalTok{ functionA(aNumber)}\OperatorTok{+}\DecValTok{1}
\BuiltInTok{print}\NormalTok{(functionB(}\DecValTok{1}\NormalTok{))}
\end{Highlighting}
\end{Shaded}
\end{frame}

\begin{frame}[fragile]{MCQ test correction}
\protect\hypertarget{mcq-test-correction-87}{}
Q44. What will the following code print?

\begin{Shaded}
\begin{Highlighting}[]
\KeywordTok{def}\NormalTok{ functionA(aNumber):}
    \ControlFlowTok{return}\NormalTok{ aNumber}\OperatorTok{+}\DecValTok{1}
\KeywordTok{def}\NormalTok{ functionB(aNumber):}
    \ControlFlowTok{return}\NormalTok{ functionA(aNumber)}\OperatorTok{+}\DecValTok{1}
\BuiltInTok{print}\NormalTok{(functionB(}\DecValTok{1}\NormalTok{))}
\end{Highlighting}
\end{Shaded}

\begin{verbatim}
## 3
\end{verbatim}
\end{frame}

\begin{frame}[fragile]{MCQ test correction}
\protect\hypertarget{mcq-test-correction-88}{}
Q45. What will the following program print?

\begin{Shaded}
\begin{Highlighting}[]
\NormalTok{aVar }\OperatorTok{=} \DecValTok{12}
\KeywordTok{def}\NormalTok{ myFun():}
\NormalTok{    aVar }\OperatorTok{=} \DecValTok{7}\OperatorTok{+}\DecValTok{3}
\BuiltInTok{print}\NormalTok{(aVar)}
\end{Highlighting}
\end{Shaded}
\end{frame}

\begin{frame}[fragile]{MCQ test correction}
\protect\hypertarget{mcq-test-correction-89}{}
Q45. What will the following program print?

\begin{Shaded}
\begin{Highlighting}[]
\NormalTok{aVar }\OperatorTok{=} \DecValTok{12}
\KeywordTok{def}\NormalTok{ myFun():}
\NormalTok{    aVar }\OperatorTok{=} \DecValTok{7}\OperatorTok{+}\DecValTok{3}
\BuiltInTok{print}\NormalTok{(aVar)}
\end{Highlighting}
\end{Shaded}

\begin{verbatim}
## 12
\end{verbatim}
\end{frame}

\begin{frame}[fragile]{MCQ test correction}
\protect\hypertarget{mcq-test-correction-90}{}
Q46. What will the following program print?

\begin{Shaded}
\begin{Highlighting}[]
\NormalTok{aVar }\OperatorTok{=} \DecValTok{12}
\KeywordTok{def}\NormalTok{ myFun():}
\NormalTok{    aVar }\OperatorTok{=} \DecValTok{7}\OperatorTok{+}\DecValTok{3}
\BuiltInTok{print}\NormalTok{ (aVar)}
\NormalTok{myFun()}
\end{Highlighting}
\end{Shaded}
\end{frame}

\begin{frame}[fragile]{MCQ test correction}
\protect\hypertarget{mcq-test-correction-91}{}
Q46. What will the following program print?

\begin{Shaded}
\begin{Highlighting}[]
\NormalTok{aVar }\OperatorTok{=} \DecValTok{12}
\KeywordTok{def}\NormalTok{ myFun():}
\NormalTok{    aVar }\OperatorTok{=} \DecValTok{7}\OperatorTok{+}\DecValTok{3}
\BuiltInTok{print}\NormalTok{ (aVar)}
\end{Highlighting}
\end{Shaded}

\begin{verbatim}
## 12
\end{verbatim}

\begin{Shaded}
\begin{Highlighting}[]
\NormalTok{myFun()}
\end{Highlighting}
\end{Shaded}
\end{frame}

\begin{frame}[fragile]{MCQ test correction}
\protect\hypertarget{mcq-test-correction-92}{}
Q47. What will the following program print?

\begin{Shaded}
\begin{Highlighting}[]
\NormalTok{aVar }\OperatorTok{=} \DecValTok{12}
\KeywordTok{def}\NormalTok{ myFun():}
\NormalTok{    aVar }\OperatorTok{=} \DecValTok{7}\OperatorTok{+}\DecValTok{3}
    \BuiltInTok{print}\NormalTok{ (aVar)}
\NormalTok{myFun()}
\end{Highlighting}
\end{Shaded}
\end{frame}

\begin{frame}[fragile]{MCQ test correction}
\protect\hypertarget{mcq-test-correction-93}{}
Q47. What will the following program print?

\begin{Shaded}
\begin{Highlighting}[]
\NormalTok{aVar }\OperatorTok{=} \DecValTok{12}
\KeywordTok{def}\NormalTok{ myFun():}
\NormalTok{    aVar }\OperatorTok{=} \DecValTok{7}\OperatorTok{+}\DecValTok{3}
    \BuiltInTok{print}\NormalTok{ (aVar)}
\NormalTok{myFun()}
\end{Highlighting}
\end{Shaded}

\begin{verbatim}
## 10
\end{verbatim}
\end{frame}

\begin{frame}[fragile]{MCQ test correction}
\protect\hypertarget{mcq-test-correction-94}{}
Q48. What will the following program print?

\begin{Shaded}
\begin{Highlighting}[]
\NormalTok{aVar }\OperatorTok{=} \DecValTok{12}
\KeywordTok{def}\NormalTok{ myFun():}
\NormalTok{    aVar }\OperatorTok{=} \DecValTok{7}\OperatorTok{+}\DecValTok{3}
    \ControlFlowTok{return}\NormalTok{ aVar}
\BuiltInTok{print}\NormalTok{(myFun())}
\end{Highlighting}
\end{Shaded}
\end{frame}

\begin{frame}[fragile]{MCQ test correction}
\protect\hypertarget{mcq-test-correction-95}{}
Q48. What will the following program print?

\begin{Shaded}
\begin{Highlighting}[]
\NormalTok{aVar }\OperatorTok{=} \DecValTok{12}
\KeywordTok{def}\NormalTok{ myFun():}
\NormalTok{    aVar }\OperatorTok{=} \DecValTok{7}\OperatorTok{+}\DecValTok{3}
    \ControlFlowTok{return}\NormalTok{ aVar}
\BuiltInTok{print}\NormalTok{(myFun())}
\end{Highlighting}
\end{Shaded}

\begin{verbatim}
## 10
\end{verbatim}
\end{frame}

\begin{frame}[fragile]{MCQ test correction}
\protect\hypertarget{mcq-test-correction-96}{}
Q49. What will the following program print?

\begin{Shaded}
\begin{Highlighting}[]
\NormalTok{myVar }\OperatorTok{=} \DecValTok{12}
\KeywordTok{def}\NormalTok{ myFun():}
\NormalTok{    aVar }\OperatorTok{=} \DecValTok{7}\OperatorTok{+}\DecValTok{3}
    \ControlFlowTok{return}\NormalTok{ aVar}
\NormalTok{myVar }\OperatorTok{=}\NormalTok{ myFun()}
\BuiltInTok{print}\NormalTok{(myVar)}
\end{Highlighting}
\end{Shaded}
\end{frame}

\begin{frame}[fragile]{MCQ test correction}
\protect\hypertarget{mcq-test-correction-97}{}
Q49. What will the following program print?

\begin{Shaded}
\begin{Highlighting}[]
\NormalTok{myVar }\OperatorTok{=} \DecValTok{12}
\KeywordTok{def}\NormalTok{ myFun():}
\NormalTok{    aVar }\OperatorTok{=} \DecValTok{7}\OperatorTok{+}\DecValTok{3}
    \ControlFlowTok{return}\NormalTok{ aVar}
\NormalTok{myVar }\OperatorTok{=}\NormalTok{ myFun()}
\BuiltInTok{print}\NormalTok{(myVar)}
\end{Highlighting}
\end{Shaded}

\begin{verbatim}
## 10
\end{verbatim}
\end{frame}

\begin{frame}[fragile]{MCQ test correction}
\protect\hypertarget{mcq-test-correction-98}{}
Q50. What will the following program print?

\begin{Shaded}
\begin{Highlighting}[]
\KeywordTok{def}\NormalTok{ myFun(aVar):}
\NormalTok{    test }\OperatorTok{=}\NormalTok{ aVar}\OperatorTok{+}\DecValTok{3}
    \ControlFlowTok{return}\NormalTok{ test}
\NormalTok{aVar }\OperatorTok{=} \DecValTok{15}
\NormalTok{bVar }\OperatorTok{=}\NormalTok{ myFun(aVar)}
\BuiltInTok{print}\NormalTok{(aVar)}
\end{Highlighting}
\end{Shaded}
\end{frame}

\begin{frame}[fragile]{MCQ test correction}
\protect\hypertarget{mcq-test-correction-99}{}
Q50. What will the following program print?

\begin{Shaded}
\begin{Highlighting}[]
\KeywordTok{def}\NormalTok{ myFun(aVar):}
\NormalTok{    test }\OperatorTok{=}\NormalTok{ aVar}\OperatorTok{+}\DecValTok{3}
    \ControlFlowTok{return}\NormalTok{ test}
\NormalTok{aVar }\OperatorTok{=} \DecValTok{15}
\NormalTok{bVar }\OperatorTok{=}\NormalTok{ myFun(aVar)}
\BuiltInTok{print}\NormalTok{(aVar)}
\end{Highlighting}
\end{Shaded}

\begin{verbatim}
## 15
\end{verbatim}
\end{frame}

\begin{frame}[fragile]{MCQ test correction}
\protect\hypertarget{mcq-test-correction-100}{}
Q51. What will the following program print?

\begin{Shaded}
\begin{Highlighting}[]
\KeywordTok{def}\NormalTok{ myFun(aVar):}
\NormalTok{    test }\OperatorTok{=}\NormalTok{ aVar}\OperatorTok{+}\DecValTok{3}
    \ControlFlowTok{return}\NormalTok{ test}
\NormalTok{bVar }\OperatorTok{=} \DecValTok{15}
\NormalTok{bVar }\OperatorTok{=}\NormalTok{ myFun(bVar)}
\BuiltInTok{print}\NormalTok{(bVar)}
\end{Highlighting}
\end{Shaded}
\end{frame}

\begin{frame}[fragile]{MCQ test correction}
\protect\hypertarget{mcq-test-correction-101}{}
Q51. What will the following program print?

\begin{Shaded}
\begin{Highlighting}[]
\KeywordTok{def}\NormalTok{ myFun(aVar):}
\NormalTok{    test }\OperatorTok{=}\NormalTok{ aVar}\OperatorTok{+}\DecValTok{3}
    \ControlFlowTok{return}\NormalTok{ test}
\NormalTok{bVar }\OperatorTok{=} \DecValTok{15}
\NormalTok{bVar }\OperatorTok{=}\NormalTok{ myFun(bVar)}
\BuiltInTok{print}\NormalTok{(bVar)}
\end{Highlighting}
\end{Shaded}

\begin{verbatim}
## 18
\end{verbatim}
\end{frame}

\begin{frame}[fragile]{MCQ test correction}
\protect\hypertarget{mcq-test-correction-102}{}
Q52. What will the following program print?

\begin{Shaded}
\begin{Highlighting}[]
\KeywordTok{def}\NormalTok{ myFun(aList):}
\NormalTok{    aList[}\DecValTok{1}\NormalTok{] }\OperatorTok{=} \DecValTok{100}

\NormalTok{test }\OperatorTok{=}\NormalTok{ [}\DecValTok{1}\NormalTok{, }\DecValTok{2}\NormalTok{, }\DecValTok{3}\NormalTok{]}
\NormalTok{myFun(test)}
\BuiltInTok{print}\NormalTok{(test[}\DecValTok{1}\NormalTok{])}
\end{Highlighting}
\end{Shaded}
\end{frame}

\begin{frame}[fragile]{MCQ test correction}
\protect\hypertarget{mcq-test-correction-103}{}
Q52. What will the following program print?

\begin{Shaded}
\begin{Highlighting}[]
\KeywordTok{def}\NormalTok{ myFun(aList):}
\NormalTok{    aList[}\DecValTok{1}\NormalTok{] }\OperatorTok{=} \DecValTok{100}

\NormalTok{test }\OperatorTok{=}\NormalTok{ [}\DecValTok{1}\NormalTok{, }\DecValTok{2}\NormalTok{, }\DecValTok{3}\NormalTok{]}
\NormalTok{myFun(test)}
\BuiltInTok{print}\NormalTok{(test[}\DecValTok{1}\NormalTok{])}
\end{Highlighting}
\end{Shaded}

\begin{verbatim}
## 100
\end{verbatim}
\end{frame}

\begin{frame}[fragile]{MCQ test correction}
\protect\hypertarget{mcq-test-correction-104}{}
Q53. What's the strangest part of the following function definition?

\begin{Shaded}
\begin{Highlighting}[]
\KeywordTok{def}\NormalTok{ myFun(aVar):}
\NormalTok{    aVar }\OperatorTok{=} \DecValTok{3}
    \ControlFlowTok{return}\NormalTok{ aVar }\OperatorTok{+} \DecValTok{5}
\end{Highlighting}
\end{Shaded}
\end{frame}

\begin{frame}[fragile]{MCQ test correction}
\protect\hypertarget{mcq-test-correction-105}{}
Q53. What's the strangest part of the following function definition?

\begin{Shaded}
\begin{Highlighting}[]
\KeywordTok{def}\NormalTok{ myFun(aVar):}
\NormalTok{    aVar }\OperatorTok{=} \DecValTok{3}
    \ControlFlowTok{return}\NormalTok{ aVar }\OperatorTok{+} \DecValTok{5}
\end{Highlighting}
\end{Shaded}
\end{frame}

\begin{frame}[fragile]{MCQ test correction}
\protect\hypertarget{mcq-test-correction-106}{}
Q54. What will the following code print?

\begin{Shaded}
\begin{Highlighting}[]
\KeywordTok{def}\NormalTok{ validateAge(userAge):}
    \ControlFlowTok{if}\NormalTok{ userAge}\OperatorTok{\textless{}}\DecValTok{18}\NormalTok{:}
        \ControlFlowTok{return} \VariableTok{False}
    \ControlFlowTok{else}\NormalTok{:}
        \ControlFlowTok{return} \VariableTok{True}

\KeywordTok{def}\NormalTok{ isAdult():}
\NormalTok{    age }\OperatorTok{=} \BuiltInTok{input}\NormalTok{(}\StringTok{"Please type your age"}\NormalTok{) }\CommentTok{\# assume user input 19}
    \ControlFlowTok{return}\NormalTok{ validateAge(}\BuiltInTok{int}\NormalTok{(age))}

\BuiltInTok{print}\NormalTok{(isAdult())}
\end{Highlighting}
\end{Shaded}
\end{frame}

\begin{frame}[fragile]{MCQ test correction}
\protect\hypertarget{mcq-test-correction-107}{}
Q54. What will the following code print?

\begin{Shaded}
\begin{Highlighting}[]
\KeywordTok{def}\NormalTok{ validateAge(userAge):}
    \ControlFlowTok{if}\NormalTok{ userAge}\OperatorTok{\textless{}}\DecValTok{18}\NormalTok{:}
        \ControlFlowTok{return} \VariableTok{False}
    \ControlFlowTok{else}\NormalTok{:}
        \ControlFlowTok{return} \VariableTok{True}

\KeywordTok{def}\NormalTok{ isAdult():}
\NormalTok{    age }\OperatorTok{=} \BuiltInTok{input}\NormalTok{(}\StringTok{"Please type your age"}\NormalTok{) }\CommentTok{\# assume user input 19}
    \ControlFlowTok{return}\NormalTok{ validateAge(}\BuiltInTok{int}\NormalTok{(age))}

\BuiltInTok{print}\NormalTok{(isAdult())}
\end{Highlighting}
\end{Shaded}
\end{frame}

\begin{frame}[fragile]{MCQ test correction}
\protect\hypertarget{mcq-test-correction-108}{}
Q55. What is the value of aVar after the following executes?

\begin{Shaded}
\begin{Highlighting}[]
\KeywordTok{def}\NormalTok{ someGreatFunction():}
    \ControlFlowTok{return} \DecValTok{12}
\NormalTok{aVar }\OperatorTok{=}\NormalTok{ someGreatFunction}
\end{Highlighting}
\end{Shaded}
\end{frame}

\begin{frame}[fragile]{MCQ test correction}
\protect\hypertarget{mcq-test-correction-109}{}
Q55. What is the value of aVar after the following executes?

\begin{Shaded}
\begin{Highlighting}[]
\KeywordTok{def}\NormalTok{ someGreatFunction():}
    \ControlFlowTok{return} \DecValTok{12}
\NormalTok{aVar }\OperatorTok{=}\NormalTok{ someGreatFunction}
\end{Highlighting}
\end{Shaded}
\end{frame}

\begin{frame}[fragile]{MCQ test correction}
\protect\hypertarget{mcq-test-correction-110}{}
Q56. What is the value of aVar after the following executes?

\begin{Shaded}
\begin{Highlighting}[]
\KeywordTok{def}\NormalTok{ someGreatFunction():}
    \ControlFlowTok{return} \DecValTok{12}
\NormalTok{bVar }\OperatorTok{=}\NormalTok{ someGreatFunction}
\NormalTok{aVar }\OperatorTok{=}\NormalTok{ bVar()}
\end{Highlighting}
\end{Shaded}
\end{frame}

\begin{frame}[fragile]{MCQ test correction}
\protect\hypertarget{mcq-test-correction-111}{}
Q56. What is the value of aVar after the following executes?

\begin{Shaded}
\begin{Highlighting}[]
\KeywordTok{def}\NormalTok{ someGreatFunction():}
    \ControlFlowTok{return} \DecValTok{12}
\NormalTok{bVar }\OperatorTok{=}\NormalTok{ someGreatFunction}
\NormalTok{aVar }\OperatorTok{=}\NormalTok{ bVar()}
\end{Highlighting}
\end{Shaded}
\end{frame}

\begin{frame}[fragile]{MCQ test correction}
\protect\hypertarget{mcq-test-correction-112}{}
Q57. In the following code what is the name of the module ?

\begin{Shaded}
\begin{Highlighting}[]
\ImportTok{from}\NormalTok{ random }\ImportTok{import}\NormalTok{ choice}
\end{Highlighting}
\end{Shaded}
\end{frame}

\begin{frame}{MCQ test correction}
\protect\hypertarget{mcq-test-correction-113}{}
Q57. In the following code what is the name of the module ?

from \textbf{random} import choice
\end{frame}

\begin{frame}[fragile]{MCQ test correction}
\protect\hypertarget{mcq-test-correction-114}{}
Q58. After the following import statement, how can we use the shuffle
function?

\begin{Shaded}
\begin{Highlighting}[]
\ImportTok{from}\NormalTok{ random }\ImportTok{import}\NormalTok{ shuffle}
\end{Highlighting}
\end{Shaded}
\end{frame}

\begin{frame}[fragile]{MCQ test correction}
\protect\hypertarget{mcq-test-correction-115}{}
Q58. After the following import statement, how can we use the shuffle
function?

\begin{Shaded}
\begin{Highlighting}[]
\ImportTok{from}\NormalTok{ random }\ImportTok{import}\NormalTok{ shuffle}

\NormalTok{shuffle()}
\end{Highlighting}
\end{Shaded}

\begin{verbatim}
## shuffle() missing 1 required positional argument: 'x'
\end{verbatim}
\end{frame}

\begin{frame}[fragile]{MCQ test correction}
\protect\hypertarget{mcq-test-correction-116}{}
Q59. After the following import statement, how can we use the randint
function?

\begin{Shaded}
\begin{Highlighting}[]
\ImportTok{from}\NormalTok{ random }\ImportTok{import} \OperatorTok{*}
\end{Highlighting}
\end{Shaded}
\end{frame}

\begin{frame}[fragile]{MCQ test correction}
\protect\hypertarget{mcq-test-correction-117}{}
Q59. After the following import statement, how can we use the randint
function?

\begin{Shaded}
\begin{Highlighting}[]
\ImportTok{from}\NormalTok{ random }\ImportTok{import} \OperatorTok{*}

\NormalTok{randint(}\BuiltInTok{min}\NormalTok{, }\BuiltInTok{max}\NormalTok{)}
\end{Highlighting}
\end{Shaded}

\begin{verbatim}
## unsupported operand type(s) for +: 'builtin_function_or_method' and 'int'
\end{verbatim}
\end{frame}

\begin{frame}[fragile]{MCQ test correction}
\protect\hypertarget{mcq-test-correction-118}{}
Q60. After the following import statement, how can we use the randint
function?

\begin{Shaded}
\begin{Highlighting}[]
\ImportTok{import}\NormalTok{ random}
\end{Highlighting}
\end{Shaded}
\end{frame}

\begin{frame}[fragile]{MCQ test correction}
\protect\hypertarget{mcq-test-correction-119}{}
Q60. After the following import statement, how can we use the randint
function?

\begin{Shaded}
\begin{Highlighting}[]
\ImportTok{import}\NormalTok{ random}

\NormalTok{random.randint(}\BuiltInTok{min}\NormalTok{, }\BuiltInTok{max}\NormalTok{)}
\end{Highlighting}
\end{Shaded}

\begin{verbatim}
## unsupported operand type(s) for +: 'builtin_function_or_method' and 'int'
\end{verbatim}
\end{frame}

\begin{frame}[fragile]{MCQ test correction}
\protect\hypertarget{mcq-test-correction-120}{}
Q61. What does the ``*data'' means in the following code

\begin{Shaded}
\begin{Highlighting}[]
\KeywordTok{def}\NormalTok{ someFun(}\OperatorTok{*}\NormalTok{data):}
    \BuiltInTok{print}\NormalTok{(data)}
\end{Highlighting}
\end{Shaded}
\end{frame}

\begin{frame}[fragile]{MCQ test correction}
\protect\hypertarget{mcq-test-correction-121}{}
Q61. What does the ``*data'' means in the following code

\begin{Shaded}
\begin{Highlighting}[]
\KeywordTok{def}\NormalTok{ someFun(}\OperatorTok{*}\NormalTok{data):}
    \BuiltInTok{print}\NormalTok{(data)}
\end{Highlighting}
\end{Shaded}
\end{frame}

\begin{frame}[fragile]{MCQ test correction}
\protect\hypertarget{mcq-test-correction-122}{}
Q62. What will be the output of this code

\begin{Shaded}
\begin{Highlighting}[]
\BuiltInTok{print}\NormalTok{(}\BuiltInTok{type}\NormalTok{(}\BuiltInTok{type}\NormalTok{(}\BuiltInTok{int}\NormalTok{)))}
\end{Highlighting}
\end{Shaded}
\end{frame}

\begin{frame}[fragile]{MCQ test correction}
\protect\hypertarget{mcq-test-correction-123}{}
Q62. What will be the output of this code

\begin{Shaded}
\begin{Highlighting}[]
\BuiltInTok{print}\NormalTok{(}\BuiltInTok{type}\NormalTok{(}\BuiltInTok{type}\NormalTok{(}\BuiltInTok{int}\NormalTok{)))}
\end{Highlighting}
\end{Shaded}

\begin{verbatim}
## <class 'type'>
\end{verbatim}
\end{frame}

\begin{frame}{MCQ test correction}
\protect\hypertarget{mcq-test-correction-124}{}
Q62. Which of these is not a core data type

\begin{itemize}
\tightlist
\item
  Tuple
\item
  Integer
\item
  Dict
\item
  Datetime
\end{itemize}
\end{frame}

\begin{frame}{MCQ test correction}
\protect\hypertarget{mcq-test-correction-125}{}
Q62. Which of these is not a core data type

\begin{itemize}
\tightlist
\item
  Tuple
\item
  Integer
\item
  Dict
\item
  \textbf{Datetime}
\end{itemize}
\end{frame}

\begin{frame}{MCQ test correction}
\protect\hypertarget{mcq-test-correction-126}{}
Q63. What is the correct way to instantiate an empty dictionary?
(several option possible)

\begin{itemize}
\tightlist
\item
  mydict = dict()
\item
  mydict = \{\}
\item
  mydict = ()
\item
  mydict = dictionary\{\}
\item
  mydict = dict\{\}
\end{itemize}
\end{frame}

\begin{frame}{MCQ test correction}
\protect\hypertarget{mcq-test-correction-127}{}
Q63. What is the correct way to instantiate an empty dictionary?
(several option possible)

\begin{itemize}
\tightlist
\item
  \textbf{mydict = dict()}
\item
  \textbf{mydict = \{\}}
\item
  mydict = ()
\item
  mydict = dictionary\{\}
\item
  mydict = dict\{\}
\end{itemize}
\end{frame}

\begin{frame}{MCQ test correction}
\protect\hypertarget{mcq-test-correction-128}{}
Q65. Which method can be used to separate a string in a list?

\begin{itemize}
\tightlist
\item
  split()
\item
  len()
\item
  cut()
\item
  strip()
\end{itemize}
\end{frame}

\begin{frame}{MCQ test correction}
\protect\hypertarget{mcq-test-correction-129}{}
Q65. Which method can be used to separate a string in a list?

\begin{itemize}
\tightlist
\item
  \textbf{split()}
\item
  len()
\item
  cut()
\item
  strip()
\end{itemize}
\end{frame}

\begin{frame}{MCQ test correction}
\protect\hypertarget{mcq-test-correction-130}{}
Q66. Which statement is used to stop a loop?

\begin{itemize}
\tightlist
\item
  exit
\item
  break
\item
  continue
\item
  return
\end{itemize}
\end{frame}

\begin{frame}{MCQ test correction}
\protect\hypertarget{mcq-test-correction-131}{}
Q66. Which statement is used to stop a loop?

\begin{itemize}
\tightlist
\item
  exit
\item
  \textbf{break}
\item
  continue
\item
  return
\end{itemize}
\end{frame}

\begin{frame}{MCQ test correction}
\protect\hypertarget{mcq-test-correction-132}{}
Q67. Which statement is used to skip an iteration of a loop?

\begin{itemize}
\tightlist
\item
  next
\item
  jump
\item
  continue
\item
  skip
\end{itemize}
\end{frame}

\begin{frame}{MCQ test correction}
\protect\hypertarget{mcq-test-correction-133}{}
Q67. Which statement is used to skip an iteration of a loop?

\begin{itemize}
\tightlist
\item
  next
\item
  jump
\item
  \textbf{continue}
\item
  skip
\end{itemize}
\end{frame}

\begin{frame}{MCQ test correction}
\protect\hypertarget{mcq-test-correction-134}{}
Q68. Which collection data type is ordered, changeable and allows
duplicate items?

\begin{itemize}
\tightlist
\item
  Set
\item
  List
\item
  Tuple
\item
  Dictionnary
\end{itemize}
\end{frame}

\begin{frame}{MCQ test correction}
\protect\hypertarget{mcq-test-correction-135}{}
Q68. Which collection data type is ordered, changeable and allows
duplicate items?

\begin{itemize}
\tightlist
\item
  Set
\item
  \textbf{List}
\item
  Tuple
\item
  Dictionnary
\end{itemize}
\end{frame}

\begin{frame}{MCQ test correction}
\protect\hypertarget{mcq-test-correction-136}{}
Q69. Which of the following options is provided by pdb? (several option
possible)

\begin{itemize}
\tightlist
\item
  Executing the program line by line
\item
  List all the errors in your script
\item
  Checking variables values at any step of the code execution
\item
  Set break points
\end{itemize}
\end{frame}

\begin{frame}{MCQ test correction}
\protect\hypertarget{mcq-test-correction-137}{}
Q69. Which of the following options is provided by pdb? (several option
possible)

\begin{itemize}
\tightlist
\item
  \textbf{Executing the program line by line}
\item
  List all the errors in your script
\item
  \textbf{Checking variables values at any step of the code execution}
\item
  \textbf{Set break points}
\end{itemize}
\end{frame}

\begin{frame}{MCQ test correction}
\protect\hypertarget{mcq-test-correction-138}{}
Q70. What are the possibilities of the logging module in python?
(several option possible)

\begin{itemize}
\tightlist
\item
  Generating more precise error message
\item
  Creating custom log levels according to the ``criticality'' of events
  in your script**
\item
  Preventing your script from crashing
\item
  Basic logging to track variables values
\end{itemize}
\end{frame}

\begin{frame}{MCQ test correction}
\protect\hypertarget{mcq-test-correction-139}{}
Q70. What are the possibilities of the logging module in python?
(several option possible)

\begin{itemize}
\tightlist
\item
  Generating more precise error message
\item
  \textbf{Creating custom log levels according to the ``criticality'' of
  events in your script}
\item
  Preventing your script from crashing
\item
  \textbf{Basic logging to track variables values}
\end{itemize}
\end{frame}

\end{document}
